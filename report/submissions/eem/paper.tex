\documentclass[conference]{IEEEtran}
\usepackage{amsmath,amssymb,amsfonts}
\usepackage{algorithmic}
\usepackage{graphicx}
\usepackage{textcomp}
\usepackage{xcolor}
\usepackage{tabularx, multirow}
\usepackage{fancyhdr,lipsum}

% \IEEEoverridecommandlockouts
% The preceding line is only needed to identify funding in the first footnote. If that is unneeded, please comment it out.
% \fancypagestyle{firstpage}{% Page style for first page
%   \fancyhf{}% Clear header/footer
%   \fancyfoot[l]{}% Footer
% }
% \thispagestyle{firstpage}

% \usepackage{cleveref}

% \crefname{align}{}{}
% \Crefname{align}{Equation}{Equations}

\def\BibTeX{{\rm B\kern-.05em{\sc i\kern-.025em b}\kern-.08em
    T\kern-.1667em\lower.7ex\hbox{E}\kern-.125emX}}
\usepackage[%
backend=biber,bibencoding=utf8, %instead of bibtex
language=auto,
style=ieee,
sorting=none, % nyt for name, year, title
maxbibnames=10, % default: 3, et al.
%backref=true,%
natbib=true % natbib compatibility mode (\citep and~\citet still work)
]{biblatex}
\bibliography{../../references.bib}
% Template from https://de.overleaf.com/latex/templates/ieee-conference-template/grfzhhncsfqn

%define approx proportional
\def\app#1#2{%
  \mathrel{%
    \setbox0=\hbox{$#1\sim$}%
    \setbox2=\hbox{%
      \rlap{\hbox{$#1\propto$}}%
      \lower1.1\ht0\box0%
    }%
    \raise0.25\ht2\box2%
  }%
}
\def\approxprop{\mathpalette\app\relax}

\graphicspath{
    {../../../results/eem/figures/},
}

\begin{document}


\title{Carbon Management Strategies for a Climate-Neutral European Economy}

\thanks{Fabian Hofmann is funded by the Breakthrough Energy Project "Hydrogen Integration and Carbon Management in Energy System Models".}

\author{
    \IEEEauthorblockN{Fabian Hofmann}
    \IEEEauthorblockA{\textit{Institute of Energy Technology} \\
        \textit{Technical University of Berlin}\\
        Berlin, Germany \\
        m.hofmann@tu-berlin.de}
    \and
    \IEEEauthorblockN{Christoph Tries}
    \IEEEauthorblockA{\textit{Institute of Energy Technology} \\
        \textit{Technical University of Berlin}\\
        Berlin, Germany \\
        christoph.tries@tu-berlin.de}
    \and
    \IEEEauthorblockN{Fabian Neumann}
    \IEEEauthorblockA{\textit{Institute of Energy Technology} \\
        \textit{Technical University of Berlin}\\
        Berlin, Germany \\
        f.neumann@tu-berlin.de}
    \and
    \IEEEauthorblockN{Lisa Zeyen}
    \IEEEauthorblockA{\textit{Institute of Energy Technology} \\
        \textit{Technical University of Berlin}\\
        Berlin, Germany \\
        e.zeyen@tu-berlin.de}
    \and
    \IEEEauthorblockN{Tom Brown}
    \IEEEauthorblockA{\textit{Institute of Energy Technology} \\
        \textit{Technical University of Berlin}\\
        Berlin, Germany \\
        t.brown@tu-berlin.de}
}

\maketitle

\begin{abstract}
    Enabling a carbon-neutral European economy requires internationally coordinated investments into all sectors of the energy system. In the light of hard-to-abate emissions from process industry, energy-intensive transport and backup heat or power sources, it is important to determine the extent to which a system-wide carbon management is needed. Different technologies such as direct air capture, carbon capture at point sources and carbon transport, utilization in hydrocarbon synthesis and storage are considered as valuable contributors to formulate a comprehensive carbon strategy and take the final step to climate-neutrality.
    % While the literature has already showed the synergies of these technologies, the question of how to design the infrastructure at high resolution together with a CO$_2$ transport system has not been addressed so far. Moreover, there is a growing number of individually planned carbon management projects whereas the formulation of a European carbon strategy is still missing from the political agenda.
    In this paper, we present the first study of a cost-optimal design of the European energy system with a full representation of carbon management technologies including a CO$_2$ transport network, considering all sectors with high spatial resolution. In addition, the system is allowed to optimize the layout of renewable energy sources as well as energy transmission and storage systems based on hydrogen and methane.
    Assuming net-zero emissions, we show that carbon management technologies provide critical cost-beneficial flexibilities to the European energy system. In particular, we show that building a CO$_2$ network complements the built out of a hydrogen network and is cost-effective across different rates of carbon sequestration in the offshore regions (200 -- 1000 Mt/a). Combined with a hydrogen grid, the CO$_2$ grid leads to cost savings of about 6\%, as it allows industrial and densely populated inland sites to source hydrogen from coastal regions with favorable renewable energy potential and, in return, supply CO$_2$ that can be sequestered or utilized for synthetic hydrocarbons for petrochemical feedstocks, aviation and shipping. At high sequestration rates, direct air capture facilities complement carbon capture technologies at point sources, leading however to higher use of fossil energy carriers in the power and heating sector.
    % The optimization identifies three main carbon sinks that dominate and determine the CO$_2$ network design, namely a large sequestration cluster in the North Sea and two clusters on the coasts of Portugal and Greece.
    The results highlight the potential of a comprehensive carbon management strategy as a basis for designing future carbon networks and identifying the most promising sites for carbon capture and storage.
\end{abstract}

\begin{IEEEkeywords}
    power system analysis computing, power system management, power system planning, renewable energy source
\end{IEEEkeywords}


\section{Introduction}
\label{sec:introduction}

In net-zero economies, carbon capture is considered a key technology for balancing emissions from hard-to-abate sectors and inevitable emissions from land use and agriculture. Capturing carbon at point sources and from the atmosphere by direct air capture (DAC) opens up new opportunities for producing drop-in fuels such as methane, methanol, or carbon-neutral liquid hydrocarbons, known as Carbon Capture and Utilization (CCU), and for storing carbon in geological formations, known as Carbon Capture and Storage (CCS). Together with carbon transport, these technologies are considered to be part of what is called the New Carbon Economy~\cite{arniehellerNewCarbonEconomy2019}.

The literature has extensively discussed the individual potentials of carbon management technologies~\cite{burandtDecarbonizingChinaEnergy2019,caprosEnergysystemModellingEU2019,damoreOptimalDesignEuropean2021,larsonNetZeroAmericaPotential2021,maModelingOptimizationCombined2021,mikulcicFlexibleCarbonCapture2019,williamsCarbonNeutralPathwaysUnited2021}.
% CC

% CCU

% CCS
The Capture Map in~\cite{ToolsGreenTransition} estimates a potential of 1.4~Gt of carbon capture from point sources in Europe. A sequestration potential study in~\cite{weiProposedGlobalLayout2021} suggests that 92 GtCO$_2$ can be mitigated globally, of which 59 Gt can be sequestered. A recent study on carbon capture, transport and storage in~\cite{CaptureMapGetSitespecific} has shown that

% combined analysis
Finally, optimization tools have been developed in recent years to model synergies between different carbon management technologies and support infrastructure decisions~\cite{bakkenLinearModelsOptimization2008,morbeeOptimisedDeploymentEuropean2012,oeiModelingCarbonCapture2014,elahiMultiperiodLeastCost2014,middletonSimCCSOpensourceTool2020,bjerketvedtOptimalDesignCost2020,weiProposedGlobalLayout2021,damoreOptimalDesignEuropean2021,becattiniCarbonDioxideCapture2022}. In contrast to Integrated Assessment Models these consider the spatial distribution of carbon sources and sink. Among these, the recent work in~\cite{becattiniCarbonDioxideCapture2022} presents a comprehensive mixed-integer model to optimize the time-evolution of a CO$_2$ transport system in Switzerland with a connection to a remote sequestration site in Norway.


Recently, policymakers and industry in Europe have been committing to carbon management strategies, planning the first infrastructure components, and developing business models for emerging sectors of the economy~\cite{adomaitisEquinorRWEBuild2023,apnewswireGermanyDrawLegislation2023,KohlenstoffKannKlimaschutz2023,OGETESJoin2022,TESHydrogenLife2023}. Business models from companies like Tree Energy~\cite{TESHydrogenLife2023}, Carbfix [cite], and Equinor~\cite{adomaitisEquinorRWEBuild2023} advertise carbon management hubs that provide green hydrogen, methane, or synfuels on the one hand and offtake CO$_2$ on the other hand. ...
% TODO other businesses, sequestration business models
Despite the growing number of individually planned carbon management projects, an internationally coordinated carbon management strategy is still missing from the political agenda.

% Hydrogen
% Germany's climate protection act~\cite{KlimaschutzgesetzKlimaneutralitaetBis} targets a 65\% reduction in CO$_2$ emissions by 2030 and climate neutrality by 2045. Similarly, the EU's Fit-for-55 package aims for a 55\% reduction in emissions by 2030. To accelerate the energy transition and ensure energy security, REpowerEU~\cite{REPowerEU} proposes a fast ramp-up of a green Hydrogen system, with a domestic production target of 10 MT H$_2$ by 2030. Potential carbon management hubs such as Tree Energy, Carbfix, and Equinor are emerging as industry leaders.
% The Mission Innovation Hydrogen Valley Platform~\cite{H2ValleysMissionInnovation} is a platform that collects hydrogen flagship projects to facilitate collaboration and knowledge sharing. Finally, the European Hydrogen Backbone~\cite{gasforclimateEuropeanHydrogenBackbone2022} is a project that aims to develop a cross-border hydrogen infrastructure network to enable the transport of renewable energy across Europe.



However, to formulate such a international strategy, it is important to carefully assess the benefits of carbon infrastructure in geographic terms and in terms of interaction with the energy system. As we show in our study, the dynamics of carbon capture, transport, use, and storage in different regions are highly dependent on the availability of renewable energy, carbon pipelines and sequestration sinks. To our knowledge, there has never been a study that has examined carbon management with a detailed geographic representation for the future European energy system.

In this paper, we therefore present a detailed study of the European energy system for 2050, which includes high geographical resolution and a comprehensive representation of carbon management technologies. The study is conducted using the PyPSA-EUR-Sec energy system model and encompasses all relevant sectors. We enable the system to optimize the design of renewable energy sources and storage technologies, as well as the transmission of electricity, hydrogen, methane, and carbon. Our evaluation focuses on the transport dynamics of carbon and hydrogen through their respective networks on the European continent. We also analyze how an energy system with limited annual sequestration potential prioritizes decarbonization and fuel switching in various sectors, and how the construction of carbon networks varies based on different levels of available carbon sequestration potential.


\section{Methodology}
\label{sec:methodology}

To model the detailed expansion of the carbon network, we use the open-source PyPSA-Eur-Sec~\cite{PyPSAEurSecSectorCoupledOpen2023} model. This is a fully sector-coupled model with high spatial and temporal resolution and detailed transmission infrastructure for electricity, hydrogen, methane, and carbon networks. The model enables joint optimization of investment and operational decisions for generation, storage, conversion, and transmission infrastructures. For our study, we use a model representation with 90~geographic regions for Europe and 4~hours time resolution to capture the dynamics of material flow between carbon sources and sinks while ensuring computational feasibility.

\begin{figure}
    \centering
    \includegraphics[width=\linewidth]{sequestration_map.pdf}
    \caption{Sequestration map}
    \label{fig:sequestration_map}
\end{figure}

PyPSA-Eur-Sec uses exogenous assumptions for the regional energy demand of the power sector, industry, transport, buildings and agriculture. This includes energy demand for shipping and aviation, and non-energy feedstocks for the chemical industry. To take account of climate neutrality targets, we have set a binding upper limit for zero emissions in line with the EU's 2050 target. Biomass use in each region is limited by its biomass potential, but biomass transport between regions is allowed and the model accounts for transport costs. Furthermore, we limit the expansion of the electric transmission system to 20\% of today's total capacity to account for the difficulties in placing new transmission projects. The geographical layout of process emissions from industry is derived from~\cite{piamanzGeoreferencedIndustrialSites2018}. These include emissions from the cement, chemical, paper and printing, glass and steel industry as well as refineries and amount 158~Mt CO$_2$ per year in total.
For carbon sequestration, only offshore sites are considered as potential sinks (see Figure~\ref{fig:sequestration_map}). This is partly because offshore storage potential tends to be greater than onshore potential, and partly because carbon storage infrastructure projects near population centers may be difficult to implement due to public safety concerns. We make conservative assumptions in estimating storage potential by truncating the total potential to 25~Mt per site and calculate annual storage availability when filled over 25 years.
On the carbon utilization side, we include three drop-in fuel production technologies in our model: Steam Methane Reforming (SMR), Methanolization, and Fischer-Tropsch (FT) synthesis. After processing, the fuels are made available for free use and can transported accross regions without additional costs. Synthetic methane can be used as a substitute for natural gas or biogas and is used in combined heat and power (CHP) plants, in gas boilers for home heating, or to meet the gas needs of industry. Synthetic methanol is used in the model to decarbonize fuel demand in the marine industry. Finally, FT fuels can replace fossil oil in the production of naphtha for industry and kerosene for aviation, or be used as machinery oil for agriculture.
All technology cost assumptions for 2050 are listed taken from~\cite{lisazeyenPyPSATechnologydataTechnology2023}.

To explore the impact of carbon sequestration on the optimal layout of carbon management technologies, we vary the global carbon sequestration potential from 200~Mt to 1000~Mt per year in steps of 200~Mt.


\section{Results}
\label{sec:results}

Figure~\ref{fig:capacity_map_carbon_co2network_1000} shows the optimal capacities of carbon capture and transport technologies in the European sector-coupled system for a total sequestration of 1000~Mt/a. The colors of the offshore regions are reflecting the optimal sequestration capacity per region. As shown, the optimization identifies three main carbon sinks that dominate and determine the CO$_2$ network design, namely a large sequestration cluster in the North Sea and two clusters on the coasts of Portugal and Greece. The CO$_2$ network is primarily built around the North Sea, connecting onshore carbon capture facilities to the coast. At the same time, large SMR facilities with CC are placed close to sequestration areas.


\begin{figure}[h]
    \centering
    \includegraphics[width=\linewidth]{capacity_map_carbon_co2network_1000.pdf}
    \caption{Optimal capacities per of the carbon sector for a sequestration of 1000 Mt/a. All technologies with an "*" are explicitely augmented technologies to capture carbon emissions.}
    \label{fig:capacity_map_carbon_co2network_1000}
\end{figure}



\begin{figure}[h]
    \centering
    \includegraphics[width=\linewidth]{capacity_map_electricity_co2network_1000.pdf}
    \caption{Optimal capacities per of the electricity sector for a sequestration of 1000 Mt/a.}
    \label{fig:capacity_map_electricity_co2network_1000}
\end{figure}


\begin{figure}[h]
    \centering
    \includegraphics[width=\linewidth]{capacity_map_hydrogen_co2network_1000.pdf}
    \caption{Optimal capacities per of the hydrogen sector for a sequestration of 1000 Mt/a.}
    \label{fig:capacity_map_hydrogen_co2network_1000}
\end{figure}



\begin{figure*}
    \centering
    \includegraphics[width=\linewidth]{operation_map_carbon_co2network_1000.pdf}
    \caption{Optimal operation per sector for a sequestration of 1000 Mt/a.}
    \label{fig:operation_map_noco2network_1000}
\end{figure*}


\begin{figure}
    \centering
    \includegraphics[width=\linewidth]{operation_area_carbon_co2network.pdf}
    \caption{Balance of captured CO$_2$ emissions for the optimal operation with CO$_2$ network.}
    \label{fig:operation_area_carbon_co2network}
\end{figure}

\begin{figure}
    \centering
    \includegraphics[width=\linewidth]{operation_area_carbon_noco2network.pdf}
    \caption{Balance of captured CO$_2$ emissions for the optimal operation without CO$_2$ network.}
    \label{fig:operation_area_carbon_noco2network}
\end{figure}


\begin{figure}
    \centering
    \includegraphics[width=\linewidth]{operation_area_hydrogen_co2network.pdf}
    \caption{Balance of H$_2$ production and consumption with CO$_2$ network.}
    \label{fig:operation_area_carbon_co2network}
\end{figure}

\begin{figure}
    \centering
    \includegraphics[width=\linewidth]{operation_area_hydrogen_noco2network.pdf}
    \caption{Balance of H$_2$ production and consumption without CO$_2$ network.}
    \label{fig:operation_area_carbon_noco2network}
\end{figure}


\section{Limitations}
\label{sec:limitations}
Economy of scale? Cost projections? Other uncertainties?
Industry processes not endougenous.

\section{Conclusion}
\label{sec:conclusion}

\section*{Acknowledgements}
\label{sec:acknowledgements}


\newpage

\printbibliography

\appendix

Here come the appendices


\end{document}

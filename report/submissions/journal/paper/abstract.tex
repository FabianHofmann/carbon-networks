    The transition to a carbon-neutral European economy is a pressing challenge that demands coordinated action across various energy sectors, particularly in emissions-intensive industries like heavy manufacturing. While management of both \hydrogen{} and \carbon{} is considered a critical component of this transition, a gap exists in understanding how new hydrogen infrastructures effectively interact with comprehensive carbon management technologies, including carbon capture, transport, use, and storage. In particular, the role of a carbon transport network as a complementing or substituting infrastructure to hydrogen transport remains unclear. To address this gap, our study employs optimization techniques to design the first cost-optimal European energy system that fully incorporates carbon management technologies, hydrogen transport and storage, and renewable energy sources with a high spatial and temporal resolution considering all energy intensive sectors.
    Our findings reveal that in a either-or scenario, the hydrogen network is superior to the carbon network as it enables the transportation cheap hydrogen inland to serve industrial processes and carbon capture and utilization at point sources. A carbon network facilitates point-source carbon capture and reduces the need for direct air capture, but leaves the scarcity of hydrogen in energy-intensive regions such as industrial clusters.
    However, in a hybrid scenario, the carbon network is a cost-effective complement to the hydrogen network facilitating carbon capture at point sources and transport to sequestration sites.
    We show that theses findings holds true against the backdrop of a net-zero energy system, as well as scenarios where the net-negative emissions are 5\% and 10\% lower than the 1990 baseline levels.
    Overall, our work demonstrates the cost-effectiveness of a multi-grid system that includes both hydrogen and carbon transport networks and power grid expansion to achieve climate neutrality in Europe.
    The paper underpins the need for a complementing hydrogen network and \carbon{} network to achieve climate neutrality.


% LEFT OVER
% =========

% To understand the dynamics of the \carbonscenario{}, we observe that the system locates CU in the regions where it can minimize costs for its material inputs, carbon and hydrogen. As the \carbon{} network largely smoothes out carbon prices across regions, CU is located in the hydrogen price valleys on the periphery of the continent. Similarly, the model locates CS in regions where it can minimize the costs of the carbon to be sequestered. Thus, CS is located in the carbon price valleys among the subset of regions with access to offshore sequestration sites. Due to the \carbon{} network these low-price regions are not strongly pronounced, but still prevalent and impact CS locations. DAC is only used for CU and thus co-located in regions with a mix of low electricity costs (for running DAC) and low hydrogen costs.

% To explain the dynamics of the \hydrogenscenario{}, we observe that the model locates CU in the carbon price valleys across Western, Central and Eastern Europe (see top right), as the \hydrogen{} network smoothes out hydrogen prices across the entire continent (see bottom right). CS is again located in the carbon price valleys among those regions with access to offshore sequestration sites. Due to the lack of a \carbon{} network the full CC potentials from point sources are exhausted in all offshore sequestration regions, and the carbon is then sequestered. The additionally needed carbon to reach 200~Mt of CS is captured via DAC in the regions with access to offshore sequestration sites with the lowest costs for DAC, with the largest share deployed in Portugal (see top right). DAC is also used for CU in this model, and for both purposes DAC is located in the regions with the lowest costs for electricity.


% The increase in captured carbon from point sources in the models with \carbon{} and/or \hydrogen{} transport is more related to an upgrading of existing industrial or power plant facilities to capture carbon rather than a shift in technologies. This is evident when analyzing the capture share of the different CC technologies (see Fig.~\ref{fig:captureshare_line}).



% In summary, the implementation of \carbon{} and \hydrogen{} networks is instrumental in unlocking CC potentials and reducing system costs. However, despite the \carbonscenario{} demonstrating higher amounts and shares of CC from point sources and therefore less DAC deployment, the \hydrogenscenario{} achieves even  lower total system costs (see Fig.~\ref{fig:cost_bar}).
%
%
% Comparison carbon and hydrogen network
%
% [Dynamics \carbonscenario{} vs \hydrogenscenario{} (old text blocks)]
% There are four notable dynamics or differences between the two network strategies:
% First, the production of FT fuels always moves to where low-cost carbon and low-cost hydrogen can be brought together. In the \carbonscenario{ }, carbon is moved from Western Europe to the Iberian Peninsula, the British Isles and to Denmark to produce FT fuels centrally in few regions. In the \hydrogenscenario{}, hydrogen is transported in the opposite direction to produce FT fuels decentrally across all regions in Central Europe with locally captured \carbon{} from point sources.
% Second, the \carbonscenario{ } captures more low-cost carbon from biomass, especially in Central and Eastern Europe, and less high-cost carbon from DAC in Greece and Southern Italy. Instead, the \carbon{} network transports carbon from Central and Eastern Europe to Greece and Italy, where it is sequestered, or feeds into CU together with locally produced, low-cost electrolytic hydrogen. The \hydrogenscenario{} directly operates electrolyzers and DAC in Greece and Southern Italy, with electricity from low-cost renewables, and thus avoids significant network buildouts in this region.
% Third, sequestration in the \hydrogenscenario{} occurs more decentralized, spread out across several regions with access to offshore sequestration sites, making use of point-source CC potentials in each of these regions. In the \carbonscenario{ }, sequestration is clustered at fewer sequestration sinks, each collecting low-cost \carbon{} from neighboring regions. However, there is no centralization at only one or two sites, since all the sequestration sites offer the same costs and the system minimizes system-wide carbon transport distances to about eight to ten sequestration sites.
% And fourth, how the three types of CU are spread across the continent is driven by the costs of carbon and hydrogen (the latter mostly driven by the costs of renewable electricity) as well as by renewables capacity factors: Clearly, all carbonaceous fuel production requires sites where both carbon and hydrogen are available, and they moreover prefer sites with low carbon and hydrogen prices. Based on our model assumptions, FT fuel synthesis has very limited flexibility and needs to be run alost at baseload (ramp-down only possible to 90\% of maximum load). Thus, for FT fuel synthesis sites in Northern Europe with more electricity from wind power are preferred to sites in Southern Europe, powered mostly by solar. Methanathion and Methanolisation both have higher flexibility, being able to ramp down to 50\% load during the night, for example. Thus, these fuel production sites prefer the best renewable resource sites in Southern Europe. And finally, as methanation produces synthetic methane that needs to be fed into the gas grid, methanation prefers those sites with good interconnection to the existing gas network. Since Italy possesses a large gas network with easy interconnection to Central Europe, methanation prefers sites in Southern Italy (and Greece, which would be connected to Southern Italy) over the Iberian Peninsula, where Methanolisation is more prevalent.

% In summary, the implementation of a \carbon{} network allows for connecting regions with abundant, low-price carbon (from point sources) to regions that can sequester the carbon or that can produce low-cost hydrogen to make CU products (FT fuel, methanol, synthetic methane). DAC and electrolysis are deployed in the regions with the best renewable resources to meet carbonaceous fuel demands.
% In the absence of a \carbon{} network and with an \hydrogen{} network on the other hand, regions with access to offshore sequestration sites capture all carbon from low-cost point sources and then sequester it. The system deploys additional DAC facilities at the locations with sequestration access and low-cost renewables to capture the remaining necessary carbon for sequestration. Carbonaceous fuel demand is met at the locations which offer the best mix of low-cost carbon from point sources and low-cost hydrogen from renewables, with some additional electrolysis and DAC required to fully meet the demand.
% Sequestered carbon is nearly always transported before

% Therefore, one can conclude that a \carbon{} network favors a decentralized CC and a CU centralized at regions with abundant renewable resources. An \hydrogen{} network favors a centralized electrolysis system at regions with abundant renewable resources and a decentralized CU system.

% Together with understanding the dynamics described above, we can highlight three cost advantages of the \hydrogen{} network in contrast to the \carbon{} network: Capturing carbon at the lowest-cost regions reduces 4~bn€ in spending on DAC. However, the \hydrogenscenario{} produces electrolytic hydrogen in the lowest-cost regions and in total spends 7~bn€ less on renewable energy (14~bn€ less on solar, and 7~bn€ more on wind) as well as 1~bn€ less on SMR. Furthermore, without an \hydrogen{} network, additional \hydrogen{} storage is needed in Spain at a cost of 5~bn€ to store hydrogen produced during the summer for utilization in FT fuel production during the winter.
% And finally, the \carbonscenario{ } spends 7~bn€ more on gas boilers, gas plants, gas infrastructure, and methanation.
% TODO: why??
% The costs of the respective networks, on the other hand, are almost equal: the \carbonscenario{ } spends 8~bn€ on the \carbon{} network, while the \hydrogenscenario{} even spends 10~bn€ on the \hydrogen{} network. These factors add up to account for 14~bn€ of the difference of 20~bn€ in system costs between the two scenarios (785~bn€ for the \COgrid{} and 765~bn€ for the \hydrogenscenario{}, see Fig.~\ref{fig:objective_heatmap}).


% Net-negative

% Fundamental differences in investment strategies occur in the deployment of solar, carbon capture and bio-energy technologies: While the \carbonmodel{} expands CC at point sources the most, the \hydrogenmodel{} and the \hybridmodel{} invest relatively more in new solar plants. The \carbonmodel{} invests three time more in new bio-energy sources (6~bn€) than the other models.
% All models only reveal small net downscaling of technologies. However, as we discuss below, the geographic impact can vary depending on the model, sometimes leading to disruptive changes in optimal network strategies and stranded assets.


% In contrast to its layout in the Net-Zero scenario, the \carbonmodel{} fundamentally reorganizes the carbon transport pathways (see Fig.~\ref{fig:balance_maps_co2_nn}).
% In Spain, carbon utilization (CU) is now predominantly supplied by carbon extracted from DAC. Central Europe sees a major transformation, with carbon mainly being transported to sequestration sites in the north and south-east. This reorganization leads to stranded investment in the carbon network infrastructure, amounting to approximately 3~bn€.

% The \hydrogenscenario{} on the other hand, primarily leaves the network topology unchanged and only expands DAC facilities at regions with access to sequestration sites as well as additional solar, wind and heat pumps in the vicinity to supply power and heat (see Fig.~\ref{fig:balance_maps_h2_nn}).

% Finally, the \hydrogenscenarioybrid{} expands the sequestration with carbon captured from bio-energetic inputs and transported from Central and Eastern Europe to the near shores (see Fig.~\ref{fig:balance_maps_full_nn}), leading to an additional investment in the \carbongrid{} (1.2~bn€, see Fig.~\ref{fig:cost_bar_transmission_nn}). At the same time, carbonaceous fuel production in Central Europe partially moves to Spain, where it uses carbon from new DAC facilities without the installation of a \carbon{} network. As a trade-off, the model transports less hydrogen from the Iberian Peninsula to Central Europe, resulting in a dismantling of 2~bn€ in hydrogen network costs compared to the Net-Zero scenario.

% These findings highlight the continued trends of the two systemic approaches which are combined in the \hybridmodel. With an increased demand for sequestration, the \carbongrid{} relatively gains more importance, leading to higher cost-benefits by enabling to combine increased need DAC with a more dominant collection and transportation scheme of carbon from distributed point sources.

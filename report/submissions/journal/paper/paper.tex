\documentclass[twocolumn]{article}
\usepackage{amsmath,amssymb,amsfonts}
\usepackage{cuted}  % Add the cuted package
\usepackage{caption}
\captionsetup{font=small}
\usepackage{algorithmic}
\usepackage{graphicx}
\usepackage{textcomp}
\usepackage{xcolor}
\usepackage{tabularx, multirow}
\usepackage{fancyhdr,lipsum}
\usepackage{subcaption}
\usepackage[shortcuts]{extdash}

\usepackage[%
backend=biber,bibencoding=utf8, %instead of bibtex
language=auto,
style=ieee,
sorting=none, % nyt for name, year, title
maxbibnames=10, % default: 3, et al.
%backref=true,%
natbib=true % natbib compatibility mode (\citep and~\citet still work)
]{biblatex}
\bibliography{../../../references.bib}

%define approx proportional
\def\app#1#2{%
  \mathrel{%
    \setbox0=\hbox{$#1\sim$}%
    \setbox2=\hbox{%
      \rlap{\hbox{$#1\propto$}}%
      \lower1.1\ht0\box0%
    }%
    \raise0.25\ht2\box2%
  }%
}
\def\approxprop{\mathpalette\app\relax}

% make abbreviation for co2
\newcommand{\carbon}{CO$_2$}
\newcommand{\CO}{CO$_2$}
\newcommand{\hydrogen}{H$_2$}
\newcommand{\Htwo}{H$_2$}
\newcommand{\carbongrid}{\carbon{}--Grid}
\newcommand{\hydrogengrid}{\hydrogen{}--Grid}
\newcommand{\modBase}{Baseline model}
\newcommand{\modCO}{CO$_2$-Grid model}
\newcommand{\modH}{H$_2$-Grid model}
\newcommand{\modHybrid}{Hybrid model}

\graphicspath{
    % {paper-figures}
    {figures}
    {../../../../results/}
    {results/}
}

\begin{document}


% FORMATTING:
% - https://www.nature.com/nenergy/submission-guidelines/aip-and-formatting
% - https://www.nature.com/nenergy/content abstract: 150 words, main text: 3000 words, figures: 8

\title{Assessing \hydrogen{} and \carbon{} Network Strategies in Europe}
% \title{Synthetic fuels in Europe: Transport Hydrogen to Carbon, or Carbon to Hydrogen?}

\author{
    Fabian Hofmann, Christoph Tries, Fabian Neumann, Lisa Zeyen, Tom Brown \\
    \textit{Institute of Energy Technology} \\
    \textit{Technical University of Berlin}\\
    Berlin, Germany \\
    m.hofmann@tu-berlin.de
}


\maketitle

\begin{abstract}
    % Hydrogen and carbon dioxide transportation in climate-neutral energy systems fulfills distinct yet connected roles: Hydrogen transport allows energy-intensive areas to import low-cost hydrogen, while carbon transportation allows high-emitting regions to capture and export carbon for sequestration or re-use. The synergy or competition of hydrogen and carbon transport is yet to be determined.
    % To address this gap, our study employs optimization techniques to design the first cost-optimal European energy system that fully incorporates transport, storage and sequestration for carbon dioxide, transport and storage for hydrogen, and renewable energy sources with a high spatial and temporal resolution considering all energy intensive sectors.
    % Our research shows that in direct comparison a hydrogen network is more cost-effective than a carbon network. It allows regions across Europe to import low-cost hydrogen from centralized production sites, supplying local hydrogen demand and enabling local carbon capture and utilization. A carbon network facilitates point-source carbon capture and reduces the need for direct air capture, but leaves the scarcity of hydrogen in energy-intensive regions such as industrial clusters.
    % However, in a hybrid scenario, the carbon network is a cost-effective complement to the hydrogen network facilitating carbon capture at point sources and transport to sequestration sites.
    % In a hybrid model, the carbon network is a cost-effective complement to the hydrogen network, unlocking carbon capture potentials from biomass with subsequent transport to sequestration and leading to a reduced deployment of direct air capture.
    % When strengthening the emission reduction target to net-negative realm, additional carbon removal is achieved by expanding direct air capture at coastal regions with favorable renewable resources and sequestration potentials.
    % Our research underscores the efficiency of integrating hydrogen, carbon transport networks, and power grid expansion in infrastructure planning for achieving climate neutrality in Europe.

    % Condensed version with 150 words
    Hydrogen and carbon dioxide transportation are considered crucial in climate-neutral energy systems, with hydrogen enabling energy imports to high-demand areas and carbon transport aiding in emissions export from high-emission areas. Yet, possible synergies and competitions between the two systems are not fully understood. Our study employs optimization techniques to develop a cost-optimal European energy system, integrating transport, storage, and sequestration of both carbon dioxide and hydrogen, along with renewable energy sources. Results indicate that a hydrogen network is more cost-effective than a carbon network, facilitating low-cost hydrogen imports and local carbon capture and utilization. However, in a hybrid scenario, the carbon network effectively complements the hydrogen network, promoting carbon capture from biomass and reducing reliance on direct air capture. This research demonstrates the effectiveness of combining hydrogen and carbon transport networks with power grid expansion in achieving climate neutrality in Europe.
\end{abstract}


\section{Introduction}

The transition to a carbon-neutral European economy is a pressing challenge that demands coordinated action across various energy sectors. While management of both \carbon{} and \hydrogen{} is considered a critical component of this transition, a gap exists in understanding how new hydrogen infrastructures effectively interact with comprehensive carbon management technologies, including carbon capture, transport, use, and storage. Hydrogen offers an efficient way to transport and store energy. Carbon, on the other hand, can be effectively captured from industrial processes and the combustion of biomass, fossil fuels, and synthetic fuels through carbon capture (CC) techniques, or harvested from the atmosphere using direct air capture (DAC). Additionally, it can be stored in geological formations, a process known as carbon sequestration (CS). In combination, \carbon{} and \hydrogen{} networks build the basis for climate-neutral fuels needed for aviation, shipping and industrial feedstocks, and are central to carbon utilization (CU) strategies.

Recently, policymakers and industry in Europe have started developing carbon management strategies~\cite{GermanyDevelopingStrategy2023,CarbonManagementStrategie}, planning infrastructure components~\cite{CONetz}, and committing on the first carbon utilization projects~\cite{EFuelsPilotPlant2022,OrstedAssumesFull,GROUNDBREAKINGEFUELPRODUCTION,DLREfuelsDLR}. With the European Union's goal of achieving climate neutrality by 2050, a wide range of programs, funding models and initiatives have been established in this regard~\cite{eu2023netzero,europeangreendeal,europeaninnovationfund}. Initiatives like the European Hydrogen Backbone~\cite{gasforclimateEuropeanHydrogenBackbone2022} or the Hydrogen Infrastructure Map~\cite{H2InfrastructureMap} showcase the potential of hydrogen as a fuel and energy carrier, and some gas pipelines are already repurposed to transport hydrogen~\cite{RohrFreiFuer}. At the same time, business models from companies like Tree Energy Solutions~\cite{TESHydrogenLife2023}, Carbfix~\cite{WeTurnCO2}, and Equinor~\cite{adomaitisEquinorRWEBuild2023} advertise carbon management hubs that provide green hydrogen, methane, and synfuels on the one hand and offer \carbon{} offtake on the other hand. The Northern Lights project in Norway~\cite{NorthernLightsWhat} is planning with a transport and sequestration capacity of 1.5 Mt \carbon{} per year  to be operative in 2024, expanding to a targeted scale of 5 Mt per year of sequestration by 2030.
% TODO: frame next two sentences to be more market oriented
For Europe, the Capture Map~\cite{ToolsGreenTransition} estimates a potential of 1.7~Gt of carbon capture from point sources per year, which represent roughly 50\% of all emissions. In combination with large sequestration potentials as stated in~\cite{weiProposedGlobalLayout2021}, this highlights the vast potential for decarbonization.
To this end, the Clean Air Task Force underlines the importance of a carbon transport system in Europe to facilitate the carbon economy~\cite{lockwoodEuropeanStrategyCarbon}.


However, up to this point, it remains unclear how the two transport systems of hydrogen and carbon may complement or replace each other, particularly when it comes to their transport systems. A hydrogen network facilitates the transport of energy from renewable sources to regions with geographically fixed hydrogen demand, such as for steelmaking, and also enables CU at the site of point-source capture. On the other hand, the carbon transport system allows for capturing and transporting carbon from areas with high emissions to regions with significant sequestration potential and abundant renewable sources, thereby enabling cost-effective CU.
In the literature, the two network approaches and underlying technologies have been discussed in a number of publications, all of which, however, dealt with the isolated effects~\cite{bakkenLinearModelsOptimization2008,morbeeOptimisedDeploymentEuropean2012,stewartFeasibilityEuropeanwideIntegrated2014,oeiModelingCarbonCapture2014,elahiMultiperiodLeastCost2014,burandtDecarbonizingChinaEnergy2019,middletonSimCCSOpensourceTool2020,bjerketvedtOptimalDesignCost2020,weiProposedGlobalLayout2021,damoreOptimalDesignEuropean2021,becattiniCarbonDioxideCapture2022}. Such techno-economic models, in contrast for example to Integrated Assessment Models, can account for the spatial distribution of carbon sources and sinks which are crucial provide a holistic view of the energy system and its technological interactions. The work in~\cite{neumannBenefitsHydrogenNetwork2022} examines the effect of a hydrogen network in Europe... (extend on hydrogen literature)
The publication by Morbee et al.~\cite{morbeeOptimisedDeploymentEuropean2012} optimizes the topology and capacity of a \carbon{} network in Europe, but only considers the power sector without co-optimizing renewable deployment. This limited sectoral scope cannot capture important dynamics of carbon management, since it neglects the sectors that will need to handle most \carbon{} in the future.
Another comprehensive example is found in~\cite{becattiniCarbonDioxideCapture2022}, which presents a mixed-integer model to optimize the time-evolution of a \carbon{} transport system in Switzerland, connecting to a remote sequestration site in Norway. Hoewever, this limited spatial scope fails to consider other sequestration sites and co-benefits from connecting the \carbon{} network to neighboring countries. To this end, it is often argued that \carbon{} pipelines are a mature technology with an expected high learning rate, given the wide-spread installations in US and Canada for enhanced oil recovery~\cite{righettiSitingCarbonDioxide2017,friedmannNETZEROGEOSPHERICRETURN}.
% However, the models are often limited with regard to both geographical scope and detail. While representing a single country with spatial resolution may neglect synergies of international cooperation, a coarse grained representation of multiple countries may neglect important geographical properties.

To our knowledge, no study has yet considered the co-optimization and comprehensive assessment of both \carbon{} and \hydrogen{} networks in a fully sector-coupled energy system. However, we argue that such an assessment is strongly needed to avoid suboptimal investments and to identify synergies between hydrogen and carbon management technologies. In this paper, we present a detailed study of the European energy system for 2050, which includes high geographical resolution and a comprehensive representation of carbon management technologies. The study is conducted using the PyPSA-Eur energy system model and encompasses all relevant energy sectors. We explore competition between \carbon{} and \hydrogen{} networks by adding them separately. Our evaluation focuses on the transport of \carbon{} and \hydrogen{} through their respective networks on the European continent. We also analyze how an energy system with limited annual sequestration potential prioritizes decarbonization and fuel switching in various sectors, and how the construction of carbon networks varies based on different levels of available sequestration potential.


\section{Methodology}
\label{sec:methodology}

The study is conducted on the basis of the open-source, capacity-expansion model PyPSA-Eur~\cite{horschPyPSAEurOpenOptimisation2018,brownSynergiesSectorCoupling2018,PyPSAEurSecSectorCoupledOpen2023}.
The model optimizes the design and operation of the European energy system, encompassing the power, heat, industry, waste, agriculture, and transport sectors, including international aviation and shipping.

\begin{figure}
    \includegraphics[width=\linewidth]{baseline/figures/90_nodes/total-demand-bar.png}
    \caption{Exogenous demand assumptions. The figure shows the total annual energy demands for each energy carrier which drive all model activities. Endogenous processes may lead to higher total production volumes of some carriers, e.g., demand in methanol requires more hydrogen and carbon as secondary (energy) inputs, which in turn need additional electricity to run electrolysis and carbon capture processes.}
    % TODO: adjust labels to show that these are exogenous assumptions
    \label{fig:total-demand-bar}
\end{figure}
%
In our configuration, the model's time horizon spans one year with a temporal resolution of 3 hours and a spatial resolution of 90 regions. Each of the regions consists of a complex subsystem with technologies for supplying, converting, storing and transporting energy. Exogenous assumptions on energy demand and non-abatable emissions are taken from various sources~\cite{piamanzGeoreferencedIndustrialSites2018,muehlenpfordtTimeSeries2019,mantzosJRCIDEES20152018,NationalEmissionsReported2023,EurostatCompleteEnergyBalance,uwekrienDemandlib2023}. The energy demand for electricity, transport, biomass, heat and gas are defined per region and time-step. Land transport demand is exogenously divided between electric vehicles (85\%) and fuel cell vehicles (15\%), the latter representing demand for heavy duty land transport. Demands for kerosene for aviation, methanol for shipping, and naphtha for industry are aggregated in the system scope and kept constant throughout all time-steps. Heat demand is regionally subdivided into shares of urban, rural and industrial sites. We show the sum of all energy demands in Fig.~\ref{fig:total-demand-bar}. The system emits 633~Mt \carbon{} per year from industry, aviation, shipping and agriculture, 153~Mt of which are fossil-based process emissions. Industrial energy demand and excess heat potentials are calculated per node on the basis of~\cite{hotmaps_industrial_db}.
% use https://pypsa-eur.readthedocs.io/en/latest/licenses.html


Endogenous model results include the expansion of renewable energy sources, storage technologies, and transmission capacities.
The model considers various energy carriers like electricity, hydrogen, methan, methanol, liquid hydrocarbons and biomass, together with conversions technologies.
Carbon-neutral electricity is provided by wind, solar, biomass, hydro and nuclear power plants. Hydro and nuclear plant capacities cannot be extended. Weather-dependent power potentials for solar, wind and hydro are calculated from reanalysis and satellite data sets~\cite{hersbachERA5GlobalReanalysis2020,pfeifrothSurfaceRadiationData2017}  per region and time-stamp, using the open-source tool Atlite~\cite{hofmannAtliteLightweightPython2021}.
Solar and wind power can be expanded in alignment with eligible land-use restrictions calculated on the basis of~\cite{eeaCorineLandCover2012,eeaNatura2000Data2016}. We restrict the electric transmission system's expansion to 20\% of its current capacity, acknowledging the challenges in inaugurating new transmission projects.
For the use of biomass we consider only residual biomass products and no energy crops. We limit regional biomass use to the medium-level potentials from the ENPRESO database~\cite{enspreso_database,instituteforenergyandtransportjointresearchcentreJRCEUTIMESModelBioenergy2015}. Inter-regional biomass transport is permitted with transport costs considered.

In our model, hydrogen can be produced from electrolysis and steam methane reforming (SMR). Geological distribution and potentials for \hydrogen{} storage in salt caverns are derived from~\cite{caglayanTechnicalPotentialSalt2020}. Re-electrification of hydrogen is possible via fuel cells. If enabled, hydrogen can be transported via pipelines between regions which can be expanded without limit, considering costs for pipeline segments and compressors. Pipeline flows are modelled using net transfer capacities and without flow dynamics, pressure valves, or energy demand for compression.
% No retrofitting of gas pipelines is considered, potentially overestimating hydrogen network cost.

The topology of the \carbon{}, \hydrogen{} and gas network is identical to the topology of the electricity network, connecting all neighboring regions. Transport of liquid fuels like oil, methanol and Fischer-Tropsch (FT) fuels is ``copper-plated'', since transport costs per unit of energy are negligible due to their high energy density. Throughout this study, we focus on \carbon{} and \hydrogen{} networks because of their high relevance for infrastructure investments and thus public policy decisions. The electricity network and gas network are both included in the model with full geographical detail, but not further analyzed. Electricity networks are already in place, and we restrict further extension due to concerns of public acceptance. Gas networks also already exist, and will most likely experience decreased use in the future, removing any bottlenecks or constraints on the optimal system buildout or operation.

Our model features three drop-in fuel production technologies for CU: methanation, methanolization, and Fischer-Tropsch synthesis. The processed fuels are not limited in their total quantity of use or production. Methane is transported through the gas network, while methanol and FT fuel can be transferred inter-regionally without additional costs, since dense fuels have negligible transport costs per unit of energy. Synthetic methane substitutes natural gas or biogas, serving combined heat and power plants, residential heating gas boilers, or industrial process heat. Synthetic methanol decarbonizes marine industry fuel demands, and FT fuels replace fossil oil for naphtha production, aviation kerosene, or agricultural machinery oil.

To supply carbon needed for CS and CU, the system can choose to deploy carbon capture technology at various point sources (see below), or through DAC facilities. The concept of a merit order for captured carbon plays a pivotal role in optimizing and prioritizing the deployment of carbon capture technologies based on their economic feasibility. This merit order ranks all CC technologies according to their relative costs to capture a ton of carbon (see Fig.~X). At the lower end of the cost spectrum, CC technologies applied to process emissions, such as those from cement, offer a cost-effective starting point. Following this, biomass combined heat and power (CHP) systems provide a dual benefit of energy production and carbon capture. Moving up the scale, gas used in industrial applications and biomass employed for industrial processes represent more costly yet viable options to capture carbon. DAC, an emerging technology capable of extracting \carbon{} directly from the atmosphere, stands at a higher cost level due to its current high capital costs and energy demand. Finally, based on our model results, biogas-to-gas upgrading, a process that refines biogas to natural gas quality, incurs the highest costs in the merit order per marginal ton of captured \carbon{}. Biogas input is a high-cost fuel which based on the endogenous modeling decisions is not used in large quantities. To the extent that biogas-to-gas facilities are used by the model to supply additional (carbon-neutral) gas as fuel, adding CC infrastructure incurs low costs (and thus ranges on the left-hand side of the merit order curve). However, the price of capturing additional \carbon{} from biogas-to-gas upgrading is high because a substantial amount of the biogas fuel costs factors into the marginal cost of captured \carbon{}. This merit order framework is crucial for strategically deploying necessary carbon capture solutions while balancing economic considerations. If we consider the spatial aspect of distributed carbon capture potentials, the \carbon{} network can exploit comparative cost advantages between CO2 bidding zones to utilize the lowest-cost CC facilities across the continent. We assume a capture rate of 90\% for CC on process emissions, SMR, biogas-to-gas, as well as gas and biomass used in industry, and 95\% for CC on biomass and gas CHPs.

To store carbon, we differentiate between short-term storage in steel tanks and long-term, irreversible sequestration in underground sequestration sites such as porous rock formations or depleted gas reservoirs. Costs for both options are included in the model. For carbon sequestration, we only consider offshore sites as potential sinks. We make this choice because offshore sequestration sites typically have larger capacity compared to onshore sequestration sites in saline aquifers and due to concerns over public safety for infrastructure near populated areas. Our estimates for carbon sequestration potentials are conservative, limiting total sequestration to 25~Mt per site and calculating annual storage availability over 25 years. Furthermore, we decide to cap the total sequestered \carbon{} to 200~Mt per year for our Net-Zero scenario. This is enough to offset the hardest-to-abate fossil process emissions and some limited slack of fossil fuels to be used where they can be most valuable. This is in order to avoid offsetting fossil emissions with carbon removal if they can technologically be avoided in the first place. All technology cost assumptions are taken for the year 2040 and sourced from an open-source database~\cite{lisazeyenPyPSATechnologydataTechnology2023}.


Addressing climate targets, we define two \carbon{} target scenarios:
\begin{itemize}
    \item[] \textit{Net--Zero}: aligning with the EU's 2050 emission targets. Also denoted as NZ.
    \item[] \textit{Net--Negative}: 10\% net-negative emissions relative to 1990 levels, equaling 460~Mt \carbon{} annually. Also denoted as NN.
\end{itemize}
Unless otherwise mentioned, our reference is the Net-Zero scenario. For the Net-Negative scenario, the cap of total carbon sequestration is adjusted from 200~Mt to 660~Mt per year accordingly.

Beyond \carbon{} targets, we consider four expansion scenarios, referred to as models:
\begin{itemize}
    \item[] \textit{Baseline:} Neither \carbon{} nor \hydrogen{} networks are constructed.
    \item[] \textit{\carbongrid:} Only the \carbon{} network is developed.
    \item[] \textit{\hydrogengrid:} Only the \hydrogen{} network is developed.
    \item[] \textit{Hybrid:} Both \carbon{} and \hydrogen{} networks are developed.
\end{itemize}



\section{Results}
\label{sec:results}


Focusing first on the Net-Zero scenario, we can highlight notable characteristics of the energy system buildouts shared by all models, as well as significant variations in system costs and technology deployment between the models (see Fig.~\ref{fig:cost_bar}).

\begin{figure}[ht!]
    \centering
    \includegraphics[width=\linewidth]{comparison/default/figures/90_nodes/cost_bar.png}
    \caption[short]{Total annual system cost for the European energy system for the different models subdivided into groups of technologies. While the Baseline model has neither a \carbon{} nor a \hydrogen{} network, the Hybrid model is allowed to expand both. ``Gas Infrastructure'' combines gas facilities for power and heat production, ``\carbon{} Infrastrucure'' combines transport, storage and sequestration, and ``H$_2$ Infrastructure'' combines transport and storage. ``Carbon Capture at Point Sources'' combines all technologies with integrated carbon capture, including the cost of the main facility (e.g., CHP unit) and the carbon capture application.}
    \label{fig:cost_bar}
\end{figure}

Total annual system costs in the \modBase{} are highest, totalling \label{}803 billion euros. The models with additional transport options for carbon and hydrogen achieve a more efficient system allocation and lower total system costs, with around \label{}2\% cost reduction (\label{}781 billion euros) in the \modCO{}, and around \label{}4\% cost reduction in the \modH{} and the \modHybrid{} (\label{}765 and \label{}764 billion euros, respectively).

In all models, almost half of the system costs are spent on primary electricity production from wind and solar power (\label{}45\% or \label{}360 of \label{}799 billion euros total costs in the Baseline model), with another \label{}12\% (\label{}100 billion euros) spent on hydro, nuclear and biomass. Costs for energy transport networks for electricity, gas, hydrogen and carbon are below \label{}12\% (\label{}100 billion euros), with the electricity grid making up more of the system costs than the other three networks combined. This is the case for all four models. Capital expenditures on carbonaceous fuel production are around \label{}3\% (\label{}25 billion euros) in each model, while electrolysis makes up around \label{}5\% (\label{}40 billion euros). Heat pump installations account for around \label{}8\% (60 billion euros) of total system costs.

System cost differences between the models are driven by renewable energy resources (solar, wind, and bioenergy), the \carbon{} and \hydrogen{} networks, as well as carbon capture technologies (DAC and CC at point sources).
% In the Baseline model, the absence of dedicated carbon and hydrogen transport networks means that the system cannot use all the CC potentials of geographically fixed point sources, but instead needs to deploy more DAC in the locations where carbon is needed for CS and CU.
% Naturally, costs for the \carbon{} and \hydrogen{} networks differ between models as they are being deployed.
% Finally, we see that the Baseline model deploys more wind and solar than the network models. This is driven by two aspects of the electrolytic hydrogen production: First, the Baseline model deploys more methanation, leading to a larger endogenous hydrogen demand and thus more electricity production. And second, in the network models, the electrolysis is deployed at sites with better renewable resources, thus decreasing the total investments in solar and wind power necessary to run the electrolysis.
As we show in the following, these are due to essential though fundamentally different flexibility gains and a shift in carbon utilization and carbon sequestration strategies through the \hydrogen{} and \carbon{} networks.

\subsection*{Transport Systems Unlock Point\=/Source Carbon Capture}

% This is a bit low-level on regional scale, start with the big picture and dynamics, then zoom in on the regional scale in 3.2 (next section) together with end-use of hydrogen and carbon
% alternative outline:
% 1. baseline model is sub-optimal -> un-used landlocked carbon capture potentials; scarcity in hydrogen and power in Central Europe due to industrial clusters and high population density; costly DAC + additional renewables at sequestration sites
% 2. introduce carbon balance figure -> transport systems unlock point-source CC potentials, in particular biomass and gas for industry, less methanation, less DAC and therefore less DAC specific renewables
% 3. introduce capture share figure -> correlation between average cf and cc share, since gas CHPs and PPs are peak load techs, no cc infrastructure;
% 4. carbon-grid has higher absolute CC and CC shares than hydrogen-grid and less DAC however hydrogen-grid is more cost-effective

% 1. baseline model is sub-optimal -> un-used landlocked carbon capture potentials; scarcity in hydrogen and power in Central Europe due to industrial clusters and high population density; costly DAC + additional renewables at sequestration sites
We can now closer examine the carbon capture activity when shifting from the \modBase{} to the other three network models. In comparison, the \modBase{} reveals significant inefficiencies: On the one hand, regions with no access to sequestration sites (i.e., ``landlocked'' regions) underutilize CC potentials from process emissions, fossil gas and biomass. On the other hand, sequestration sites rely on costly DAC to capture the carbon for CS from the air. Furthermore, hydrogen and power prices show high regional differences, indicating system inefficiencies due to transport constraints caused by non-existent (in case of hydrogen) or insufficient (in case of power) network connections. Prices are significantly higher in Central Europe, a region with high population density and high concentration of energy-intensive industries.

% 2. introduce carbon balance figure -> transport systems unlock point-source CC potentials, in particular biomass and gas for industry, less methanation, less DAC and therefore less DAC specific renewables
Transport systems emerge as a crucial component in optimizing CC. By enabling the efficient transportation of carbon, hydrogen, or both, these models reduce the reliance on high-cost DAC, and source more carbon from low-cost point-source CC on biomass and gas in industrial applications (see Fig.~\ref{fig:balance_captured_carbon}). The \modH{} only captures \label{}200 Mt of \carbon{} from DAC compared to \label{}400 Mt in the \modBase{}. Instead, the \modH{} captures more carbon from process emissions as well as gas and biomass for industry. The \modCO{} uses only \label{}150 Mt from DAC, and instead also captures more carbon than the \modH{} from gas for industry and biomass CHPs. And finally, the \modHybrid{} captures only \label{}100 Mt from DAC, but slightly more carbon from all other CC technologies.
Also, the networks reduce the reliance on methanation in the model, reducing the overall demand for CC volumes. Another knock-on effect of the reduced reliance on DAC is reduced endogenous demand for renewables to power DAC facilities.

% 3. introduce capture share figure -> correlation between average cf and cc share, since gas CHPs and PPs are peak load techs, no cc infrastructure;
The increase in captured carbon from point sources in the network models is mostly due to CC infrastructure being deployed on industrial or power plant facilities that are present in all models (as opposed to the deployment of additional electricity generating capacity for the purpose of carbon capture). This is evident when analyzing the capture share of the different CC technologies (see Fig.~\ref{fig:captureshare_line}).

In the \modBase{}, the capture share is highest for biogas-to-gas facilities, followed by process emissions, biomass for industry, biomass CHPs, gas for industry, SMR and gas CHPs. For the network models, we can observe a correlation between CC share and capacity factors of the underlying technologies: Process emissions as well as gas and biomass for industry have high capacity factors (above 80\%), and CC shares close to 100\% (the only exception being gas for industry in the \modH{}). Biomass CHPs have a capacity factor of around 40\%, and CC shares of 50\% in the \modH{} and 95\% in the \modCO{}. SMR and gas CHPs have capacity factors below 20\%, and CC shares around 50\% and 0\%, respectively, across all network models. Gas CHPs serve as peak load electricity production technology, and with very few operating hours, the high investment costs into CC applications are not efficient.\footnote[1]{Similarly, SMR serves as ``peak-load'' hydrogen production technology in regions with poor renewable resources for electrolysis.}


% 4. carbon-grid has higher absolute CC and CC shares than hydrogen-grid and less DAC however hydrogen-grid is more cost-effective


[old text]
We can now closer examine the carbon capture activity when shifting from the \modBase{} to the other three network models.
CU is preferably deployed in the regions where carbon, hydrogen as well as the inputs and by-product outputs (e.g., waste heat) of the synthesis processes have the lowest cost (or highest value). In the \modBase{}, this means combining the lowest-cost CC source (process emissions) with electrolysis in all regions, even if the renewable resources are poor and electrolysis costs are thus relatively high (e.g., non-coastal regions in Germany). In regions with medium-quality renewables (e.g., France and Scandinavia), CC biomass is also deployed -- and combined with medium-price electrolytic hydrogen -- to run additional CU. But the largest volumes of CU are deployed in the regions with the best renewables and lowest-cost electrolysis (e.g., Iberian Peninsula and British Isles), where all point-source CC potentials are exhausted and additionally necessary carbon is sourced from high-cost DAC.

In the \modH{}, the hydrogen network enables the system to transport low-cost hydrogen to make use of low-cost biomass CC potentials that were unused in the \modBase{}. Thus, less additional DAC is needed to run CU processes, contributing to the reduction of DAC to \label{}200 Mt (see Fig.~\ref{fig:balance_captured_carbon}).
CC technology on process emissions increases from \label{}90\% to \label{}100\%, on biomass in industry from \label{}65\% to \label{}100\%, and on facilities using gas in industry from \label{}40\% to \label{}75\%. CC on biomass CHPs increases slightly from \label{}45\% to \label{}50\% (see Fig.~\ref{fig:captureshare_line}).

In the \modCO{}, a similar effect is achieved by transporting the carbon from unused low-cost biomass CC potentials to the sites where the lowest-cost hydrogen is produced.


\begin{figure}[ht!]
    \centering
    \includegraphics[width=\linewidth]{comparison/default/figures/90_nodes/balance_bar_carbon.png}
    \caption{Balance of captured carbon across all system setups. Positive values indicate carbon capture, negative values indicate carbon consumption.}
    % A CO2 network unlock BECSS potentials, most biomass emissions are captured at point sources and transported to sequestration sites. cite~\cite{rosaAssessmentCarbonDioxide2021}
    \label{fig:balance_captured_carbon}
\end{figure}

\begin{figure}[ht!]
    \centering
    \includegraphics[width=\linewidth]{comparison/default/figures/90_nodes/balance_bar_hydrogen.png}
    \caption{Balance of captured hydrogen across all system setups. Positive values indicate carbon capture, negative values indicate carbon consumption.}
    % A CO2 network unlock BECSS potentials, most biomass emissions are captured at point sources and transported to sequestration sites. cite~\cite{rosaAssessmentCarbonDioxide2021}
    \label{fig:balance_hydrogen}
\end{figure}


\begin{figure}[h]
    \centering
    \includegraphics[width=\linewidth]{comparison/default/figures/90_nodes/captureshare_line.png}
    \caption{Share of facilities with integrated carbon capture for different system setups for the Net-Zero scenario.}
    % A CO2 network unlock BECSS potentials, most biomass emissions are captured at point sources and transported to sequestration sites. cite~\cite{rosaAssessmentCarbonDioxide2021}
    \label{fig:captureshare_line}
\end{figure}%

Now we turn to how carbon for CS is sourced by the different models. In the \modBase{}, because of the absence of a carbon network, all carbon that is sequestered (200 Mt, the same in each model) needs to be captured in the regions with access to an offshore CS site. The \modBase{} deploys all available low-cost point-source CC in these regions, and adds DAC in the sequestration locations with the lowest-cost electricity. In other words, the \modBase{} uses the atmosphere as a substitute carbon transport pathway, allowing \CO{} from inland point sources to escape into the atmosphere, and extracting it again from the atmosphere with DAC at the offshore locations.

In the \modCO{}, the carbon network enables the system to capture carbon from low-cost point sources and transport it to the sequestration sites, trading off CC costs with transport costs. Together with the reduction of DAC for CU explained above, this contributes to the reduction of DAC from almost \label{}400 Mt in the Baseline model to \label{}150 Mt in the \modCO{} (see Fig.~\ref{fig:balance_captured_carbon}). The CC share on process emissions, gas and biomass use in industry, as well as biomass CHPs increase to 100\% (see Fig.~\ref{fig:captureshare_line}). The increased CC from point sources compared to the \modH{} thus derives from more CC on gas for industry and biomass CHPs.

In the \modHybrid{}, both networks contribute together to an even larger reduction of DAC to only \label{}100 Mt, with a larger deployment of biomass CHPs than in the \modCO{}, and of which also 100\% are equipped with CC installations.

In summary, the \carbon{} network enables the transport of carbon from the lowest-cost CC sources to CU and CS carbon sinks, reducing the need for high-cost DAC.
The creation of a \hydrogen{} network enables the cost-effective delivery of hydrogen from regions with a surplus of low-cost renewable energy to high-demand industrial centers with exogenous hydrogen demand and a surplus of low-cost CC potentials.
This highlights how both the implementation of a \carbon{} or a \hydrogen{} network unlocks low-cost carbon capture potentials at geographically fixed point sources across the continent.


\subsection*{\hydrogen{}--Network enables decentral carbon utilization}

% \begin{figure*}[ht!]
%     \centering
%     \begin{subfigure}{0.9\linewidth}
%         \centering
%         \includegraphics[width=\linewidth]{co2-only/figures/90_nodes/balance_map_carbon.png}
%         \caption{\carbon{} Sector.}
%         \label{fig:balance_map_carbon}
%     \end{subfigure}
%     \begin{subfigure}{0.9\linewidth}
%         \centering
%         \includegraphics[width=\linewidth]{h2-only/figures/90_nodes/balance_map_hydrogen.png}
%         \caption{\hydrogen{} Sector.}
%         \label{fig:balance_map_hydrogen}
%     \end{subfigure}
%     \caption{Optimal operation per sector for a net-zero energy system in Europe with average production on the left and average consumption on the right for both, (a) the \carbon{} sector in the \carbon{}-Grid model and (b) the \hydrogen{} sector in the \hydrogen-Grid model.}
%     \label{fig:balance_map}
% \end{figure*}

\begin{figure*}[ht!]
    \centering
    \includegraphics[width=\linewidth]{comparison/single-technologies/figures/90_nodes/balance_map_dedicated.png}
    \caption{Optimal operation per sector for a net-zero energy system in Europe with average production on the left and average consumption on the right for both, (a) the \carbon{} sector in the \carbon{}-Grid model and (b) the \hydrogen{} sector in the \hydrogen-Grid model.}
    \label{fig:balance_map}
\end{figure*}


% Examining the cost-reductions across the different models raises an important question: why is the hydrogen network more cost-effective than its carbon counterpart, and how does it achieve similar cost reductions as the hybrid model? An initial hypothesis may be that the cost assumptions for \carbon{}
%  and \hydrogen{} networks play a role. However, the amount of system costs spent on either network is substantially lower than the cost difference between the models (20 billion euros less total system costs in the \hydrogengrid{} model, while it spends 10 billion euros on the \hydrogen{} grid and the \carbongrid{} model spends 8 billion euros on the \carbon{} grid). Furthermore, we have conducted a sensitivity analysis with cost assumptions for the \carbon{} network reduced to 50\% of the default assumptions, and the results show that the \hydrogen{} network still outperforms the \carbon{} network in terms of total system costs.
To find the reason for why the \hydrogen{} network achieves lower system costs than the \carbon{} network, it is crucial to understand the underlying mechanisms and geographical implications of the two transportation systems. Fig.~\ref{fig:balance_map} maps the average production, consumption and flow in the carbon (upper row) and hydrogen (lower row) sectors for the \carbongrid{} (left) and \hydrogen{} (right) model. In each region, circles represent the production (top half-circle) and consumption (bottom half-circle) of the respective energy carrier, divided into shares of each technology and the area sizes correspond to production and consumption volumes. Differences between production and consumption in a given node correspond to import or export volumes through the connected networks, represented as lines with their size and arrows indicating volumes and direction of average flows. Each region's color represents the demand-weighted average price of the considered carrier in that region.

There are four notable dynamics or differences between the two grid strategies:
First, the production of FT fuels always moves to where low-cost carbon and low-cost hydrogen can be brought together. In the \carbongrid{} model, carbon is moved from Western Europe to the Iberian Peninsula, the British Isles and to Denmark to produce FT fuels centrally in few regions. In the \hydrogengrid{} model, hydrogen is transported in the opposite direction to produce FT fuels decentrally across all regions in Central Europe with locally captured \carbon{} from point sources.
Second, the \carbongrid{} model captures more low-cost carbon from biomass, especially in Central and Eastern Europe, and less high-cost carbon from DAC in Greece and Southern Italy. Instead, the carbon grid transports carbon from Central and Eastern Europe to Greece and Italy, where it is sequestered, or feeds into CU together with locally produced, low-cost electrolytic hydrogen. The \hydrogengrid{} model directly operates electrolyzers and DAC in Greece and Southern Italy, with electricity from low-cost renewables, and thus avoids significant network buildouts in this region.
Third, sequestration in the \hydrogengrid{} model occurs more decentralized, spread out across several regions with access to offshore sequestration sites, making use of point-source CC potentials in each of these regions. In the \carbongrid{} model, sequestration is clustered at fewer sequestration sinks, each collecting low-cost \carbon{} from neighboring regions. However, there is no centralization at only one or two sites, since all the sequestration sites offer the same costs and the system minimizes system-wide carbon transport distances to about eight to ten sequestration sites.
And fourth, how the three types of CU are spread across the continent is driven by the costs of carbon and hydrogen (the latter mostly driven by the costs of renewable electricity) as well as by renewables capacity factors: Clearly, all carbonaceous fuel production requires sites where both carbon and hydrogen are available, and they moreover prefer sites with low carbon and hydrogen prices. Based on our model assumptions, FT fuel synthesis has very limited flexibility and needs to be run alost at baseload (ramp-down only possible to 90\% of maximum load). Thus, for FT fuel synthesis sites in Northern Europe with more electricity from wind power are preferred to sites in Southern Europe, powered mostly by solar. Methanathion and Methanolisation both have higher flexibility, being able to ramp down to 50\% load during the night, for example. Thus, these fuel production sites prefer the best renewable resource sites in Southern Europe. And finally, as methanation produces synthetic methane that needs to be fed into the gas grid, methanation prefers those sites with good interconnection to the existing gas network. Since Italy possesses a large gas network with easy interconnection to Central Europe, methanation prefers sites in Southern Italy (and Greece, which would be connected to Southern Italy) over the Iberian Peninsula, where Methanolisation is more prevalent.

In summary, the implementation of a \carbon{} network allows for connecting regions with abundant, low-price carbon (from point sources) to regions that can sequester the carbon or that can produce low-cost hydrogen to make CU products (FT fuel, methanol, synthetic methane). DAC and electrolysis are deployed in the regions with the best renewable resources to meet carbonaceous fuel demands.
In the absence of a \carbon{} network and with a \hydrogengrid{} on the other hand, regions with access to offshore sequestration sites capture all carbon from low-cost point sources and then sequester it. The system deploys additional DAC facilities at the locations with sequestration access and low-cost renewables to capture the remaining necessary carbon for sequestration. Carbonaceous fuel demand is met at the locations which offer the best mix of low-cost carbon from point sources and low-cost hydrogen from renewables, with some additional electrolysis and DAC required to fully meet the demand.
% Sequestered carbon is nearly always transported before

% Therefore, one can conclude that a \carbon{} network favors a decentralized CC and a CU centralized at regions with abundant renewable resources. A \hydrogen{} network favors a centralized electrolysis system at regions with abundant renewable resources and a decentralized CU system.

Together with understanding the dynamics described above, we can highlight three cost advantages of the \hydrogen{} network in contrast to the \carbon{} network: Capturing carbon at the lowest-cost regions reduces 4 billion euros in spending on DAC. However, the \hydrogengrid{} model produces electrolytic hydrogen in the lowest-cost regions and in total spends 7 billion euros less on renewable energy (14 billion euros less on solar, and 7 billion euros more on wind) as well as 1 billion euros less on SMR. Furthermore, without a \hydrogen{} network, additional \hydrogen{} storage is needed in Spain at a cost of 5 billion euros to store hydrogen produced during the summer for utilization in FT fuel production during the winter.
And finally, the \carbongrid{} model spends 7 billion euros more on gas boilers, gas plants, gas infrastructure, and methanation.
% TODO: why??
The costs of the respective networks, on the other hand, are almost equal: the \carbongrid{} model spends 8 billion euros on the \carbon{} network, while the \hydrogengrid{} model even spends 10 billion euros on the \hydrogen{} network. These factors add up to account for 14 billion euros of the difference of 20 billion euros in system costs between the two scenarios (785 billion euros for the \carbongrid{} and 765 billion euros for the \hydrogengrid{} model, see Fig.~\ref{fig:objective_heatmap}).


\subsection*{Hybrid configuration provides further flexibilities}

\begin{figure*}[ht!]
    \centering
    \begin{subfigure}{.49\textwidth}
        \centering
        \includegraphics[width=\linewidth]{full/figures/90_nodes/balance_map_single_hydrogen.png}
        \caption{\hydrogen{} Sector.}
        \label{fig:capacity_map_hydrogen_co2}
    \end{subfigure}
    \hfill
    \begin{subfigure}{.49\textwidth}
        \centering
        \includegraphics[width=\linewidth]{full/figures/90_nodes/balance_map_single_carbon.png}
        \caption{\carbon{} Sector.}
        \label{fig:capacity_map_carbon_co2}
    \end{subfigure}%
    \caption{Optimal production and transport capacities of the carbon and hydrogen sector in a net-zero energy system in Europe with both \carbon{} and \hydrogen{} network expansion (Hybrid).
    % Carbon network looks the same as in~\cite{morbeeOptimisedDeploymentEuropean2012}: two backbones, one in the nothern Europe other in south east.
    }
    \label{fig:capacity_maps}
\end{figure*}

% increased BECCS, decreased DACS
% decreased methanation

The Hybrid model combines the advantages of both hydrogen and carbon networks. While the topology of the \hydrogen{}-grid roughly corresponds to that of the \hydrogengrid{} model, the \carbon{}-grid plays more of a complementary role. Fig.~\ref{fig:capacity_maps} shows the average production, consumption and transport of the two grids in the hybrid model. Note that despite similar overall capacities, the investment in the hydrogen infrastructure is four times the investment in the carbon infrastructure.


Similar to the \hydrogengrid{} model, the Hybrid model takes advantage of a large hydrogen network that enables hydrogen to be transported from centralized production sites in western regions such as Spain to decentralized FT production sites across the continent. However, in the Hybrid model hydrogen in the UK is used locally to produce synfuels, and less hydrogen is transported to Central Europe.
The carbon network supplements this with smaller clusters of networks. In addition to a larger carbon grid cluster in Central Europe, which transports carbon to the North Sea, the system is building three large linear carbon routes. These transport carbon from Romania and Bulgaria to Greece, from northern Italy to central Italy and from Hungary, the Czech Republic and Poland to the Baltic Sea. These enable the system to transport carbon available at low cost from inland biomass sources and sequester it (see Fig.~\ref{fig:captureshare_line}). This reduces the need for DAC in the system and the sequestration sites are shifted to the end points of the carbon networks. Second-order effects can be seen in the shift in FT production compared to the \hydrogengrid{} model. Regions that have both good renewable resources and sequestration potential could now forego sequestration and use local process emissions for FT production, as can be observed in Ireland. A notable aspect of the hybrid model is the lack of overlap in the hydrogen and carbon network topologies. This finding is tightly linked to the fact of competing functionalities to offtake carbon emissions.
A more detailed of differences between the \hydrogengrid{} and the Hybrid model can be found in Fig.~.


Despite only small decrease in cost in comparison to the \hydrogengrid{} model, hybrid approach brings several advantages. These include a broader range of technologies contributing to its robustness, reduced reliance on DAC, which may be costlier than anticipated in the cost projections, and less land use for wind and solar due to the decreased necessity for DAC. The increase used of biomass with carbon capture, transport and sequestration leads to slight cost advantages in comparison to the \hydrogengrid{} model.



\subsection*{Net-negativity amplifies DAC and Bioenergy with CU and CS}

\begin{figure}[htb!]
    \centering
    \includegraphics[width=0.9\linewidth]{difference/comparison/emission-reduction-0.1/figures/90_nodes/cost_bar.png}
    \caption[short]{Net change in investments when tightening the \carbon{} emission target from net-zero to net-negative 10\% of 1990s emissions.}
    \label{fig:net-negative_cost_bar}
\end{figure}


In transitioning from a net zero to a net emissions reduction target, a continuation of investment trends are perceived. Fig.~\ref{fig:net-negative_cost_bar} shows the net investment changes per technology group for the \hydrogengrid{} and Hybrid. The scenario imposes a net carbon removal of 460 Mt/a, with both models incurring a similar net cost increase - 95 billion €/a for \hydrogengrid{} and 96 billion €/a for Hybrid, balancing the earlier cost advantage of the Hybrid model. Both models invest relatively evenly in additional DAC, wind and heat pump systems, with DAC accounting for more than a third of the additional system costs. These remove carbon directly at sequestration sites at the coast, in particular in Portugal and UK, together with heat and power from new heat pumps at site and new solar and wind installation in the vicinity.
More heat pumps, solar and wind installations across all Central and Eastern European countries as well as biogas-to-gas installations compensate for the reduction of fossil gas based heating and powering. Reduced usage of gas in general lower the investments in gas infrastructure in both models.
Fundamental difference in investment strategies occur in the carbon capture sector: In contrast to the \hydrogengrid{} model, the Hybrid model continues the strategy of sequestering carbon captured and transported from bio-energetic inputs, expanding the carbon transport routes from Central and Eastern Europe to the near shores. At the same time, carbonaceous fuel production in Central Europe partially moves to Spain and UK, where it uses carbon from new DAC facilities without the installation of a \carbon{} network. At the same time, solar based electrolysis in France, Nothern Italy and Switzerland expands requiring more solar capacities but less hydrogen imports from the Iberian Peninsula.
The \hydrogengrid{} model on the other hand, expands decentralized CU systems from the Net-Zero scenario, which are now increasingly supplied with carbon from biogas-to-gas facilities, showing up in both "Bioenergy" and "Carbon Capture at Point Sources" cost-contributions. The centralized DAC facilities at sequestration sites are more expanded than in the Hybrid model and are requiring additional heat from biomass CHPs.


These findings highlight the continuation of trends of the two systemic approaches, which are interchangable at marginal cost-difference: While a stand-alone \hydrogen{} grid combines well with highly centralized DAC and decentralized CU with bioenergy as primary carbon source, the complementation with a \carbon{} grid promotes combining DAC and CU where good renewable resources are and collecting, transporting and sequestering carbon from decentral bioenergetic processes.

% The difference in absolute investment changes between the two models, regionally displayed in Fig.~\ref{fig:net-negative_cost_map}.
% \begin{figure*}[htb!]
%     \centering
%     \includegraphics[width=\linewidth]{difference/net-negative-0.1-h2-only-full/figures/90_nodes/cost_map.png}
%     \caption[short]{Net difference in investments between the \hydrogengrid{} model and the Hybrid model for the Net-Negative scenario.}
%     \label{fig:net-negative_cost_map}
% \end{figure*}


\section{Conclusion}
\label{sec:conclusion}

% TODO: distribution of current refineries might shape the CU network. These are quite decentralized and might fit to the cost-optimal layout.

In this study\dots

\dots

Despite the detailed model representation, there are some limitations to the validity of the results. The model itself is based on a linearized optimization with perfect foresight for the entire modeling year. In reality, long-term energy demand and renewable supply can only be roughly estimated, while short-term predictions still entail some uncertainty. The model's perfect foresight may lead to non-reproducible behavior, such as precisely aligning energy storage with future energy shortages at a specific point in time.

The technology costs used in the model rely on cost projections that incorporate reductions based on learning rates. These learning rates are derived from historical data, which may not necessarily be indicative of future trends. In particular, cost assumptions on DAC, electrolysis and \hydrogen{} pipeline costs, could have a strong impact in the model results.

The modelling results are heavily driven by the demand and emissions from industrial clusters. Allowing the model to relocate these and/or incorporate flexibility measures, may lead to less dependency on both \carbon{} and \hydrogen{} networks.

The technological flexibilities in the carbon sector might not be exploited to full extent. Therefore, the \carbon{} transport via truck or ship is not considered. Fischer-Tropsch facilities require to run on baseload with at least 90\% of their nominal capacity.

\dots



% TODO: which fixed rate? you mean that all emissions are captured by exogenous assumption or that we assume 90% of carbon can be captured?
% Furthermore, the model assumes a fixed rate for industrial process emissions, which cannot be altered through investments. This simplification may not accurately represent real-world scenarios, where many industries are considering adopting low-carbon processes and technologies. The transport of fuel (FT, gas, oil) between the regions is not unlimited which may overestimate the flexibility provided by these commodities.


\printbibliography

\appendix


% \begin{figure*}
%     \centering
%     \includegraphics[width=\linewidth]{full/figures/90_nodes/sankey_diagramm.png}
%     \caption{Sankey diagram of the optimal operation for a net-zero scenario.}
%     \label{fig:sankey_diagramm}
% \end{figure*}

\begin{figure*}
    \centering
    \includegraphics[width=\linewidth]{net-negative-0.1/full/figures/90_nodes/sankey_diagramm.png}
    \caption{Sankey diagram of the optimal operation for a net-negative 10\% scenario.}
    \label{fig:sankey_diagramm}
\end{figure*}

\begin{figure}
    \centering
    \includegraphics*[width=0.8\linewidth]{comparison/emission-reduction/figures/90_nodes/objective_heatmap.png}
    \caption{Total annual system cost for the different scenarios and \carbon{} targets. While the `Baseline' scenario has neither a \carbon{} nor a \hydrogen{} network, `Hybrid' is allowed to expand both.}
    \label{fig:objective_heatmap}
\end{figure}


\begin{figure}[ht]
    \centering
    \includegraphics[width=\linewidth]{comparison/emission-reduction-full/figures/90_nodes/cost_bar_transmission.png}
    \caption{Annual transmission system cost as a function of the net carbon removal scenarios considered in the study.}
    \label{fig:cost_bar_transmission}
\end{figure}


\begin{figure*}[ht]
    \centering
    \includegraphics[width=\linewidth]{difference/h2-only-full/figures/90_nodes/cost_map.png}
    \caption{Difference in cost investments between \hydrogengrid{} and Hybrid model. The left subfigure shows higher spendings per technology and region and transport system for the \hydrogengrid{} model, the right shows higher spendings in the Hydrid model.}
    \label{fig:cost_map_difference}
\end{figure*}


\end{document}

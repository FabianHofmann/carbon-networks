\documentclass[twocolumn]{article}
\usepackage{amsmath,amssymb,amsfonts}
\usepackage{cuted}  % Add the cuted package
\usepackage{caption}
\captionsetup{font=small}
\usepackage{algorithmic}
\usepackage{graphicx}
\usepackage{textcomp}
\usepackage{xcolor}
\usepackage{tabularx, multirow}
\usepackage{fancyhdr,lipsum}
\usepackage{subcaption}
\usepackage[shortcuts]{extdash}

\usepackage[%
backend=biber,bibencoding=utf8, %instead of bibtex
language=auto,
style=ieee,
sorting=none, % nyt for name, year, title
maxbibnames=10, % default: 3, et al.
%backref=true,%
natbib=true % natbib compatibility mode (\citep and~\citet still work)
]{biblatex}
\bibliography{../../../references.bib}

%define approx proportional
\def\app#1#2{%
  \mathrel{%
    \setbox0=\hbox{$#1\sim$}%
    \setbox2=\hbox{%
      \rlap{\hbox{$#1\propto$}}%
      \lower1.1\ht0\box0%
    }%
    \raise0.25\ht2\box2%
  }%
}
\def\approxprop{\mathpalette\app\relax}

% make abbreviation for co2
\newcommand{\carbon}{CO$_2$}
\newcommand{\CO}{CO$_2$}
\newcommand{\hydrogen}{H$_2$}
\newcommand{\Htwo}{H$_2$}
\newcommand{\carbongrid}{\carbon{}--Grid}
\newcommand{\hydrogengrid}{\hydrogen{}--Grid}
\newcommand{\modBase}{Baseline model}
\newcommand{\modCO}{CO$_2$-Grid model}
\newcommand{\modH}{H$_2$-Grid model}
\newcommand{\modHybrid}{Hybrid model}

\graphicspath{
    % {paper-figures}
    {figures}
    {../../../../results/}
    {results/}
}

\begin{document}


% FORMATTING:
% - https://www.nature.com/nenergy/submission-guidelines/aip-and-formatting
% - https://www.nature.com/nenergy/content abstract: 150 words, main text: 3000 words, figures: 8

\title{Assessing \hydrogen{} and \carbon{} Network Strategies in Europe}
% \title{Synthetic fuels in Europe: Transport Hydrogen to Carbon, or Carbon to Hydrogen?}

\author{
    Fabian Hofmann, Christoph Tries, Fabian Neumann, Lisa Zeyen, Tom Brown \\
    \textit{Institute of Energy Technology} \\
    \textit{Technical University of Berlin}\\
    Berlin, Germany \\
    m.hofmann@tu-berlin.de
}


\maketitle

\begin{abstract}
    Hydrogen and carbon dioxide transportation are considered crucial in climate-neutral energy systems, with hydrogen enabling energy imports to high-demand areas and carbon transport aiding in emissions export from high-emission areas. Yet, possible synergies and competitions between the two systems are not fully understood. Our study employs optimization techniques to develop a cost-optimal European energy system, integrating transport, storage, and sequestration of both carbon dioxide and hydrogen, along with renewable energy sources. Results indicate that a hydrogen network is more cost-effective than a carbon network, facilitating low-cost hydrogen imports to Central Europe and local carbon capture and utilization. However, in a hybrid scenario, the carbon network effectively complements the hydrogen network, promoting carbon capture from biomass and reducing reliance on direct air capture. This research demonstrates the effectiveness of combining hydrogen and carbon transport networks with power grid expansion in achieving climate neutrality in Europe.
\end{abstract}


\section{Introduction}

The transition to a carbon-neutral European economy is a pressing challenge that demands coordinated action across various energy sectors. While management of both \carbon{} and \hydrogen{} is considered a critical component of this transition, a gap exists in understanding how new hydrogen infrastructures effectively interact with comprehensive carbon management technologies, including carbon capture, transport, use, and storage. Hydrogen offers an efficient way to transport and store energy. Carbon, on the other hand, can be effectively captured from industrial processes and the combustion of biomass, fossil fuels, and synthetic fuels through carbon capture (CC) techniques, or harvested from the atmosphere using direct air capture (DAC). Additionally, it can be stored in geological formations, a process known as carbon sequestration (CS). In combination, \carbon{} and \hydrogen{} networks build the basis for climate-neutral fuels needed for aviation, shipping and industrial feedstocks, and are central to carbon utilization (CU) strategies.

Recently, policymakers and industry in Europe have started developing carbon management strategies~\cite{GermanyDevelopingStrategy2023,CarbonManagementStrategie}, planning infrastructure components~\cite{CONetz}, and committing on the first carbon utilization projects~\cite{EFuelsPilotPlant2022,OrstedAssumesFull,GROUNDBREAKINGEFUELPRODUCTION,DLREfuelsDLR}. With the European Union's goal of achieving climate neutrality by 2050, a wide range of programs, funding models and initiatives have been established in this regard~\cite{eu2023netzero,europeangreendeal,europeaninnovationfund}. Initiatives like the European Hydrogen Backbone~\cite{gasforclimateEuropeanHydrogenBackbone2022} or the Hydrogen Infrastructure Map~\cite{H2InfrastructureMap} showcase the potential of hydrogen as a fuel and energy carrier, and some gas pipelines are already repurposed to transport hydrogen~\cite{RohrFreiFuer}. At the same time, business models from companies like Tree Energy Solutions~\cite{TESHydrogenLife2023}, Carbfix~\cite{WeTurnCO2}, and Equinor~\cite{adomaitisEquinorRWEBuild2023} advertise carbon management hubs that provide green hydrogen, methane, and synfuels on the one hand and offer \carbon{} offtake on the other hand. The Northern Lights project in Norway~\cite{NorthernLightsWhat} is planning with a transport and sequestration capacity of 1.5 Mt \carbon{} per year  to be operative in 2024, expanding to a targeted scale of 5 Mt per year of sequestration by 2030.
% TODO: frame next two sentences to be more market oriented
For Europe, the Capture Map~\cite{ToolsGreenTransition} estimates a potential of 1.7~Gt of carbon capture from point sources per year, which represent roughly 50\% of all emissions. In combination with large sequestration potentials as stated in~\cite{weiProposedGlobalLayout2021}, this highlights the vast potential for decarbonization.
To this end, the Clean Air Task Force underlines the importance of a carbon transport system in Europe to facilitate the carbon economy~\cite{lockwoodEuropeanStrategyCarbon}.


However, up to this point, it remains unclear how the two transport systems of hydrogen and carbon may complement or replace each other, particularly when it comes to their transport systems. A hydrogen network facilitates the transport of energy from renewable sources to regions with geographically fixed hydrogen demand, such as for steelmaking, and also enables CU at the site of point-source capture. On the other hand, the carbon transport system allows for capturing and transporting carbon from areas with high emissions to regions with significant sequestration potential and abundant renewable sources, thereby enabling cost-effective CU.
In the literature, the two network approaches and underlying technologies have been discussed in a number of publications, all of which, however, dealt with the isolated effects~\cite{bakkenLinearModelsOptimization2008,morbeeOptimisedDeploymentEuropean2012,stewartFeasibilityEuropeanwideIntegrated2014,oeiModelingCarbonCapture2014,elahiMultiperiodLeastCost2014,burandtDecarbonizingChinaEnergy2019,middletonSimCCSOpensourceTool2020,bjerketvedtOptimalDesignCost2020,weiProposedGlobalLayout2021,damoreOptimalDesignEuropean2021,becattiniCarbonDioxideCapture2022}. Such techno-economic models, in contrast for example to Integrated Assessment Models, can account for the spatial distribution of carbon sources and sinks which are crucial provide a holistic view of the energy system and its technological interactions. The work in~\cite{neumannBenefitsHydrogenNetwork2022} examines the effect of a hydrogen network in Europe... (extend on hydrogen literature)
The publication by Morbee et al.~\cite{morbeeOptimisedDeploymentEuropean2012} optimizes the topology and capacity of a \carbon{} network in Europe, but only considers the power sector without co-optimizing renewable deployment. This limited sectoral scope cannot capture important dynamics of carbon management, since it neglects the sectors that will need to handle most \carbon{} in the future.
Another comprehensive example is found in~\cite{becattiniCarbonDioxideCapture2022}, which presents a mixed-integer model to optimize the time-evolution of a \carbon{} transport system in Switzerland, connecting to a remote sequestration site in Norway. Hoewever, this limited spatial scope fails to consider other sequestration sites and co-benefits from connecting the \carbon{} network to neighboring countries. To this end, it is often argued that \carbon{} pipelines are a mature technology with an expected high learning rate, given the wide-spread installations in US and Canada for enhanced oil recovery~\cite{righettiSitingCarbonDioxide2017,friedmannNETZEROGEOSPHERICRETURN}.
% However, the models are often limited with regard to both geographical scope and detail. While representing a single country with spatial resolution may neglect synergies of international cooperation, a coarse grained representation of multiple countries may neglect important geographical properties.

To our knowledge, no study has yet considered the co-optimization and comprehensive assessment of both \carbon{} and \hydrogen{} networks in a fully sector-coupled energy system. However, we argue that such an assessment is strongly needed to avoid suboptimal investments and to identify synergies between hydrogen and carbon management technologies. In this paper, we present a detailed study of the European energy system for 2050, which includes high geographical resolution and a comprehensive representation of carbon management technologies. The study is conducted using the PyPSA-Eur energy system model and encompasses all relevant energy sectors. We explore competition between \carbon{} and \hydrogen{} networks by adding them separately. Our evaluation focuses on the transport of \carbon{} and \hydrogen{} through their respective networks on the European continent. We also analyze how an energy system with limited annual sequestration potential prioritizes decarbonization and fuel switching in various sectors, and how the construction of carbon networks varies based on different levels of available sequestration potential.


\section{Methodology}
\label{sec:methodology}

The study is conducted on the basis of the open-source, capacity-expansion model PyPSA-Eur~\cite{horschPyPSAEurOpenOptimisation2018,brownSynergiesSectorCoupling2018,PyPSAEurSecSectorCoupledOpen2023}.
The model optimizes the design and operation of the European energy system, encompassing the power, heat, industry, waste, agriculture, and transport sectors, including international aviation and shipping.

\begin{figure}
    \includegraphics[width=\linewidth]{baseline/figures/90_nodes/total-demand-bar.png}
    \caption{Assumptions on exogenous demand, derived from~\cite{piamanzGeoreferencedIndustrialSites2018,muehlenpfordtTimeSeries2019,mantzosJRCIDEES20152018,NationalEmissionsReported2023,EurostatCompleteEnergyBalance,uwekrienDemandlib2023}. The figure shows the total annual energy demand for each energy source which determine the model's endogenous investments and operation. Endogenous processes can lead to higher total production volumes of some energy carriers, e.g., the demand for methanol requires more hydrogen and carbon as secondary (energy) inputs, which are not considered here. The model are defined per region and time stamp.}
    % TODO: adjust labels to show that these are exogenous assumptions
    \label{fig:total-demand-bar}
\end{figure}
%
In our configuration, the model's time horizon spans one year with a temporal resolution of 3 hours and a spatial resolution of 90 regions. Each of the regions consists of a complex subsystem with technologies for supplying, converting, storing and transporting energy. Exogenous assumptions on energy demand and non-abatable emissions are taken from various sources~\cite{piamanzGeoreferencedIndustrialSites2018,muehlenpfordtTimeSeries2019,mantzosJRCIDEES20152018,NationalEmissionsReported2023,EurostatCompleteEnergyBalance,uwekrienDemandlib2023}. The energy demand for electricity, transport, biomass, heat and gas are defined per region and time-step.
Land transport demand is exogenously divided between electric vehicles (85\%) and fuel cell vehicles (15\%), the latter representing demand for heavy duty land transport. Demands for kerosene for aviation, methanol for shipping, and naphtha for industry are aggregated in the system scope and kept constant throughout all time-steps. Heat demand is regionally subdivided into shares of urban, rural and industrial sites. We show the sum of all energy demands in Fig.~\ref{fig:total-demand-bar}. The system emits 633~Mt \carbon{} per year from industry, aviation, shipping and agriculture, 153~Mt of which are fossil-based process emissions. Industrial energy demand and excess heat potentials are calculated per node on the basis of~\cite{hotmaps_industrial_db}.
% use https://pypsa-eur.readthedocs.io/en/latest/licenses.html


Endogenous model results include the expansion of renewable energy sources, storage technologies, and transmission capacities.
The model considers various energy carriers like electricity, hydrogen, methan, methanol, liquid hydrocarbons and biomass, together with conversions technologies.
Carbon-neutral electricity is provided by wind, solar, biomass, hydro and nuclear power plants. Hydro and nuclear plant capacities cannot be extended. Weather-dependent power potentials for solar, wind and hydro are calculated from reanalysis and satellite data sets~\cite{hersbachERA5GlobalReanalysis2020,pfeifrothSurfaceRadiationData2017}  per region and time-stamp, using the open-source tool Atlite~\cite{hofmannAtliteLightweightPython2021}.
Solar and wind power can be expanded in alignment with eligible land-use restrictions calculated on the basis of~\cite{eeaCorineLandCover2012,eeaNatura2000Data2016}. We restrict the electric transmission system's expansion to 20\% of its current capacity, acknowledging the challenges in inaugurating new transmission projects.
For the use of biomass we consider only residual biomass products and no energy crops. We limit regional biomass use to the medium-level potentials from the ENPRESO database~\cite{enspreso_database,instituteforenergyandtransportjointresearchcentreJRCEUTIMESModelBioenergy2015}. Inter-regional biomass transport is permitted with transport costs considered.

In our model, hydrogen can be produced from electrolysis and steam methane reforming (SMR). Geological distribution and potentials for \hydrogen{} storage in salt caverns are derived from~\cite{caglayanTechnicalPotentialSalt2020}. Re-electrification of hydrogen is possible via fuel cells. If enabled, hydrogen can be transported via pipelines between regions which can be expanded without limit, considering costs for pipeline segments and compressors. Pipeline flows are modelled using net transfer capacities and without flow dynamics, pressure valves, or energy demand for compression.
% No retrofitting of gas pipelines is considered, potentially overestimating hydrogen network cost.

The topology of the \carbon{}, \hydrogen{} and gas network is identical to the topology of the electricity network, connecting all neighboring regions. Transport of liquid fuels like oil, methanol and Fischer-Tropsch (FT) fuels is ``copper-plated'', since transport costs per unit of energy are negligible due to their high energy density. Throughout this study, we focus on \carbon{} and \hydrogen{} networks because of their high relevance for infrastructure investments and thus public policy decisions. The electricity network and gas network are both included in the model with full geographical detail, but not further analyzed. Electricity networks are already in place, and we restrict further extension due to concerns of public acceptance. Gas networks also already exist, and will most likely experience decreased use in the future, removing any bottlenecks or constraints on the optimal system buildout or operation.

Our model features three drop-in fuel production technologies for CU: methanation, methanolization, and Fischer-Tropsch synthesis. The processed fuels are not limited in their total quantity of use or production. Methane is transported through the gas network, while methanol and FT fuel can be transferred inter-regionally without additional costs, since dense fuels have negligible transport costs per unit of energy. Synthetic methane substitutes natural gas or biogas, serving combined heat and power plants, residential heating gas boilers, or industrial process heat. Synthetic methanol decarbonizes marine industry fuel demands, and FT fuels replace fossil oil for naphtha production, aviation kerosene, or agricultural machinery oil.

To supply carbon needed for CS and CU, the system can choose to deploy carbon capture technology at various point sources (see below), or through DAC facilities. The concept of a merit order for captured carbon plays a pivotal role in optimizing and prioritizing the deployment of carbon capture technologies based on their economic feasibility. This merit order ranks all CC technologies according to their relative costs to capture a ton of carbon (see Fig.~X). At the lower end of the cost spectrum, CC technologies applied to process emissions, such as those from cement, offer a cost-effective starting point. Following this, biomass combined heat and power (CHP) systems provide a dual benefit of energy production and carbon capture. Moving up the scale, gas used in industrial applications and biomass employed for industrial processes represent more costly yet viable options to capture carbon. DAC, an emerging technology capable of extracting \carbon{} directly from the atmosphere, stands at a higher cost level due to its current high capital costs and energy demand. Finally, based on our model results, biogas-to-gas upgrading, a process that refines biogas to natural gas quality, incurs the highest costs in the merit order per marginal ton of captured \carbon{}. Biogas input is a high-cost fuel which based on the endogenous modeling decisions is not used in large quantities. To the extent that biogas-to-gas facilities are used by the model to supply additional (carbon-neutral) gas as fuel, adding CC infrastructure incurs low costs (and thus ranges on the left-hand side of the merit order curve). However, the price of capturing additional \carbon{} from biogas-to-gas upgrading is high because a substantial amount of the biogas fuel costs factors into the marginal cost of captured \carbon{}. This merit order framework is crucial for strategically deploying necessary carbon capture solutions while balancing economic considerations. If we consider the spatial aspect of distributed carbon capture potentials, the \carbon{} network can exploit comparative cost advantages between CO2 bidding zones to utilize the lowest-cost CC facilities across the continent. We assume a capture rate of 90\% for CC on process emissions, SMR, biogas-to-gas, as well as gas and biomass used in industry, and 95\% for CC on biomass and gas CHPs.

To store carbon, we differentiate between short-term storage in steel tanks and long-term, irreversible sequestration in underground sequestration sites such as porous rock formations or depleted gas reservoirs. Costs for both options are included in the model. For carbon sequestration, we only consider offshore sites as potential sinks. We make this choice because offshore sequestration sites typically have larger capacity compared to onshore sequestration sites in saline aquifers and due to concerns over public safety for infrastructure near populated areas. Our estimates for carbon sequestration potentials are conservative, limiting total sequestration to 25~Mt per site and calculating annual storage availability over 25 years. Furthermore, we decide to cap the total sequestered \carbon{} to 200~Mt per year for our Net-Zero scenario. This is enough to offset the hardest-to-abate fossil process emissions and some limited slack of fossil fuels to be used where they can be most valuable. This is in order to avoid offsetting fossil emissions with carbon removal if they can technologically be avoided in the first place. All technology cost assumptions are taken for the year 2040 and sourced from an open-source database~\cite{lisazeyenPyPSATechnologydataTechnology2023}.


Addressing climate targets, we define two \carbon{} target scenarios:
\begin{itemize}
    \item[] \textit{Net--Zero}: aligning with the EU's 2050 emission targets. Also denoted as NZ.
    \item[] \textit{Net--Negative}: 10\% net-negative emissions relative to 1990 levels, equaling 460~Mt \carbon{} annually. Also denoted as NN.
\end{itemize}
Unless otherwise mentioned, our reference is the Net-Zero scenario. For the Net-Negative scenario, the cap of total carbon sequestration is adjusted from 200~Mt to 660~Mt per year accordingly.

Beyond \carbon{} targets, we consider four expansion scenarios, referred to as models:
\begin{itemize}
    \item[] \textit{Baseline:} Neither \carbon{} nor \hydrogen{} networks are constructed.
    \item[] \textit{\carbongrid:} Only the \carbon{} network is developed.
    \item[] \textit{\hydrogengrid:} Only the \hydrogen{} network is developed.
    \item[] \textit{Hybrid:} Both \carbon{} and \hydrogen{} networks are developed.
\end{itemize}



\section{Results}
\label{sec:results}


Focusing first on the Net-Zero scenario, we can highlight notable characteristics of the energy system buildouts shared by all models, as well as significant variations in system costs and technology deployment between the models (see Fig.~\ref{fig:cost_bar}).

\begin{figure}[ht!]
    \centering
    \includegraphics[width=\linewidth]{comparison/default/figures/90_nodes/cost_bar.png}
    \caption[short]{Total annual system cost subdivided into groups of technologies for the different models of the European energy system with a net-zero emission target. While the Baseline model has neither a \carbon{} nor a \hydrogen{} network, the Hybrid model is allowed to expand both. ``Gas Infrastructure'' combines gas facilities for power and heat production, ``\carbon{} Infrastrucure'' combines transport, storage and sequestration, and ``H$_2$ Infrastructure'' combines transport and storage. ``Carbon Capture at Point Sources'' combines all technologies with integrated carbon capture, including the cost of the main facility (e.g., CHP unit) and the carbon capture application.}
    \label{fig:cost_bar}
\end{figure}

Total annual system costs in the \modBase{} are highest, totalling \label{}803 billion euros. The models with additional transport options for carbon and hydrogen achieve a more efficient system allocation and lower total system costs, with around \label{}2\% cost reduction (\label{}781 billion euros) in the \modCO{}, and around \label{}4\% cost reduction in the \modH{} and the \modHybrid{} (\label{}765 and \label{}764 billion euros, respectively).

In all models, almost half of the system costs are spent on primary electricity production from wind and solar power (\label{}45\% or \label{}360 of \label{}799 billion euros total costs in the Baseline model), with another \label{}12\% (\label{}100 billion euros) spent on hydro, nuclear and biomass. Costs for energy transport networks for electricity, gas, hydrogen and carbon are below \label{}12\% (\label{}100 billion euros), with the electricity grid making up more of the system costs than the other three networks combined. This is the case for all four models. Capital expenditures on carbonaceous fuel production are around \label{}3\% (\label{}25 billion euros) in each model, while electrolysis makes up around \label{}5\% (\label{}40 billion euros). Heat pump installations account for around \label{}8\% (60 billion euros) of total system costs.

System cost differences between the models are driven by renewable energy resources (solar, wind, and bioenergy), the \carbon{} and \hydrogen{} networks, as well as carbon capture technologies (DAC and CC at point sources).
As we show below, these are due to substantial, albeit fundamentally different, gains in flexibility and a shift in carbon utilization and storage strategies by the \hydrogen{} and \carbon{} networks.

\subsection*{Transport Systems Unlock Point\=/Source Carbon Capture}

In comparison to the other three models, the \modBase{} reveals significant inefficiencies: On the one hand, regions with no access to sequestration sites (i.e., ``landlocked'' regions) underutilize CC potentials from process emissions, fossil gas and biomass. On the other hand, sequestration sites rely on costly DAC to capture the carbon for CS from the air. Furthermore, hydrogen and power prices show high regional differences, indicating system inefficiencies due to transport constraints caused by non-existent hydrogen or insufficient power transmission. Prices are significantly higher in Central Europe, a region with high population density and high concentration of energy-intensive industries.

By enabling the efficient transportation of carbon, hydrogen, or both, these models reduce the reliance on high-cost DAC, and source more carbon from low-cost point-source CC on biomass and gas in industrial applications (see Fig.~\ref{fig:balance_captured_carbon}). The \modH{} only captures \label{}200~Mt of \carbon{} from DAC compared to \label{}400~Mt in the \modBase{}. Instead, the \modH{} captures more carbon from process emissions as well as gas and biomass for industry. The \modCO{} uses only \label{}150~Mt from DAC, and instead also captures more carbon than the \modH{} from gas for industry and biomass CHPs. And finally, the \modHybrid{} captures only \label{}100~Mt from DAC, but slightly more carbon from all other CC technologies.
Also, the networks reduce the reliance on methanation in the model, reducing the overall demand for CC volumes. Another knock-on effect of the reduced reliance on DAC is reduced endogenous demand for renewables to power DAC facilities.


\begin{figure}[ht!]
    \centering
    \includegraphics[width=\linewidth]{comparison/default/figures/90_nodes/balance_bar_carbon.png}
    \caption{Balance of captured carbon for all models in the Net-Zero scenario. Positive values indicate carbon capture, negative values indicate carbon consumption. By integrating \hydrogen{} and \carbon{} networks, the predominant method for carbon removal shifts from Direct Air Capture (DAC) to a bioenergetic process that incorporates carbon capture. At the same time the reliance on methanation decreases.}
    \label{fig:balance_captured_carbon}
\end{figure}



The increase in captured carbon from point sources in the network models is mostly due to CC infrastructure being deployed on industrial or power plant facilities that are present in all models (as opposed to the deployment of additional electricity generating capacity for the purpose of carbon capture). This is evident when analyzing the capture share of the different CC technologies (see Fig.~\ref{fig:captureshare_line}).


\begin{figure}[h]
    \centering
    \includegraphics[width=\linewidth]{comparison/default/figures/90_nodes/captureshare_line.png}
    \caption{Share of facilities with integrated carbon capture for all models .}
    % A CO2 network unlock BECSS potentials, most biomass emissions are captured at point sources and transported to sequestration sites. cite~\cite{rosaAssessmentCarbonDioxide2021}
    \label{fig:captureshare_line}
\end{figure}%

In the \modBase{}, the capture share is highest for biogas-to-gas facilities, followed by process emissions, biomass for industry, biomass CHPs, gas for industry, SMR and gas CHPs. For the network models, we can observe a correlation between CC share and capacity factors of the underlying technologies: Process emissions as well as gas and biomass for industry have high capacity factors (above 80\%), and CC shares close to 100\% (the only exception being gas for industry in the \modH{}). Biomass CHPs have a capacity factor of around 40\%, and CC shares of 50\% in the \modH{} and 95\% in the \modCO{}. SMR and gas CHPs have capacity factors below 20\%, and CC shares around 50\% and 0\%, respectively, across all network models. Gas CHPs serve as peak load electricity production technology, and with very few operating hours, the high investment costs into CC applications are not efficient.\footnote[1]{Similarly, SMR serves as ``peak-load'' hydrogen production technology in regions with poor renewable resources for electrolysis.}


In summary, the implementation of \carbon{} and \hydrogen{} networks is instrumental in unlocking low-cost CC potentials at geographically fixed point sources across the continent. While the \carbon{} network facilitates the movement of carbon from low-cost sources to CU and CS sinks, reducing the reliance on high-cost DAC, the \hydrogen{} network enables the cost-effective delivery of hydrogen from regions rich in renewable energy to regions with low-cost CC potentials to produce CU. However, despite the \carbongrid{} model demonstrating higher absolute amounts of CC from point sources and higher CC shares across all point-source CC technologies, the \hydrogengrid{} model achieves lower total system costs (see Fig.~\ref{fig:cost_bar}). In the following section, we will explore the reasons for this.

% [potentially relevant text blobs from old 3.1]
% CU is preferably deployed in the regions where carbon, hydrogen as well as the inputs and by-product outputs (e.g., waste heat) of the synthesis processes have the lowest cost (or highest value).

% Now we turn to how carbon for CS is sourced by the different models. In the \modBase{}, because of the absence of a carbon network, all carbon that is sequestered (200 Mt, the same in each model) needs to be captured in the regions with access to an offshore CS site. The \modBase{} deploys all available low-cost point-source CC in these regions, and adds DAC in the sequestration locations with the lowest-cost electricity. In other words, the \modBase{} uses the atmosphere as a substitute carbon transport pathway, allowing \CO{} from inland point sources to escape into the atmosphere, and extracting it again from the atmosphere with DAC at the offshore locations.


\subsection*{\hydrogen{}--Network enables decentral carbon utilization}


\begin{figure*}[ht!]
    \centering
    \includegraphics[width=\linewidth]{comparison/single-technologies/figures/90_nodes/balance_map_dedicated.png}
    \caption{Optimal operation per sector for a net-zero energy system in Europe with average production on the left and average consumption on the right for both, (a) the \carbon{} sector in the \carbon{}-Grid model and (b) the \hydrogen{} sector in the \hydrogen-Grid model.}
    \label{fig:balance_map}
\end{figure*}


% Examining the cost-reductions across the different models raises an important question: why is the hydrogen network more cost-effective than its carbon counterpart, and how does it achieve similar cost reductions as the hybrid model? An initial hypothesis may be that the cost assumptions for \carbon{}
%  and \hydrogen{} networks play a role. However, the amount of system costs spent on either network is substantially lower than the cost difference between the models (20 billion euros less total system costs in the \hydrogengrid{} model, while it spends 10 billion euros on the \hydrogen{} grid and the \carbongrid{} model spends 8 billion euros on the \carbon{} grid). Furthermore, we have conducted a sensitivity analysis with cost assumptions for the \carbon{} network reduced to 50\% of the default assumptions, and the results show that the \hydrogen{} network still outperforms the \carbon{} network in terms of total system costs.
To find the reason for why the \hydrogen{} network achieves lower system costs than the \carbon{} network, it is crucial to understand the underlying mechanisms and geographical implications of the two transportation systems. Fig.~\ref{fig:balance_map} maps the average production, consumption and flow in the carbon (upper row) and hydrogen (lower row) sectors for the \carbongrid{} (left) and \hydrogen{} (right) model. Each map includes circles indicating the production (top half-circle) and consumption (bottom half-circle) for each region, with lines representing the import/export volumes and flow directions. The color of each region corresponds to the average price of the carrier.


The \carbongrid{} model captures large amounts of low-cost carbon from bioenergy and industrial activities, especially in Central and Eastern Europe (top left), almost none of which is used locally. A large part is transported from Western Europe to the Iberian Peninsula, the British Isles and Denmark, where it is used to supplement low-cost hydrogen in central plants for electrolysis, Fischer-Tropsch and methanolization (bottom left). A second part is transported to methanation plants such as those in Italy and Greece, from where the gas is fed into the gas grid, which also is set up for distribution from large gas import hubs. Finally, the third part is transported directly to sites in the North Sea, Baltic Sea and Mediterranean Sea where it is eventually enriched with \carbon{} from DAC to be finally sequestered. In absence of a \hydrogen{} network, the system distributes electrolysis where it is needed for industrial processes and fuel cell land transport and emphasizes gas imports into Central Europe.

The \hydrogengrid{} on the other hand, transports low-cost hydrogen from regions with abundant renewable resources to regions with high energy demand and abundant emissions (lower right). The main transportation routes go from Spain and the United Kingdom to Central Europe. In these regions, industries use the hydrogen for demand in industry and fuel cell land transport, and to produce synfuels with locally captured carbon from point sources, leading to a decentralized CU system across the continent. DAC and local emissions nearby the coast enable sequestration in the North Sea, Baltic Sea, Mediterranean Sea and, in particular, in Portugal (upper right).
% Furthermore, the introduction of a hydrogen network enables to reduce both DAC production and FT fuels on the British Isles and Iberian Peninsula. In Great Britain an Ireland, the remaining local carbon is then sequestered, and all the hydrogen that previously went into FT fuel production is exported to Central Europe (see above). In Spain and Portugal, some DAC remains and the carbon sourced from DAC is used, along with locally captured carbon from point sources, to produce FT fuels (but less than without a hydrogen network). Surplus hydrogen from lower FT fuel production is exported to Central Europe.

The results illustrate the fundamentally different approaches of the grids: a \carbon{} grid prefers a decentralized CC system and a co-location of electrolysis and CU centrally in regions with abundant renewable resources; a \hydrogen{} grid prefers a centralized electrolysis system and a co-location of decentralized CC and CU systems across the continent. A \hydrogen{} network not only unlocks the potential of carbon capture from point sources, but also reduces the energy shortages in energy-intensive areas.



\subsection*{Hybrid configuration provides further flexibilities}

\begin{figure*}[ht!]
    \centering
    \begin{subfigure}{.49\textwidth}
        \centering
        \includegraphics[width=\linewidth]{full/figures/90_nodes/balance_map_hydrogen.png}
        \caption{\hydrogen{} Sector.}
        \label{fig:capacity_map_hydrogen_co2}
    \end{subfigure}
    \hfill
    \begin{subfigure}{.49\textwidth}
        \centering
        \includegraphics[width=\linewidth]{full/figures/90_nodes/balance_map_carbon.png}
        \caption{\carbon{} Sector.}
        \label{fig:capacity_map_carbon_co2}
    \end{subfigure}%
    \caption{Optimal production and transport capacities of the carbon and hydrogen sector in a net-zero energy system in Europe with both \carbon{} and \hydrogen{} network expansion (Hybrid).
    % Carbon network looks the same as in~\cite{morbeeOptimisedDeploymentEuropean2012}: two backbones, one in the nothern Europe other in south east.
    }
    \label{fig:capacity_maps}
\end{figure*}

The Hybrid model combines the advantages of hydrogen and carbon networks. While the topology of the \hydrogen{}-grid roughly corresponds to that of the \hydrogengrid{} model, the \carbon{}-grid plays more of a complementary role. Fig.~\ref{fig:capacity_maps} shows the average production, consumption and transport of the two grids in the hybrid model. Note that despite similar overall capacities, the investment in the hydrogen infrastructure is four times the investment in the carbon infrastructure.


Similar to the \hydrogengrid{} model, the Hybrid model takes advantage of a large hydrogen network that enables hydrogen to be transported from centralized production sites in western regions such as Spain to decentralized FT production sites across the continent. However, in the Hybrid model hydrogen in the UK is used locally to produce synfuels, and less hydrogen is transported to Central Europe.
The carbon network supplements this with smaller clusters of networks. In addition to a larger carbon grid cluster in Central Europe, which transports carbon to the North Sea, the system is building three large linear carbon routes. These transport carbon from Romania and Bulgaria to Greece, from northern Italy to central Italy and from Hungary, the Czech Republic and Poland to the Baltic Sea. These enable the system to transport carbon available at low cost from inland biomass sources and sequester it (see Fig.~\ref{fig:captureshare_line}). This reduces the need for DAC in the system and the sequestration sites are shifted to the end points of the carbon networks. Second-order effects can be seen in the shift in FT production compared to the \hydrogengrid{} model. Regions that have both good renewable resources and sequestration potential could now forego sequestration and use local process emissions for FT production, as can be observed in Ireland. A notable aspect of the hybrid model is the lack of overlap in the hydrogen and carbon network topologies. This finding is tightly linked to the fact of competing functionalities to offtake carbon emissions.
A more detailed of differences between the \hydrogengrid{} and the Hybrid model can be found in Fig.~.


Despite only small decrease in cost in comparison to the \hydrogengrid{} model, hybrid approach brings several advantages. These include a broader range of technologies contributing to its robustness, reduced reliance on DAC, which may be costlier than anticipated in the cost projections, and less land use for wind and solar due to the decreased necessity for DAC. The increase used of biomass with carbon capture, transport and sequestration leads to slight cost advantages in comparison to the \hydrogengrid{} model.


\subsection*{Net-negativity amplifies DAC and Bioenergy with CU and CS}

\begin{figure}[htb!]
    \centering
    \includegraphics[width=0.9\linewidth]{difference/comparison/emission-reduction-0.1/figures/90_nodes/cost_bar.png}
    \caption[short]{Net change in investments when tightening the \carbon{} emission target from net-zero to net-negative 10\% of 1990s emissions.}
    \label{fig:net-negative_cost_bar}
\end{figure}


In transitioning from a net zero to a net emissions reduction target, a continuation of investment trends are perceived. Fig.~\ref{fig:net-negative_cost_bar} shows the net investment changes per technology group for the \hydrogengrid{} and Hybrid. The scenario imposes a net carbon removal of 460 Mt/a, with both models incurring a similar net cost increase - 95 billion €/a for \hydrogengrid{} and 96 billion €/a for Hybrid, balancing the earlier cost advantage of the Hybrid model. Both models invest relatively evenly in additional DAC, wind and heat pump systems, with DAC accounting for more than a third of the additional system costs. These remove carbon directly at sequestration sites at the coast, in particular in Portugal and UK, together with heat and power from new heat pumps at site and new solar and wind installation in the vicinity.
More heat pumps, solar and wind installations across all Central and Eastern European countries as well as biogas-to-gas installations compensate for the reduction of fossil gas based heating and powering. Reduced usage of gas in general lower the investments in gas infrastructure in both models.
Fundamental difference in investment strategies occur in the carbon capture sector: In contrast to the \hydrogengrid{} model, the Hybrid model continues the strategy of sequestering carbon captured and transported from bio-energetic inputs, expanding the carbon transport routes from Central and Eastern Europe to the near shores. At the same time, carbonaceous fuel production in Central Europe partially moves to Spain and UK, where it uses carbon from new DAC facilities without the installation of a \carbon{} network. At the same time, solar based electrolysis in France, Nothern Italy and Switzerland expands requiring more solar capacities but less hydrogen imports from the Iberian Peninsula.
The \hydrogengrid{} model on the other hand, expands decentralized CU systems from the Net-Zero scenario, which are now increasingly supplied with carbon from biogas-to-gas facilities, showing up in both "Bioenergy" and "Carbon Capture at Point Sources" cost-contributions. The centralized DAC facilities at sequestration sites are more expanded than in the Hybrid model and are requiring additional heat from biomass CHPs.


These findings highlight the continuation of trends of the two systemic approaches, which are interchangable at marginal cost-difference: While a stand-alone \hydrogen{} grid combines well with highly centralized DAC and decentralized CU with bioenergy as primary carbon source, the complementation with a \carbon{} grid promotes combining DAC and CU where good renewable resources are and collecting, transporting and sequestering carbon from decentral bioenergetic processes.

% The difference in absolute investment changes between the two models, regionally displayed in Fig.~\ref{fig:net-negative_cost_map}.
% \begin{figure*}[htb!]
%     \centering
%     \includegraphics[width=\linewidth]{difference/net-negative-0.1-h2-only-full/figures/90_nodes/cost_map.png}
%     \caption[short]{Net difference in investments between the \hydrogengrid{} model and the Hybrid model for the Net-Negative scenario.}
%     \label{fig:net-negative_cost_map}
% \end{figure*}


% \section{Limitations}
% \label{sec:limitations}
% Despite the detailed model representation, there are some limitations to the validity of the results. The model itself is based on a linearized optimization with perfect foresight for the entire modeling year. In reality, long-term energy demand and renewable supply can only be roughly estimated, while short-term predictions still entail some uncertainty. The model's perfect foresight may lead to non-reproducible behavior, such as precisely aligning energy storage with future energy shortages at a specific point in time.
% The technology costs used in the model rely on cost projections that incorporate reductions based on learning rates. These learning rates are derived from historical data, which may not necessarily be indicative of future trends. In particular, cost assumptions on DAC, electrolysis and \hydrogen{} pipeline costs, could have a strong impact in the model results.
% The modelling results are heavily driven by the demand and emissions from industrial clusters. Allowing the model to relocate these and/or incorporate flexibility measures, may lead to less dependency on both \carbon{} and \hydrogen{} networks.
% The technological flexibilities in the carbon sector might not be exploited to full extent. Therefore, the \carbon{} transport via truck or ship is not considered. Fischer-Tropsch facilities require to run on baseload with at least 90\% of their nominal capacity.
% Furthermore, the model assumes a fixed rate for industrial process emissions, which cannot be altered through investments. This simplification may not accurately represent real-world scenarios, where many industries are considering adopting low-carbon processes and technologies. The transport of fuel (FT, gas, oil) between the regions is not unlimited which may overestimate the flexibility provided by these commodities.


\section{Conclusion}
\label{sec:conclusion}

This study offers a detailed exploration of Europe's energy system in 2050, focusing on the integration and optimization of hydrogen and carbon networks to achieve climate neutrality. The analysis reveals significant efficiencies gained through these networks, particularly in terms of leveraging low-cost carbon sources and facilitating a decentralized approach to carbon utilization.

The hydrogen network demonstrates superior cost-effectiveness, primarily due to its capacity to transport low-cost hydrogen to high-demand regions and promote decentralized carbon capture and utilization. This network reduces reliance on direct air capture (DAC), offering a shift towards more economically viable and sustainable solutions.
The combination of hydrogen and carbon networks in the hybrid model introduces additional flexibilities. This model not only achieves further cost reductions but also shows resilience against potential increases in DAC costs and reduces land use for renewable energy installations.
Under net-negative emission scenarios, the importance of DAC and bioenergy with carbon capture and storage (BECCS) is amplified. Both the hydrogen and hybrid models demonstrate adaptability to these more stringent climate targets, highlighting the robustness of the networked approaches.

% TODO: distribution of current refineries might shape the CU network. These are quite decentralized and might fit to the cost-optimal layout.

The study conclusively demonstrates the pivotal role of strategically integrated hydrogen and carbon networks in steering Europe towards a sustainable and climate-neutral energy future. It emphasizes the necessity of comprehensive infrastructure planning, integrating diverse energy carriers and technologies to achieve these objectives effectively.


\printbibliography

\appendix


% \begin{figure*}
%     \centering
%     \includegraphics[width=\linewidth]{full/figures/90_nodes/sankey_diagramm.png}
%     \caption{Sankey diagram of the optimal operation for a net-zero scenario.}
%     \label{fig:sankey_diagramm}
% \end{figure*}

% \begin{figure*}
%     \centering
%     \includegraphics[width=\linewidth]{net-negative-0.1/full/figures/90_nodes/sankey_diagramm.png}
%     \caption{Sankey diagram of the optimal operation for a net-negative 10\% scenario.}
%     \label{fig:sankey_diagramm}
% \end{figure*}

\begin{figure}
    \centering
    \includegraphics*[width=0.8\linewidth]{comparison/emission-reduction/figures/90_nodes/objective_heatmap.png}
    \caption{Total annual system cost for the different scenarios and \carbon{} targets. While the `Baseline' scenario has neither a \carbon{} nor a \hydrogen{} network, `Hybrid' is allowed to expand both.}
    \label{fig:objective_heatmap}
\end{figure}


\begin{figure}[ht]
    \centering
    \includegraphics[width=\linewidth]{comparison/emission-reduction-full/figures/90_nodes/cost_bar_transmission.png}
    \caption{Annual transmission system cost as a function of the net carbon removal scenarios considered in the study.}
    \label{fig:cost_bar_transmission}
\end{figure}


\begin{figure*}[ht]
    \centering
    \includegraphics[width=\linewidth]{difference/h2-only-full/figures/90_nodes/cost_map.png}
    \caption{Difference in cost investments between \hydrogengrid{} and Hybrid model. The left subfigure shows higher spendings per technology and region and transport system for the \hydrogengrid{} model, the right shows higher spendings in the Hydrid model.}
    \label{fig:cost_map_difference}
\end{figure*}


\end{document}

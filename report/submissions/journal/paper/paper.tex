\documentclass[twocolumn]{article}
\usepackage{amsmath,amssymb,amsfonts}
\usepackage{cuted}  % Add the cuted package
\usepackage{caption}
\captionsetup{font=small}
\usepackage{algorithmic}
\usepackage{graphicx}
\usepackage{textcomp}
\usepackage{xcolor}
\usepackage{tabularx, multirow}
\usepackage{fancyhdr,lipsum}
\usepackage{subcaption}

\usepackage[%
backend=biber,bibencoding=utf8, %instead of bibtex
language=auto,
style=ieee,
sorting=none, % nyt for name, year, title
maxbibnames=10, % default: 3, et al.
%backref=true,%
natbib=true % natbib compatibility mode (\citep and~\citet still work)
]{biblatex}
\bibliography{../../../references.bib}

%define approx proportional
\def\app#1#2{%
  \mathrel{%
    \setbox0=\hbox{$#1\sim$}%
    \setbox2=\hbox{%
      \rlap{\hbox{$#1\propto$}}%
      \lower1.1\ht0\box0%
    }%
    \raise0.25\ht2\box2%
  }%
}
\def\approxprop{\mathpalette\app\relax}

% make abbreviation for co2
\newcommand{\carbon}{CO$_2$}
\newcommand{\hydrogen}{H$_2$}
\newcommand{\hydrogengrid}{\hydrogen{}--Grid}
\newcommand{\carbongrid}{\carbon{}--Grid}
\newcommand{\scbase}{Baseline scenario}
\newcommand{\scCO}{CO$_2$~Grid scenario}
\newcommand{\scH}{H$_2$~Grid scenario}
\newcommand{\schybrid}{Hybrid scenario}

\graphicspath{
    % {paper-figures}
    {figures}
    {../../../../results/}
    {results/}
}

\begin{document}


% FORMATTING: https://www.nature.com/nenergy/submission-guidelines/aip-and-formatting


\title{From Net-Zero to Net-Negative: Assessing \hydrogen{} and \carbon{} Network Strategies in Europe}

\author{
    Fabian Hofmann, Christoph Tries, Fabian Neumann, Lisa Zeyen, Tom Brown \\
    \textit{Institute of Energy Technology} \\
    \textit{Technical University of Berlin}\\
    Berlin, Germany \\
    m.hofmann@tu-berlin.de
}


\maketitle

\begin{abstract}
    The transition to a carbon-neutral European economy is a pressing challenge that demands coordinated action across various energy sectors, particularly in emissions-intensive industries like heavy manufacturing. While management of both \hydrogen{} and \carbon{} is considered a critical component of this transition, a gap exists in understanding how new hydrogen infrastructures effectively interact with comprehensive carbon management technologies, including carbon capture, transport, use, and storage. In particular, the role of a carbon transport network as a complementing or substituting infrastructure to hydrogen transport remains unclear. To address this gap, our study employs optimization techniques to design the first cost-optimal European energy system that fully incorporates carbon management technologies, hydrogen transport and storage, and renewable energy sources with a high spatial and temporal resolution considering all energy intensive sectors.
    Our findings reveal that in a either-or scenario, the hydrogen network is superior to the carbon network as it enables the transportation cheap hydrogen inland to serve industrial processes and carbon capture and utilization at point sources. A carbon network facilitates point-source carbon capture and reduces the need for direct air capture, but leaves the scarcity of hydrogen in energy-intensive regions such as industrial clusters.
    However, in a hybrid scenario, the carbon network is a cost-effective complement to the hydrogen network facilitating carbon capture at point sources and transport to sequestration sites.
    We show that theses findings holds true against the backdrop of a net-zero energy system, as well as scenarios where the net-negative emissions are 5\% and 10\% lower than the 1990 baseline levels.
    Overall, our work demonstrates the cost-effectiveness of a multi-grid system that includes both hydrogen and carbon transport networks and power grid expansion to achieve climate neutrality in Europe.
    The paper underpins the need for a complementing hydrogen network and \carbon{} network to achieve climate neutrality.
\end{abstract}



An integrated management of both hydrogen and carbon is considered a crucial component to achieve climate neutrality in energy systems including hard-to-abate sectors. Hydrogen offers an efficient way to transport and store energy from different sources. Carbon, on the other hand, can be effectively captured from industrial processes and the combustion of both fossil and synthetic fuels through Carbon Capture (CC) techniques or harvested directly from the atmosphere using Direct Air Capture (DAC). Additionally, it can be stored in geological formations, a process known as carbon sequestration or carbon capture and storage (CCS). In combination, both together build the basis for climate-neutral fuels like methane, methanol or Fischer-Tropsch fuels, central to Power-to-X (PtX) and Carbon Capture and Utilization (CCU) strategies.

Recently, policymakers and industry in Europe have been committing to both hydrogen and carbon management projects, planning infrastructure components, and developing business models for emerging sectors of the economy. With the European Union's goal of achieving climate neutrality by 2050, a wide range of programs, funding models and initiatives have been established in this regard~\cite{eu2023netzero,europeangreendeal,europeaninnovationfund}. Initiatives like the European Hydrogen Backbone~\cite{gasforclimateEuropeanHydrogenBackbone2022} or the Hydrogen infrastructure Map~\cite{H2InfrastructureMap} showcase the potential of hydrogen as a fuel and energy carrier. At the same time, business models from companies like Tree Energy Solutions~\cite{TESHydrogenLife2023}, Carbfix~\cite{WeTurnCO2}, and Equinor~\cite{adomaitisEquinorRWEBuild2023} advertise carbon management hubs that provide green hydrogen, methane, or synfuels on the one hand and offer \carbon{} offtake on the other hand. For Europe, the Capture Map~\cite{ToolsGreenTransition} estimates a potential of 1.4~Gt of carbon capture from point sources per year. In combination with large sequestration potentials as stated in~\cite{weiProposedGlobalLayout2021}, this highlights the vast potential for decarbonization. The Northern Lights project in Norway~\cite{NorthernLightsWhat} is planning with a 1.5 Mt/a co2 transport and sequestration capacity to be operating in 2024, expanding to a targeted scale of 5 Mt/a sequestration by 2030.
To this end, the Clean Air Task Force underlines the importance of a carbon transport system in Europe to facilitate the carbon economy~\cite{lockwoodEuropeanStrategyCarbon}.


However, up to this point, it remains unclear how the two management systems of hydrogen and carbon may complement or replace each other, particularly when it comes to their transport systems. A hydrogen network facilitates the transport of energy from renewable sources to regions with high energy demand, also enabling Carbon Capture and Utilization (CCU) from specific point sources. On the other hand, the carbon transport system allows for capturing and transporting carbon from areas with high emissions to regions with significant sequestration potential and abundant renewable sources, thereby enabling cost-effective CCU.
In the literature, the two network approaches and underlying technologies have been discussed in a number of publications, all of which, however, dealt with the isolated effects~\cite{bakkenLinearModelsOptimization2008,morbeeOptimisedDeploymentEuropean2012,stewartFeasibilityEuropeanwideIntegrated2014,oeiModelingCarbonCapture2014,elahiMultiperiodLeastCost2014,burandtDecarbonizingChinaEnergy2019,middletonSimCCSOpensourceTool2020,bjerketvedtOptimalDesignCost2020,weiProposedGlobalLayout2021,damoreOptimalDesignEuropean2021,becattiniCarbonDioxideCapture2022}. In contrast to Integrated Assessment Models, the underlying optimization tools account for the spatial distribution of carbon sources and sinks which are crucial provide a holistic view of the energy system and its technological interactions.. The work in~\cite{neumannBenefitsHydrogenNetwork2022} examines the effect of a hydrogen network in Europe... (extend on hydrogen literature)
The publication by Morbee et al.~\cite{morbeeOptimisedDeploymentEuropean2012} optimizes the topology and capacity of a \carbon{} network in Europe, but only considers the power sector without co-optimizing renewable deployment.
Another comprehensive example is found in~\cite{becattiniCarbonDioxideCapture2022}, which presents a mixed-integer model to optimize the time-evolution of a \carbon{} transport system in Switzerland, connecting to a remote sequestration site in Norway. To this end, it is often argued that \carbon{} pipelines are a mature technology with an expected high learning rate, given the wide-spread installations in US and Canada for enhanced oil recovery~\cite{righettiSitingCarbonDioxide2017,friedmannNETZEROGEOSPHERICRETURN}.
% However, the models are often limited with regard to both geographical scope and detail. While representing a single country with spatial resolution may neglect synergies of international cooperation, a coarse grained representation of multiple countries may neglect important geographical properties.

To our knowledge, no study has yet considered the co-optimization and comprehensive assessment of both \hydrogen{} and \carbon{} networks in a fully sector-coupled energy system. However, we argue that such an assessment is strongly needed to avoid suboptimal investments and to identify synergies between hydrogen and carbon management technologies. In this paper, we present a detailed study of the European energy system for 2050, which includes high geographical resolution and a comprehensive representation of carbon management technologies. The study is conducted using the PyPSA-Eur energy system model and encompasses all relevant energy sectors. We enable the system to optimize the design of renewable energy sources and storage technologies, as well as the transmission of electricity, \hydrogen{}, methane, and \carbon{}. We show that the dynamics of carbon capture, transport, use, and storage in different regions are highly dependent on the availability of renewable energy, \carbon{} pipelines and sequestration sinks. Our evaluation focuses on the transport dynamics of \carbon{} and \hydrogen{} through their respective networks on the European continent. We also analyze how an energy system with limited annual sequestration potential prioritizes decarbonization and fuel switching in various sectors, and how the construction of carbon networks varies based on different levels of available sequestration potential.


\section{Methodology}
\label{sec:methodology}

The study is conducted on the basis of the open-source, capacity-expansion model PyPSA-Eur~\cite{horschPyPSAEurOpenOptimisation2018,brownSynergiesSectorCoupling2018,PyPSAEurSecSectorCoupledOpen2023}.
The model optimizes the design and operation of the European energy system, encompassing the power, heat, industry, waste, and transport sectors.

\begin{figure}
    \includegraphics[width=\linewidth]{baseline/figures/90_nodes/total-demand-bar.png}
    \caption{Exogenous demand assumptions. The figure shows the total amount of yearly energy demands for each energy carrier and type of end-consumer considered in the model.}
    \label{fig:total-demand-bar}
\end{figure}
%
In our configuration, the model's time horizon spans one year with a temporal resolution of 3 hours and a spatial resolution of 90 regions. Each of the regions consists of a complex subsystem with technologies for supplying, converting, storing and transporting energy. Exogenous assumptions on energy demand and non-abatable emissions are taken from various sources~\cite{piamanzGeoreferencedIndustrialSites2018,muehlenpfordtTimeSeries2019,mantzosJRCIDEES20152018,NationalEmissionsReported2023,EurostatCompleteEnergyBalance,uwekrienDemandlib2023}. The energy demand for electricity, transport, biomass, heat and gas are defined per region and time-step with heat demand being regionally subdivided into shares of urban, rural and industrial sites. Demands for aviation kerosene, methanol for shipping and naphtha for industry are aggregated in the system scope and kept constant throughout all time-steps. We show the sum of all energy demands in \ref{fig:total-demand-bar}. The system emits of 633~Mt \carbon{} per year through non-abatable emissions from industry, aviation, shipping and agriculture into account, 153~Mt of which are fossil-based industrial emissions. Industrial energy demand and excess heat potentials are calculated per node on the basis of~\cite{hotmaps_industrial_db}.
% use https://pypsa-eur.readthedocs.io/en/latest/licenses.html
%
Endogenous assumptions include the expansion of renewable energy sources, storage technologies, and transmission capacities.
The model considers the various energy carriers like electricity, hydrogen, methan, methanol, liquid hydrocarbons and biomass, together with conversions technologies to convert these into each other.
Carbon-neutral electricity is provided by wind, solar, biomass, hydro and nuclear power plants. Weather-dependent power potentials for solar, wind and hydro are calculated from reanalysis and satellite data sets~\cite{hersbachERA5GlobalReanalysis2020,pfeifrothSurfaceRadiationData2017}  per region and time-stamp, using the open-source tool Atlite~\cite{hofmannAtliteLightweightPython2021}.
Solar and wind power can be expanded in alignment with eligible land-use restrictions calculated on the basis of~\cite{eeaCorineLandCover2012,eeaNatura2000Data2016}. We restrict the electric transmission system's expansion to 20\% of its current capacity, acknowledging the challenges in inaugurating new transmission projects.
The regional use of biomass is restricted by its feasible potential derived from~\cite{enspreso_database,instituteforenergyandtransportjointresearchcentreJRCEUTIMESModelBioenergy2015}. Inter-regional biomass transport is permitted with transport costs considered.

Hydrogen can be gained from electrolysis and steam methane reforming (SMR). Geological distribution and potentials for \hydrogen storage in salt caverns are derived from~\cite{caglayanTechnicalPotentialSalt2020}. Re-electrification of hydrogen is possible via fuel cells. If enabled, hydrogen can be transported via pipelines between regions which can be expanded without limit. The topology of the \hydrogen network is mirroring the topology of the electricity network.

Our model features three drop-in fuel production technologies for carbon utilization: methanation, methanolization, and Fischer-Tropsch synthesis. The processed fuels are not limited in their total quantity of use or production. Methane is transported through the gas network, while methanol and FT-synfuel can be transferred inter-regionally without additional costs. Synthetic methane substitutes natural gas or biogas, serving combined heat and power plants, residential heating gas boilers, or industrial gas needs. Synthetic methanol decarbonizes marine industry fuel demands, and FT fuels replace fossil oil for naphtha production, aviation kerosene, or agricultural machinery oil.

To supply carbon needed for CS and CU, the system can choose to deploy carbon capture technology at various point sources, or centralized DAC facilities. The concept of a ``merit order for captured carbon`` plays a pivotal role in optimizing and prioritizing the deployment of carbon capture technologies based on their economic feasibility. This merit order ranks all CC technologies according to their relative costs to capture a ton of carbon (see Fig.~X). At the lower end of the cost spectrum, CC technologies applied to process emissions, such as those from cement, offer a cost-effective starting point. Following this, biomass combined heat and power (CHP) systems provide a dual benefit of energy production and carbon capture. Moving up the scale, gas used in industrial applications and biomass employed for industrial processes represent more costly yet viable options to capture carbon. DAC, an emerging technology capable of extracting \carbon{} directly from the atmosphere, stands at a higher cost level due to its current technological and operational expenses. Finally, biogas-to-gas upgrading, a process that refines biogas to natural gas quality, typically incurs the highest costs in the merit order per ton of captured \carbon{} but offers a flexible (yet carbon emitting) energy source as a product. This merit order framework is crucial for strategically deploying necessary carbon capture solutions while balancing economic considerations.

We only consider offshore sites as potential sinks for carbon sequestration. This choice favors offshore storage due to its typically larger capacity compared to onshore and concerns over public safety for infrastructure near populated areas. Our estimates for storage potential are conservative, limiting total potential to 25~Mt per site, calculating annual storage availability over 25 years, and capping the total sequestered \carbon{} to 200~Mt/a. All technology cost assumptions for 2040 are sourced from an open-source database~\cite{lisazeyenPyPSATechnologydataTechnology2023}.


Addressing climate targets, we define two \carbon{} target scenarios:
\begin{itemize}
    \item[] \textit{Net--Zero}: aligning with the EU's 2050 emission targets. Also denoted as NZ.
    \item[] \textit{Net--Negative}: 10\% net-negative emissions relative to 1990 levels, equaling 460~Mt \carbon{} annually. Also denoted as NN.
\end{itemize}
Unless otherwise mentioned, our reference is the Net-zero scenario. For 5\% Net-negative and 10\% Net-negative scenarios, the carbon sequestration potential is adjusted to 400~Mt/a and 600~Mt/a, respectively.

Beyond \carbon{} targets, we consider four expansion scenarios, referred to as models:
\begin{itemize}
    \item[] \textit{Baseline:} Neither \carbon{} nor \hydrogen{} networks are constructed.
    \item[] \textit{\hydrogengrid:} Only the \hydrogen{} network is developed.
    \item[] \textit{\carbongrid:} Only the \carbon{} network is developed.
    \item[] \textit{Hybrid:} Both \carbon{} and \hydrogen{} networks are developed.
\end{itemize}



\section{Results}
\label{sec:results}


Focusing first on the net-zero scenario, we can highlight significant variations in system costs and technology deployment across the models (see Fig.~\ref{fig:cost_bar}).

\begin{figure}[ht!]
    \centering
    \includegraphics[width=\linewidth]{comparison/default/figures/90_nodes/cost_bar.png}
    \caption[short]{Total annual system cost for the European energy system for the different scenarios subdivided into groups of technologies. While the `Baseline' scenario has neither a \hydrogen{} or \carbon{} network, `Hybrid' is allowed to expand both. "Gas Infrastructure" combines gas facilities for power and heat production, "H$_2$ Infrastructure" combines H$_2$ production, transport and re-electrification. "Carbon Capture at Point Sources" combines all technologies with integrated carbon capture.}
    \label{fig:cost_bar}
\end{figure}

System cost differences between the scenarios are driven by renewable energy resources (solar, wind and bioenergy), the \carbon{} and \hydrogen{} networks, as well as carbon capture technologies (DAC and CC at point sources).
In the Baseline scenario, the absence of dedicated carbon and hydrogen transport networks requires the system to deploy carbon production technologies (DAC and CC at point sources) in the regions with carbon sequestration sites, and small amounts of hydrogen production technologies (mainly electrolysis) in each region to meet local hydrogen demand for industry and land transport fuel cells.
For CU and its respective carbon and hydrogen demands, the system chooses the sites where on-site production of carbon, hydrogen and the synthesis process is cheapest. This leads to renewable resources and storage solutions being deployed not necessarily in the regions with the best resources or lowest costs for clean electricity generation, but where they are situated closer to the energy demand locations of electrolysers and DAC.
Thus, total system costs in the baseline scenario are highest, while the scenarios with additional transport options for carbon and hydrogen achieve a more efficient system allocation and lower total system costs.

We can now closer examine the cost reductions and shifts in cost allocation when shifting from the baseline scenario to the other three network scenarios.
The implementation of a \carbon{} transport network (\textit{\carbon{}~Grid}) facilitates efficient point-source carbon capture, substantially reducing the dependency on DAC technologies. While DAC supplies around 50\% of the carbon needed in the Baseline scenario (400 Mt \carbon{}, see Fig.~\ref{fig:balance_captured_carbon}), in the \scCO scenario, DAC is only used to supply around 20\% of the carbon (150 Mt \carbon{}). The remaining carbon is supplied by additional carbon capture technology on all plants burning biomass in industry facilities and in CHPs as well as all industry facilities using gas (see Fig.~\ref{fig:captureshare_line}), because the cost to deploy on-site carbon capture technology is relatively cheaper than DAC. The \carbon{} transport network enables the transport of carbon from the locations of the cheapest captured carbon supply sources to the locations of the carbon demand sinks at CU amd CS sites, reducing the need for expensive DAC. The improved carbon management infrastructure also allows for a more strategic deployment of renewables, particularly in areas with high sequestration potential, leading to cost savings in renewable generation and storage.


\begin{figure}[h]
    \centering
    \includegraphics[width=\linewidth]{comparison/default/figures/90_nodes/captureshare_line.png}
    \caption{Share of facilities with integrated carbon capture for different system setups for the net-zero emission scenario.}
    % A CO2 network unlock BECSS potentials, most biomass emissions are captured at point sources and transported to sequestration sites. cite~\cite{rosaAssessmentCarbonDioxide2021}
    \label{fig:captureshare_line}
\end{figure}


\begin{figure}[h]
    \centering
    \includegraphics[width=\linewidth]{comparison/default/figures/90_nodes/energy_balance_bar_carbon.png}
    \caption{Balance of captured carbon across all system setups. Positive values indicate carbon capture, negative values indicate carbon consumption.}
    % A CO2 network unlock BECSS potentials, most biomass emissions are captured at point sources and transported to sequestration sites. cite~\cite{rosaAssessmentCarbonDioxide2021}
    \label{fig:balance_captured_carbon}
\end{figure}


The \hydrogen{}~Grid scenario underscores the pivotal role of hydrogen infrastructure. The creation of a hydrogen transport network enables the cost-effective delivery of hydrogen from regions with a surplus of cheap renewable energy to high-demand industrial centers. This scenario shows a pronounced decrease in investments in solar and gas  infrastructure. The \hydrogen{} network reduces DAC similarly to the \carbon{} network, but still produces 200 Mt of carbon from DAC. CC technology on process emissions increases from 90\% to 100\%, on biomass CHPs from 60\% to 100\%, and on facilities using gas in industry from 50\% to 75\%. CC on biomass use in industry remains at 50\%, same as the \scbase. This variation highlights how the implementation of \hydrogen{} and \carbon{} networks unlocks specific carbon capture potentials.

While the cost-reduction in the Hybrid scenario compared to the \hydrogen~Grid scenario is marginal, the allocation of costs reveals a propagation of reduced DAC facilities and increased CC at point sources. The Hybrid scenario relies much heavier on investments into the \hydrogen{} network which outweigh investments in the \carbon{} network by a factor of 4.


\begin{figure*}[h]
    \centering
    \begin{subfigure}{\linewidth}
        \centering
        \includegraphics[width=0.95\linewidth]{co2-only/figures/90_nodes/operation_map_carbon.png}
        \caption{\carbon{} Sector.}
        \label{fig:operation_map_carbon}
    \end{subfigure}
    \begin{subfigure}{\linewidth}
        \centering
        \includegraphics[width=0.95\linewidth]{h2-only/figures/90_nodes/operation_map_hydrogen.png}
        \caption{\hydrogen{} Sector.}
        \label{fig:operation_map_hydrogen}
    \end{subfigure}
    \caption{Optimal operation per sector for a net-zero energy system in Europe with average production on the left and average consumption on the right for both, (a) the \carbon sector in the \carbon{}-Grid model and (b) the \hydrogen{} sector in the \hydrogen-Grid model.}
    \label{fig:operation_map}
\end{figure*}


\subsection*{}

Examining the cost-reductions across the different models raises an important question: why is the hydrogen network more cost-effective than its carbon counterpart, and how does it achieve similar cost reductions as the hybrid model? To answer this question, it is crucial to understand the underlying mechanisms and geographical implications of the two transportation systems. Fig.~\ref{fig:operation_map_carbon} maps the average production, consumption and flow in the carbon sector for the \carbongrid model. Fig. \ref{fig:operation_map_hydrogen} shows equivalent quantities in the hydrogen sector for the \hydrogengrid model. In each region, circles divide into contributions from technologies, and their sizes correspond proportionally to production on the left and consumption on the right. The region's colors represent the average price of the considered carrier.
The two fundamental differences between the grid strategies are quickly evident: The \carbongrid model captures large amounts of carbon from bioenergy and industrial processes, especially in central and eastern Europe, at low cost. The carbon grid transports cheap carbon to regions where it is sequestered in offshore caverns or processed into synfuel, methane and methanol together with cheaply available green hydrogen. End-use primarily depends on the quality of renewable resources. In Italy, Greece, North Germany, and Denmark, the transported carbon is predominantly sequestered and reused to some extent. Conversely, in Spain and the United Kingdom, which both have high renewable potentials, the transported carbon is fully utilized, complementing air-captured carbon. Thus, the implementation of a \carbon network allows for connecting regions with abundant, low-price carbon to regions with carbon scarcity where in the absence of a \carbon network DAC facilities had to provide most carbon.
% Sequestered carbon is nearly always transported before

The \hydrogengrid on the other hand, transports cheap hydrogen from regions with abundant renewable resources to regions with high energy demand and abundant emissions. The main transportation routes go from Spain and the United Kingdom to Central Europe. In these regions, industries use the hydrogen to fuel processes and land transport. Additionally, the imported hydrogen allows for the direct reuse of local \carbon emissions in synfuel production, leading to decentralized synfuel production across the continent. Hydrogen produced in Italy contributes to local methane production, which then travels to Central Europe via gas pipelines to supply CHPs and industries."


% Therefore, one can conclude that a \carbon network favors a decentralized CC and a CU centralized at regions with abundant renewable resources. A \hydrogen network favors a centralized electrolysis system at regions with abundant renewable resources and a decentralized CU system.
These findings highlight the pivotal advantage of the \hydrogen network in contrast to the \carbon network. Central Europe with its industrial clusters and dense population is characterized by a scarcity in energy and a surplus and abundance of \carbon. While the \carbon network cost-efficiently resolves the surplus of carbon, the hydrogen resolves both the energy scarcity and the carbon surplus.

\begin{figure*}[h]
    \centering
    \begin{subfigure}{.5\textwidth}
        \centering
        \includegraphics[width=\linewidth]{full/figures/90_nodes/capacity_map_carbon.png}
        \caption{\carbon{} Sector.}
        \label{fig:capacity_map_carbon_co2}
    \end{subfigure}%
    \hfill
    \begin{subfigure}{.5\textwidth}
        \centering
        \includegraphics[width=\linewidth]{full/figures/90_nodes/capacity_map_hydrogen.png}
        \caption{\hydrogen{} Sector.}
        \label{fig:capacity_map_hydrogen_co2}
    \end{subfigure}
    \caption{Optimal production and transport capacities of the carbon and hydrogen sector in a net-zero energy system in Europe with both \carbon{} and \hydrogen{} network expansion (Hybrid).
    % Carbon network looks the same as in~\cite{morbeeOptimisedDeploymentEuropean2012}: two backbones, one in the nothern Europe other in south east.
    }
    \label{fig:capacity_maps}
\end{figure*}

As depicted in Fig.~\ref{fig:capacity_maps}, the hybrid model incorporates the advantages from the a large hydrogen network and complements it with a carbon network. Mirroring the behavior of \hydrogengrid model, it transports hydrogen from Western regions like Spain and the UK to Central Europe, with a centralized electrolysis and decentralized carbon utilization. The carbon network complements this by transporting cheap carbon to sequestration sites in the North Sea, Baltic Sea, Greece, and Italy, minimizing the need for Direct Air Capture (DAC) at these locations. A notable aspect of the hybrid model is the lack of overlap in the routes used by the hydrogen and carbon networks. The advantages of this model include a broader range of technologies contributing to its robustness, reduced reliance on DAC (which may be costlier than anticipated), and less land use for wind and solar due to the decreased necessity for DAC.

% extend on flexibility gains

\subsection*{Net-Negative Scenarios}

\begin{figure}[h]
    \centering
    \includegraphics[width=0.9\linewidth]{difference/comparison/emission-reduction-0.1/figures/90_nodes/cost_bar.png}
    \caption[short]{Net change in investments when tightening the \carbon{} emission target from net-zero to net-negative 10\% of 1990s emissions.}
    \label{fig:net-negative_cost_bar}
\end{figure}

We now examine the case of moving from a net zero emissions target to a net emissions reduction target. Fig.~\ref{fig:net-negative_cost_bar} illustrates the shift in investments for the \hydrogengrid and the Hybrid model between the net-zero and the net-negative scenario.




The \hydrogengrid model emphasizes a substantial investment in hydrogen infrastructure, signifying a strategic bet on hydrogen's role in decarbonization. There is also a significant allocation for methanation and the expansion of electricity grids, underscoring the integration of hydrogen production with renewable energy sources.


% The Hybrid model, in contrast, adopts a more balanced investment approach, with notable allocations not just to hydrogen infrastructure but also to wind, solar, and CO2 infrastructure, as well as direct air capture (DAC) technology. This suggests a more comprehensive strategy that leverages a variety of technologies and energy sources to achieve emission reductions. The distributed investment in the Hybrid model indicates a systemic approach to decarbonization, integrating renewable generation, energy storage, carbon capture, and hydrogen production to address both energy and carbon management challenges.



\section{Limitations}
\label{sec:limitations}

Despite the detailed model representation, there are some limitations to the validity of the results. The model itself is based on a linearized optimization with perfect foresight for the entire modeling year. In reality, long-term energy demand and renewable supply can only be roughly estimated, while short-term predictions still entail some uncertainty. The model's perfect foresight may lead to non-reproducible behavior, such as precisely aligning energy storage with future energy shortages at a specific point in time.

The technology costs used in the model rely on cost projections that incorporate reductions based on learning rates. These learning rates are derived from historical data, which may not necessarily be indicative of future trends. The model does not account for uncertainties arising from disruptive market behavior, such as the gas price peak that occurred in 2022.

% TODO: which fixed rate? you mean that all emissions are captured by exogenous assumption or that we assume 90% of carbon can be captured?
% Furthermore, the model assumes a fixed rate for industrial process emissions, which cannot be altered through investments. This simplification may not accurately represent real-world scenarios, where many industries are considering adopting low-carbon processes and technologies. The transport of fuel (FT, gas, oil) between the regions is not unlimited which may overestimate the flexibility provided by these commodities.

Lastly, the model's spatial and temporal resolution is insufficient to capture all relevant dynamics. Variations in renewable energy supply and energy demand below the applied time resolution, as well as more detailed interregional energy transport constraints, are currently not considered in the optimization. Transitioning to an hourly representation and a higher spatial resolution would increase the model's computational time and complexity, but would also enhance its robustness and validity.

\section{Conclusion}
\label{sec:conclusion}

In this study, we investigated the impact of a \carbon{} transport system with different carbon sequestration potentials on the optimal configuration of technologies in a fully sector-coupled energy system in Europe, focusing on the deployment of carbon capture and Utilization (CCU) and Carbon Capture and Storage (CCS) technologies. Our results indicate that increasing carbon sequestration potential leads to a decrease in total system costs, with a more pronounced cost reduction observed when a carbon transport network is present. Higher sequestration rates result in a shift from CCU and hydrogen infrastructure towards CCS and unabated fossil fuel usage compensated by carbon dioxide removal elsewhere in the system, particularly based on direct air and biomass carbon capture with sequestration.

Furthermore, we observed that with a \carbon{} transport network, the system tends to build out more carbon capture at point sources. With an increased sequestration potential, the share of technologies equipped with integrated carbon capture rises, particularly for Steam Methane Reforming (SMR) and Gas Combined Heat and Power (CHP) plants. Independent of a setup with or without \carbon{} transport, the spatial distribution of carbon capture technologies becomes more centralized, with installations concentrated near sequestration sites.

Despite limitations of the study considering fixed model assumptions and cost uncertainties, our findings contribute valuable insights into the role of a carbon transport system and carbon sequestration potential in the transition towards a climate-neutral and net-negative energy system. As Europe strives to achieve its 2050 climate neutrality target, understanding the interplay between carbon sequestration, CCU, and CCS technologies will be crucial for the design of effective policies and strategies.


\printbibliography

\appendix


% \begin{figure*}
%     \centering
%     \includegraphics[width=\linewidth]{net-negative-0.1/full/figures/90_nodes/sankey_diagramm.png}
%     \caption{Sankey diagram of the optimal operation for a net-negative 10\% scenario.}
%     \label{fig:sankey_diagramm}
% \end{figure*}

% \begin{figure}
%     \centering
%     \includegraphics*[width=0.8\linewidth]{comparison/emission-reduction/figures/90_nodes/objective_heatmap.png}
%     \caption{Total annual system cost for the different scenarios and \carbon{} targets. While the `Baseline' scenario has neither a \hydrogen{} or \carbon{} network, `Hybrid' is allowed to expand both.}
%     \label{fig:objective_heatmap}
% \end{figure}


% \begin{figure}[h]
%     \centering
%     \includegraphics[width=\linewidth]{comparison/emission-reduction-full/figures/90_nodes/transmission_cost_bar.png}
%     \caption{Annual transmission system cost as a function of the net carbon removal scenarios considered in the study.}
%     \label{fig:transmission_cost_bar}
% \end{figure}


\end{document}

\documentclass[twocolumn]{article}
\usepackage{amsmath,amssymb,amsfonts}
\usepackage{cuted}  % Add the cuted package
\usepackage{caption}
\captionsetup{font=small}
\usepackage{algorithmic}
\usepackage{graphicx}
\usepackage{textcomp}
\usepackage{xcolor}
\usepackage{tabularx, multirow}
\usepackage{fancyhdr,lipsum}
\usepackage{subcaption}

\usepackage[%
backend=biber,bibencoding=utf8, %instead of bibtex
language=auto,
style=ieee,
sorting=none, % nyt for name, year, title
maxbibnames=10, % default: 3, et al.
%backref=true,%
natbib=true % natbib compatibility mode (\citep and~\citet still work)
]{biblatex}
\bibliography{../../../references.bib}

%define approx proportional
\def\app#1#2{%
  \mathrel{%
    \setbox0=\hbox{$#1\sim$}%
    \setbox2=\hbox{%
      \rlap{\hbox{$#1\propto$}}%
      \lower1.1\ht0\box0%
    }%
    \raise0.25\ht2\box2%
  }%
}
\def\approxprop{\mathpalette\app\relax}

% make abbreviation for co2
\newcommand{\carbon}{CO$_2$}
\newcommand{\hydrogen}{H$_2$}
\newcommand{\carbongrid}{\carbon{}--Grid}
\newcommand{\hydrogengrid}{\hydrogen{}--Grid}
\newcommand{\modBase}{Baseline model}
\newcommand{\modCO}{CO$_2$-Grid model}
\newcommand{\modH}{H$_2$-Grid model}
\newcommand{\modHybrid}{Hybrid model}

\graphicspath{
    % {paper-figures}
    {figures}
    {../../../../results/}
    {results/}
}

\begin{document}


% FORMATTING: https://www.nature.com/nenergy/submission-guidelines/aip-and-formatting


\title{From Net-Zero to Net-Negative: Assessing \hydrogen{} and \carbon{} Network Strategies in Europe}
% \title{Synthetic fuels in Europe: Transport Hydrogen to Carbon, or Carbon to Hydrogen?}

\author{
    Fabian Hofmann, Christoph Tries, Fabian Neumann, Lisa Zeyen, Tom Brown \\
    \textit{Institute of Energy Technology} \\
    \textit{Technical University of Berlin}\\
    Berlin, Germany \\
    m.hofmann@tu-berlin.de
}


\maketitle

\begin{abstract}
    In future climate-neutral scenarios, the transport of hydrogen and carbon dioxide can play competing roles.
    Hydrogen transport allows energy-intensive regions to import low-cost hydrogen. Carbon transport allows regions with carbon sinks such as carbon sequestration and synthetic fuel production to import low-cost carbon from distributed carbon capture sources.
    Whether carbon and hydrogen transport networks are complementing or substituting each other remains unclear. This is especially relevant for the production of carbonaceous fuels such as Fischer-Tropsch fuels, methanol or synthetic methane, which are made from both hydrogen and carbon dioxide, and in most locations require at least one of these feedstocks to be imported.
    To address this gap, our study employs optimization techniques to design the first cost-optimal European energy system that fully incorporates transport, storage and sequestration for carbon dioxide, transport and storage for hydrogen, and renewable energy sources with a high spatial and temporal resolution considering all energy intensive sectors.
    Our findings reveal that when comparing either-or-models, a system with an hydrogen network provides lower system costs than a system with a carbon network. The former enables the transportation of hydrogen from centralized production sites with low-cost renewable energy in the Iberian Peninsula and the British Isles to serve distributed hydrogen demands across the European continent. The latter facilitates carbon capture at lower costs from distributed point sources and reduces the need for higher cost direct air capture. However, the system with only a carbon network needs to meet distributed hydrogen demand across the continent by locally deploying electrolysis regardless of renewable energy costs.
    In a hybrid model with both types of networks, the carbon network is a cost-effective complement to the hydrogen network facilitating point-source carbon capture at low costs and transport to sequestration or carbon utilization sites. In all models, additional demand for synthetic fuels is met at sites with favorable renewable energy resources that enable both electrolysis and direct air capture at low costs.
    We show that theses findings holds true against the backdrop of a net-zero energy system as well as a scenario with net-negative emissions.
    Overall, our work demonstrates the cost-effectiveness of a multi-grid system that includes both hydrogen and carbon transport networks and power grid expansion to achieve climate neutrality in Europe.
    The paper highlights the benefits of combining hydrogen and \carbon{} networks to achieve climate neutrality.
\end{abstract}


\section{Introduction}

The transition to a carbon-neutral European economy is a pressing challenge that demands coordinated action across various energy sectors. While management of both \carbon{} and \hydrogen{} is considered a critical component of this transition, a gap exists in understanding how new hydrogen infrastructures effectively interact with comprehensive carbon management technologies, including carbon capture, transport, use, and storage. Hydrogen offers an efficient way to transport and store energy. Carbon, on the other hand, can be effectively captured from industrial processes and the combustion of biomass, fossil fuels, and synthetic fuels through carbon capture (CC) techniques, or harvested from the atmosphere using direct air capture (DAC). Additionally, it can be stored in geological formations, a process known as carbon sequestration (CS). In combination, \carbon{} and \hydrogen{} networks build the basis for climate-neutral fuels needed for aviation, shipping and industrial feedstocks, and are central to carbon utilization (CU) strategies.

Recently, policymakers and industry in Europe have started developing carbon management strategies~\cite{GermanyDevelopingStrategy2023,CarbonManagementStrategie}, planning infrastructure components~\cite{CONetz}, and committing on the first carbon utilization projects~\cite{EFuelsPilotPlant2022,OrstedAssumesFull,GROUNDBREAKINGEFUELPRODUCTION,DLREfuelsDLR}. With the European Union's goal of achieving climate neutrality by 2050, a wide range of programs, funding models and initiatives have been established in this regard~\cite{eu2023netzero,europeangreendeal,europeaninnovationfund}. Initiatives like the European Hydrogen Backbone~\cite{gasforclimateEuropeanHydrogenBackbone2022} or the Hydrogen Infrastructure Map~\cite{H2InfrastructureMap} showcase the potential of hydrogen as a fuel and energy carrier, and some gas pipelines are already repurposed to transport hydrogen~\cite{RohrFreiFuer}. At the same time, business models from companies like Tree Energy Solutions~\cite{TESHydrogenLife2023}, Carbfix~\cite{WeTurnCO2}, and Equinor~\cite{adomaitisEquinorRWEBuild2023} advertise carbon management hubs that provide green hydrogen, methane, or synfuels on the one hand and offer \carbon{} offtake on the other hand. For Europe, the Capture Map~\cite{ToolsGreenTransition} estimates a potential of 1.7~Gt of carbon capture from point sources per year, which represent roughly 50\% of all emissions. In combination with large sequestration potentials as stated in~\cite{weiProposedGlobalLayout2021}, this highlights the vast potential for decarbonization. The Northern Lights project in Norway~\cite{NorthernLightsWhat} is planning with a transport and sequestration capacity of 1.5 Mt \carbon{} per year  to be operative in 2024, expanding to a targeted scale of 5 Mt per year of sequestration by 2030.
To this end, the Clean Air Task Force underlines the importance of a carbon transport system in Europe to facilitate the carbon economy~\cite{lockwoodEuropeanStrategyCarbon}.


However, up to this point, it remains unclear how the two transport systems of hydrogen and carbon may complement or replace each other, particularly when it comes to their transport systems. A hydrogen network facilitates the transport of energy from renewable sources to regions with geographically fixed hydrogen demand, such as for steelmaking, and also enables CU at the site of point-source capture. On the other hand, the carbon transport system allows for capturing and transporting carbon from areas with high emissions to regions with significant sequestration potential and abundant renewable sources, thereby enabling cost-effective CU.
In the literature, the two network approaches and underlying technologies have been discussed in a number of publications, all of which, however, dealt with the isolated effects~\cite{bakkenLinearModelsOptimization2008,morbeeOptimisedDeploymentEuropean2012,stewartFeasibilityEuropeanwideIntegrated2014,oeiModelingCarbonCapture2014,elahiMultiperiodLeastCost2014,burandtDecarbonizingChinaEnergy2019,middletonSimCCSOpensourceTool2020,bjerketvedtOptimalDesignCost2020,weiProposedGlobalLayout2021,damoreOptimalDesignEuropean2021,becattiniCarbonDioxideCapture2022}. Such techno-economic models, in contrast for example to Integrated Assessment Models, can account for the spatial distribution of carbon sources and sinks which are crucial provide a holistic view of the energy system and its technological interactions. The work in~\cite{neumannBenefitsHydrogenNetwork2022} examines the effect of a hydrogen network in Europe... (extend on hydrogen literature)
The publication by Morbee et al.~\cite{morbeeOptimisedDeploymentEuropean2012} optimizes the topology and capacity of a \carbon{} network in Europe, but only considers the power sector without co-optimizing renewable deployment. This limited sectoral scope cannot capture important dynamics of carbon management, since it neglects the sectors that will need to handle most \carbon{} in the future.
Another comprehensive example is found in~\cite{becattiniCarbonDioxideCapture2022}, which presents a mixed-integer model to optimize the time-evolution of a \carbon{} transport system in Switzerland, connecting to a remote sequestration site in Norway. Hoewever, this limited spatial scope fails to consider other sequestration sites and co-benefits from connecting the \carbon{} network to neighboring countries. To this end, it is often argued that \carbon{} pipelines are a mature technology with an expected high learning rate, given the wide-spread installations in US and Canada for enhanced oil recovery~\cite{righettiSitingCarbonDioxide2017,friedmannNETZEROGEOSPHERICRETURN}.
% However, the models are often limited with regard to both geographical scope and detail. While representing a single country with spatial resolution may neglect synergies of international cooperation, a coarse grained representation of multiple countries may neglect important geographical properties.

To our knowledge, no study has yet considered the co-optimization and comprehensive assessment of both \carbon{} and \hydrogen{} networks in a fully sector-coupled energy system. However, we argue that such an assessment is strongly needed to avoid suboptimal investments and to identify synergies between hydrogen and carbon management technologies. In this paper, we present a detailed study of the European energy system for 2050, which includes high geographical resolution and a comprehensive representation of carbon management technologies. The study is conducted using the PyPSA-Eur energy system model and encompasses all relevant energy sectors. We explore competition between \carbon{} and \hydrogen{} networks by adding them separately. Our evaluation focuses on the transport of \carbon{} and \hydrogen{} through their respective networks on the European continent. We also analyze how an energy system with limited annual sequestration potential prioritizes decarbonization and fuel switching in various sectors, and how the construction of carbon networks varies based on different levels of available sequestration potential.


\section{Methodology}
\label{sec:methodology}

The study is conducted on the basis of the open-source, capacity-expansion model PyPSA-Eur~\cite{horschPyPSAEurOpenOptimisation2018,brownSynergiesSectorCoupling2018,PyPSAEurSecSectorCoupledOpen2023}.
The model optimizes the design and operation of the European energy system, encompassing the power, heat, industry, waste, agriculture, and transport sectors (including international aviation and shipping).

\begin{figure}
    \includegraphics[width=\linewidth]{baseline/figures/90_nodes/total-demand-bar.png}
    \caption{Exogenous demand assumptions. The figure shows the total annual energy demands for each energy carrier which drive all model activities. Endogenous processes may lead to higher total production volumes of some carriers, e.g., demand in methanol requires more hydrogen and carbon as secondary (energy) inputs, which in turn need additional electricity to run electrolysis and carbon capture processes.}
    % TODO: adjust labels to show that these are exogenous assumptions
    \label{fig:total-demand-bar}
\end{figure}
%
In our configuration, the model's time horizon spans one year with a temporal resolution of 3 hours and a spatial resolution of 90 regions. Each of the regions consists of a complex subsystem with technologies for supplying, converting, storing and transporting energy. Exogenous assumptions on energy demand and non-abatable emissions are taken from various sources~\cite{piamanzGeoreferencedIndustrialSites2018,muehlenpfordtTimeSeries2019,mantzosJRCIDEES20152018,NationalEmissionsReported2023,EurostatCompleteEnergyBalance,uwekrienDemandlib2023}. The energy demand for electricity, transport, biomass, heat and gas are defined per region and time-step. Land transport demand is exogenously divided between electric vehicles (85\%) and fuel cell vehicles (15\%), the latter representing demand for heavy duty land transport. Demands for kerosene for aviation, methanol for shipping, and naphtha for industry are aggregated in the system scope and kept constant throughout all time-steps. Heat demand is regionally subdivided into shares of urban, rural and industrial sites. We show the sum of all energy demands in Fig.~\ref{fig:total-demand-bar}. The system emits 633~Mt \carbon{} per year from industry, aviation, shipping and agriculture, 153~Mt of which are fossil-based process emissions. Industrial energy demand and excess heat potentials are calculated per node on the basis of~\cite{hotmaps_industrial_db}.
% use https://pypsa-eur.readthedocs.io/en/latest/licenses.html
%
Endogenous model results include the expansion of renewable energy sources, storage technologies, and transmission capacities.
The model considers the various energy carriers like electricity, hydrogen, methan, methanol, liquid hydrocarbons and biomass, together with conversions technologies to convert these into each other.
Carbon-neutral electricity is provided by wind, solar, biomass, hydro and nuclear power plants. Hydro and nuclear plant capacities cannot be extended. Weather-dependent power potentials for solar, wind and hydro are calculated from reanalysis and satellite data sets~\cite{hersbachERA5GlobalReanalysis2020,pfeifrothSurfaceRadiationData2017}  per region and time-stamp, using the open-source tool Atlite~\cite{hofmannAtliteLightweightPython2021}.
Solar and wind power can be expanded in alignment with eligible land-use restrictions calculated on the basis of~\cite{eeaCorineLandCover2012,eeaNatura2000Data2016}. We restrict the electric transmission system's expansion to 20\% of its current capacity, acknowledging the challenges in inaugurating new transmission projects.
For the use of biomass we consider only residual biomass products and no energy crops. We limit regional biomass use to the medium-level potentials from the ENPRESO database~\cite{enspreso_database,instituteforenergyandtransportjointresearchcentreJRCEUTIMESModelBioenergy2015}. Inter-regional biomass transport is permitted with transport costs considered.

In our model, hydrogen can be produced from electrolysis and steam methane reforming (SMR). Geological distribution and potentials for \hydrogen{} storage in salt caverns are derived from~\cite{caglayanTechnicalPotentialSalt2020}. Re-electrification of hydrogen is possible via fuel cells. If enabled, hydrogen can be transported via pipelines between regions which can be expanded without limit, considering costs for pipeline segments and compressors. Pipeline flows are modelled using net transfer capacities and without flow dynamics, pressure valves, or energy demand for compression. No retrofitting of gas pipelines is considered, potentially overestimating hydrogen network cost.

The topology of the \carbon{}, \hydrogen{} and gas network is identical to the topology of the electricity network, connecting all neighboring regions. Transport of biomass is considered as a variable cost component, and transport of liquid fuels like oil, methanol and Fischer-Tropsch (FT) fuels is ``copper-plated'', since transport costs per unit of energy are negligible due to their high energy density. Throughout this study, we focus on \carbon{} and \hydrogen{} networks because of their high relevance for infrastructure investments and thus public policy decisions. The electricity network and gas network are both included in the model with full geographical detail, but not further analyzed. Electricity networks are already in place, and we restrict further extension due to concerns of public acceptance. Gas networks also already exist, and will most likely experience decreased use in the future, removing any bottlenecks or constraints on the optimal system buildout or operation.

Our model features three drop-in fuel production technologies for CU: methanation, methanolization, and Fischer-Tropsch synthesis. The processed fuels are not limited in their total quantity of use or production. Methane is transported through the gas network, while methanol and FT fuel can be transferred inter-regionally without additional costs, since dense fuels have negligible transport costs per unit of energy. Synthetic methane substitutes natural gas or biogas, serving combined heat and power plants, residential heating gas boilers, or industrial process heat. Synthetic methanol decarbonizes marine industry fuel demands, and FT fuels replace fossil oil for naphtha production, aviation kerosene, or agricultural machinery oil.

To supply carbon needed for CS and CU, the system can choose to deploy carbon capture technology at various point sources (see below), or through DAC facilities. The concept of a ``merit order for captured carbon`` plays a pivotal role in optimizing and prioritizing the deployment of carbon capture technologies based on their economic feasibility. This merit order ranks all CC technologies according to their relative costs to capture a ton of carbon (see Fig.~X). At the lower end of the cost spectrum, CC technologies applied to process emissions, such as those from cement, offer a cost-effective starting point. Following this, biomass combined heat and power (CHP) systems provide a dual benefit of energy production and carbon capture. Moving up the scale, gas used in industrial applications and biomass employed for industrial processes represent more costly yet viable options to capture carbon. DAC, an emerging technology capable of extracting \carbon{} directly from the atmosphere, stands at a higher cost level due to its current high capital costs and energy demand. Finally, based on our model results, biogas-to-gas upgrading, a process that refines biogas to natural gas quality, incurs the highest costs in the merit order per marginal ton of captured \carbon{}. Biogas input is a high-cost fuel which based on the endogenous modeling decisions is not used in large quantities. To the extent that biogas-to-gas facilities are used by the model to supply additional (carbon-neutral) gas as fuel, adding CC infrastructure incurs low costs (and thus ranges on the left-hand side of the merit order curve). However, the price of capturing additional \carbon{} from biogas-to-gas upgrading is high because a substantial amount of the biogas fuel costs factors into the marginal cost of captured \carbon{}. This merit order framework is crucial for strategically deploying necessary carbon capture solutions while balancing economic considerations. If we consider the spatial aspect of distributed carbon capture potentials, the \carbon{} network can exploit comparative cost advantages between ``CO2 bidding zones'' to utilize the lowest-cost CC facilities across the continent. We assume a capture rate of 90\% for CC on process emissions, SMR, biogas-to-gas, as well as gas and biomass used in industry, and 95\% for CC on biomass and gas CHPs.

To store carbon, we differentiate between short-term storage in steel tanks and long-term, irreversible sequestration in underground sequestration sites such as porous rock formations or depleted gas reservoirs. Costs for both options are included in the model. For carbon sequestration, we only consider offshore sites as potential sinks. We make this choice because offshore sequestration sites typically have larger capacity compared to onshore sequestration sites in saline aquifers and due to concerns over public safety for infrastructure near populated areas. Our estimates for carbon sequestration potentials are conservative, limiting total sequestration to 25~Mt per site and calculating annual storage availability over 25 years. Furthermore, we decide to cap the total sequestered \carbon{} to 200~Mt per year for our Net-Zero scenario. This is enough to offset the hardest-to-abate fossil process emissions and some limited slack of fossil fuels to be used where they can be most valuable. This is in order to avoid offsetting fossil emissions with carbon removal if they can technologically be avoided in the first place. All technology cost assumptions are taken for the year 2040 and sourced from an open-source database~\cite{lisazeyenPyPSATechnologydataTechnology2023}.


Addressing climate targets, we define two \carbon{} target scenarios:
\begin{itemize}
    \item[] \textit{Net--Zero}: aligning with the EU's 2050 emission targets. Also denoted as NZ.
    \item[] \textit{Net--Negative}: 10\% net-negative emissions relative to 1990 levels, equaling 460~Mt \carbon{} annually. Also denoted as NN.
\end{itemize}
Unless otherwise mentioned, our reference is the Net-Zero scenario. For the Net-Negative scenario, the cap of total carbon sequestration is adjusted from 200~Mt to 660~Mt per year accordingly.

Beyond \carbon{} targets, we consider four expansion scenarios, referred to as models:
\begin{itemize}
    \item[] \textit{Baseline:} Neither \carbon{} nor \hydrogen{} networks are constructed.
    \item[] \textit{\carbongrid:} Only the \carbon{} network is developed.
    \item[] \textit{\hydrogengrid:} Only the \hydrogen{} network is developed.
    \item[] \textit{Hybrid:} Both \carbon{} and \hydrogen{} networks are developed.
\end{itemize}



\section{Results}
\label{sec:results}


Focusing first on the Net-Zero scenario, we can highlight significant variations in system costs and technology deployment across the models (see Fig.~\ref{fig:cost_bar}).

\begin{figure}[ht!]
    \centering
    \includegraphics[width=\linewidth]{comparison/default/figures/90_nodes/cost_bar.png}
    \caption[short]{Total annual system cost for the European energy system for the different models subdivided into groups of technologies. While the Baseline model has neither a \carbon{} nor a \hydrogen{} network, the Hybrid model is allowed to expand both. ``Gas Infrastructure'' combines gas facilities for power and heat production, ``\carbon{} Infrastrucure'' combines transport, storage and sequestration, and ``H$_2$ Infrastructure'' combines transport and storage. ``Carbon Capture at Point Sources'' combines all technologies with integrated carbon capture, including the cost of the main facility (e.g., CHP unit) and the carbon capture application.}
    \label{fig:cost_bar}
\end{figure}

System cost differences between the models are driven by renewable energy resources (solar, wind, and bioenergy), the \carbon{} and \hydrogen{} networks, as well as carbon capture technologies (DAC and CC at point sources).
In the Baseline model, the absence of dedicated carbon and hydrogen transport networks requires the system to deploy carbon production technologies (DAC and CC at point sources) in the regions with carbon sequestration sites, and distributed amounts of hydrogen production technologies (mainly electrolysis) in each region to meet local hydrogen demand for industry and fuel cells in heavy-duty land transport vehicles.
For CU and its respective carbon and hydrogen demands, the system chooses the sites where on-site production of carbon, hydrogen as well as the inputs and by-product outputs (e.g., waste heat) of the synthesis process have the lowest cost (or highest value). This leads to renewable resources and storage solutions being deployed not necessarily in the regions with the best resources or lowest costs for clean electricity generation, but where they are situated closer to the endogenously defined locations of energy demand for electrolysers and DAC.
Thus, total annual system costs in the Baseline model are highest, totalling \label{}803 billion euros (see Fig.~\ref{fig:objective_heatmap}). The models with additional transport options for carbon and hydrogen achieve a more efficient system allocation and lower total system costs, with around \label{}2\% cost reductions (\label{}781 billion euros) for the \carbon{} network, and around \label{}4\% cost reductions for the \hydrogen{} network and the hybrid scenario (\label{}765 and \label{}764 billion euros, respectively).

We can now closer examine the cost reductions and shifts in cost allocation when shifting from the Baseline model to the other three network models.
The implementation of a \carbon{} transport network (\textit{\carbon{}~Grid}) facilitates carbon capture from the point sources with the lowest costs across regions, substantially reducing the dependency on DAC. While DAC supplies around 50\% of the carbon needed in the Baseline model (400 Mt \carbon{}, see Fig.~\ref{fig:balance_captured_carbon}), in the \modCO{}, DAC is only used to supply around 20\% of the carbon handled by the system (150 Mt \carbon{}). The remaining carbon is supplied by additional carbon capture technology on all plants burning biomass in industry facilities and in CHPs as well as all industry facilities using gas (see Fig.~\ref{fig:captureshare_line}), because the cost to deploy on-site carbon capture technology has relatively lower costs than DAC. The \carbon{} transport network enables the transport of carbon from the locations of the lowest-cost captured carbon sources to the locations of the carbon sinks at CU and CS sites, reducing the need for higher cost DAC. The option to transport carbon also allows for the deployment of renewables in areas that combine high quality renewable resources and high sequestration potentials, leading to system-wide cost savings in renewable generation and storage.


\begin{figure}[h]
    \centering
    \includegraphics[width=\linewidth]{comparison/default/figures/90_nodes/captureshare_line.png}
    \caption{Share of facilities with integrated carbon capture for different system setups for the Net-Zero scenario.}
    % A CO2 network unlock BECSS potentials, most biomass emissions are captured at point sources and transported to sequestration sites. cite~\cite{rosaAssessmentCarbonDioxide2021}
    \label{fig:captureshare_line}
\end{figure}%

\begin{figure}[ht!]
    \centering
    \includegraphics[width=\linewidth]{comparison/default/figures/90_nodes/energy_balance_bar_carbon.png}
    \caption{Balance of captured carbon across all system setups. Positive values indicate carbon capture, negative values indicate carbon consumption.}
    % A CO2 network unlock BECSS potentials, most biomass emissions are captured at point sources and transported to sequestration sites. cite~\cite{rosaAssessmentCarbonDioxide2021}
    \label{fig:balance_captured_carbon}
\end{figure}


The \hydrogen{}-Grid model underscores the beneficial role of hydrogen infrastructure. The creation of a hydrogen transport network enables the cost-effective delivery of hydrogen from regions with a surplus of low-cost renewable energy to high-demand industrial centers. This model shows a pronounced decrease in investments in DAC, and accordingly also solar generation which provides most of the electricity to run DAC facilities. The reduction in DAC is similar to the \carbon{} network, but the \hydrogen{} network still produces 200 Mt of carbon from DAC. CC technology on process emissions increases from 90\% to 100\%, on biomass CHPs from 60\% to 100\%, and on facilities using gas in industry from 50\% to 75\%. CC on biomass use in industry remains at 50\%, same as the \modCO{} (see Fig.~\ref{fig:captureshare_line}). This is because the \hydrogen{} network can supply low-cost \hydrogen{} to \carbon{} point sources, where without the \hydrogen{} network, fuel synthethis with locally captured \carbon{} and locally sourced electrolytic hydrogen would have been more expensive than synthesizing fuel with \carbon{} from DAC and hydrogen from electrolysis at the best renewable resource sites. This highlights how the implementation of \carbon{} and \hydrogen{} networks unlocks carbon capture potentials at distributed point sources across the continent.

While the cost reduction in the Hybrid model compared to the \hydrogen-Grid model is marginal, the allocation of costs reveals a propagation of reduced DAC facilities and increased CC at point sources. In other words, the \carbon{} network allows to tap into far away biomass potentials to capture carbon indirectly from the air, which provides marginally lower costs than DAC, even considering carbon transport costs. The Hybrid model relies slightly less on investments into the \hydrogen{} network than the \modH, but adds some investments into the \carbon{} network. Still, investments into the \hydrogen{} network outweigh investments into the \carbon{} network by a factor of four.

\subsection*{Superiority of \hydrogen{} network}

% \begin{figure*}[ht!]
%     \centering
%     \begin{subfigure}{0.9\linewidth}
%         \centering
%         \includegraphics[width=\linewidth]{co2-only/figures/90_nodes/operation_map_carbon.png}
%         \caption{\carbon{} Sector.}
%         \label{fig:operation_map_carbon}
%     \end{subfigure}
%     \begin{subfigure}{0.9\linewidth}
%         \centering
%         \includegraphics[width=\linewidth]{h2-only/figures/90_nodes/operation_map_hydrogen.png}
%         \caption{\hydrogen{} Sector.}
%         \label{fig:operation_map_hydrogen}
%     \end{subfigure}
%     \caption{Optimal operation per sector for a net-zero energy system in Europe with average production on the left and average consumption on the right for both, (a) the \carbon{} sector in the \carbon{}-Grid model and (b) the \hydrogen{} sector in the \hydrogen-Grid model.}
%     \label{fig:operation_map}
% \end{figure*}

\begin{figure*}[ht!]
    \centering
    \includegraphics[width=\linewidth]{comparison/single-technologies/figures/90_nodes/operation_map_dedicated.png}
    \caption{Optimal operation per sector for a net-zero energy system in Europe with average production on the left and average consumption on the right for both, (a) the \carbon{} sector in the \carbon{}-Grid model and (b) the \hydrogen{} sector in the \hydrogen-Grid model.}
    \label{fig:operation_map}
\end{figure*}


Examining the cost-reductions across the different models raises an important question: why is the hydrogen network more cost-effective than its carbon counterpart, and how does it achieve similar cost reductions as the hybrid model? An initial hypothesis may be that the cost assumptions for \carbon{}
 and \hydrogen{} networks play a role. However, the amount of system costs spent on either network is substantially lower than the cost difference between the models (20 billion euros less total system costs in the \hydrogengrid{} model, while it spends 10 billion euros on the \hydrogen{} grid and the \carbongrid{} model spends 8 billion euros on the \carbon{} grid). Furthermore, we have conducted a sensitivity analysis with cost assumptions for the \carbon{} network reduced to 50\% of the default assumptions, and the results show that the \hydrogen{} network still outperforms the \carbon{} network in terms of total system costs.
To find the reason for why the \hydrogen{} network achieves lower system costs than the \carbon{} network, it is crucial to understand the underlying mechanisms and geographical implications of the two transportation systems. Fig.~\ref{fig:co2-only-operation_map_carbon} and \ref{fig:co2-only-operation_map_hydrogen} map the average production, consumption and flow in the carbon and hydrogen sectors for the \carbongrid{} model. Fig. \ref{fig:h2-only-operation_map_carbon} and \ref{fig:h2-only-operation_map_hydrogen} show equivalent quantities for the \hydrogengrid{} model. In each region, circles represent the production (top half-circle) and consumption (bottom half-circle) of the respective energy carrier, divided into shares of each technology and the area sizes correspond to production and consumption volumes. Differences between production and consumption in a given node correspond to import or export volumes through the connected networks, represented as lines with their size and arrows indicating volumes and direction of average flows. Each region's color represents the demand-weighted average price of the considered carrier in that region.

There are four notable dynamics or differences between the two grid strategies: 
First, the production of FT fuels always moves to where low-cost carbon and low-cost hydrogen can be brought together. In the \carbongrid{} model, carbon is moved from Western Europe to the Iberian Peninsula, the British Isles and to Denmark to produce FT fuels centrally in few regions. In the \hydrogengrid{} model, hydrogen is transported in the opposite direction to produce FT fuels decentrally across all regions in Central Europe with locally captured \carbon{} from point sources.
Second, the \carbongrid{} model captures more low-cost carbon from biomass, especially in Central and Eastern Europe, and less high-cost carbon from DAC in Greece and Southern Italy. Instead, the carbon grid transports carbon from Central and Eastern Europe to Greece and Italy, where it is sequestered, or feeds into CU together with locally produced, low-cost electrolytic hydrogen. The \hydrogengrid{} model directly operates electrolyzers and DAC in Greece and Southern Italy, with electricity from low-cost renewables, and thus avoids significant network buildouts in this region.
Third, sequestration in the \hydrogengrid{} model occurs more decentralized, spread out across several regions with access to offshore sequestration sites, making use of point-source CC potentials in each of these regions. In the \carbongrid{} model, sequestration is clustered at fewer sequestration sinks, each collecting low-cost \carbon{} from neighboring regions. However, there is no centralization at only one or two sites, since all the sequestration sites offer the same costs and the system minimizes system-wide carbon transport distances to about eight to ten sequestration sites.
And fourth, how the three types of CU are spread across the continent is driven by the costs of carbon and hydrogen (the latter mostly driven by the costs of renewable electricity) as well as by renewables capacity factors: Clearly, all carbonaceous fuel production requires sites where both carbon and hydrogen are available, and they moreover prefer sites with low carbon and hydrogen prices. Based on our model assumptions, FT fuel synthesis has very limited flexibility and needs to be run alost at baseload (ramp-down only possible to 90\% of maximum load). Thus, for FT fuel synthesis sites in Northern Europe with more electricity from wind power are preferred to sites in Southern Europe, powered mostly by solar. Methanathion and Methanolisation both have higher flexibility, being able to ramp down to 50\% load during the night, for example. Thus, these fuel production sites prefer the best renewable resource sites in Southern Europe. And finally, as methanation produces synthetic methane that needs to be fed into the gas grid, methanation prefers those sites with good interconnection to the existing gas network. Since Italy possesses a large gas network with easy interconnection to Central Europe, methanation prefers sites in Southern Italy (and Greece, which would be connected to Southern Italy) over the Iberian Peninsula, where Methanolisation is more prevalent.

In summary, the implementation of a \carbon{} network allows for connecting regions with abundant, low-price carbon (from point sources) to regions that can sequester the carbon or that can produce low-cost hydrogen to make CU products (FT fuel, methanol, synthetic methane). DAC and electrolysis are deployed in the regions with the best renewable resources to meet carbonaceous fuel demands.
In the absence of a \carbon{} network and with a \hydrogengrid{} on the other hand, regions with access to offshore sequestration sites capture all carbon from low-cost point sources and then sequester it. The system deploys additional DAC facilities at the locations with sequestration access and low-cost renewables to capture the remaining necessary carbon for sequestration. Carbonaceous fuel demand is met at the locations which offer the best mix of low-cost carbon from point sources and low-cost hydrogen from renewables, with some additional electrolysis and DAC required to fully meet the demand.
% Sequestered carbon is nearly always transported before

% Therefore, one can conclude that a \carbon{} network favors a decentralized CC and a CU centralized at regions with abundant renewable resources. A \hydrogen{} network favors a centralized electrolysis system at regions with abundant renewable resources and a decentralized CU system.

Together with understanding the dynamics described above, we can highlight three cost advantages of the \hydrogen{} network in contrast to the \carbon{} network: Capturing carbon at the lowest-cost regions reduces 4 billion euros in spending on DAC. However, the \hydrogengrid{} model produces electrolytic hydrogen in the lowest-cost regions and in total spends 7 billion euros less on renewable energy (14 billion euros less on solar, and 7 billion euros more on wind) as well as 1 billion euros less on SMR. Furthermore, without a \hydrogen{} network, additional \hydrogen{} storage is needed in Spain at a cost of 5 billion euros to store hydrogen produced during the summer for utilization in FT fuel production during the winter.
And finally, the \carbongrid{} model spends 7 billion euros more on gas boilers, gas plants, gas infrastructure, and methanation.
% TODO: why??
The costs of the respective networks, on the other hand, are almost equal: the \carbongrid{} model spends 8 billion euros on the \carbon{} network, while the \hydrogengrid{} model even spends 10 billion euros on the \hydrogen{} network. These factors add up to account for 14 billion euros of the difference of 20 billion euros in system costs between the two scenarios (785 billion euros for the \carbongrid{} and 765 billion euros for the \hydrogengrid{} model, see Fig.~\ref{fig:objective_heatmap}).


\subsection*{Flexibility gains through hybrid configuration}

\begin{figure*}[ht!]
    \centering
    \begin{subfigure}{.49\textwidth}
        \centering
        \includegraphics[width=\linewidth]{full/figures/90_nodes/capacity_map_hydrogen.png}
        \caption{\hydrogen{} Sector.}
        \label{fig:capacity_map_hydrogen_co2}
    \end{subfigure}
    \hfill
    \begin{subfigure}{.49\textwidth}
        \centering
        \includegraphics[width=\linewidth]{full/figures/90_nodes/capacity_map_carbon.png}
        \caption{\carbon{} Sector.}
        \label{fig:capacity_map_carbon_co2}
    \end{subfigure}%
    \caption{Optimal production and transport capacities of the carbon and hydrogen sector in a net-zero energy system in Europe with both \carbon{} and \hydrogen{} network expansion (Hybrid).
    % Carbon network looks the same as in~\cite{morbeeOptimisedDeploymentEuropean2012}: two backbones, one in the nothern Europe other in south east.
    }
    \label{fig:capacity_maps}
\end{figure*}

% increased BECCS, decreased DACS
% decreased methanation

The Hybrid model combines the advantages of both hydrogen and carbon networks. While the topology of the \hydrogen{}-grid roughly corresponds to that of the \hydrogen{} grid model, the \carbon{}-grid plays more of a complementary role. Fig.~\ref{fig:capacity_maps} shows the capacities of the two grids in the hybrid model, with the production and transport capacities given in kt/h. Note that despite similar overall capacities, the investment in the hydrogen infrastructure is four times the investment in the carbon infrastructure.


Similar to the \hydrogengrid{} model, the Hybrid model takes advantage of a large hydrogen network that enables hydrogen to be transported from centralized production sites in western regions such as Spain to decentralized FT production sites across the continent. However, in the Hybrid model hydrogen in the UK is used locally to produce synfuels, and less hydrogen is transported to Central Europe.
The carbon network supplements this with smaller clusters of networks. In addition to a larger carbon grid cluster in Central Europe, which transports carbon to the North Sea, the system is building three large linear carbon routes. These transport carbon from Romania and Bulgaria to Greece, from northern Italy to central Italy and from Hungary, the Czech Republic and Poland to the Baltic Sea. These enable the system to transport carbon available at low cost from inland biomass sources and sequester it (see Fig.~\ref{fig:captureshare_line}). This reduces the need for DAC in the system and the sequestration sites are shifted to the end points of the carbon networks. Second-order effects can be seen in the shift in FT production compared to the \hydrogengrid{} model. Regions that have both good renewable resources and sequestration potential could now forego sequestration and use local process emissions for FT production, as can be observed in Ireland. A more detailed of differences between the \hydrogengrid{} and the Hybrid model can be found in Fig.~.


A notable aspect of the hybrid model is the lack of overlap in the hydrogen and carbon network topologies which is tightly connected to the nodal carbon strategy. Depending on the exogeneous need in hydrogen for example from industry, and the costs for locally produced carbon, a region can either import hydrogen to produce FT fuels or export carbon, but it will neither import one and export the other nor import or export both.


Despite only small decrease in cost in comparison to the \hydrogengrid{} model, hybrid approach brings several advantages. These include a broader range of technologies contributing to its robustness, reduced reliance on DAC, which may be costlier than anticipated in the cost projections, and less land use for wind and solar due to the decreased necessity for DAC. The increase used of biomass with carbon capture, transport and sequestration leads to slight cost advantages in comparison to the \hydrogengrid{} model.



\subsection*{Net-Negative Scenarios}

\begin{figure}[htb!]
    \centering
    \includegraphics[width=0.9\linewidth]{difference/comparison/emission-reduction-0.1/figures/90_nodes/cost_bar.png}
    \caption[short]{Net change in investments when tightening the \carbon{} emission target from net-zero to net-negative 10\% of 1990s emissions.}
    \label{fig:net-negative_cost_bar}
\end{figure}


We now examine the case of moving from a net zero emissions target to a net emissions reduction target. Fig.~\ref{fig:net-negative_cost_bar} illustrates the shift in investments for the \hydrogengrid{} and the Hybrid model between the Net-Zero and the Net-Negative scenario.

Both systems show a stron increase in DAC


\section{Conclusion}
\label{sec:conclusion}



In this study\dots

\dots

Despite the detailed model representation, there are some limitations to the validity of the results. The model itself is based on a linearized optimization with perfect foresight for the entire modeling year. In reality, long-term energy demand and renewable supply can only be roughly estimated, while short-term predictions still entail some uncertainty. The model's perfect foresight may lead to non-reproducible behavior, such as precisely aligning energy storage with future energy shortages at a specific point in time.

The technology costs used in the model rely on cost projections that incorporate reductions based on learning rates. These learning rates are derived from historical data, which may not necessarily be indicative of future trends. In particular, cost assumptions on DAC, electrolysis and \hydrogen{} pipeline costs, could have a strong impact in the model results.

The modelling results are heavily driven by the demand and emissions from industrial clusters. Allowing the model to relocate these and/or incorporate flexibility measures, may lead to less dependency on both \carbon{} and \hydrogen{} networks.

The technological flexibilities in the carbon sector might not be exploited to full extent. Therefore, the \carbon{} transport via truck or ship is not considered. Fischer-Tropsch facilities require to run on baseload with at least 90\% of their nominal capacity.

\dots



% TODO: which fixed rate? you mean that all emissions are captured by exogenous assumption or that we assume 90% of carbon can be captured?
% Furthermore, the model assumes a fixed rate for industrial process emissions, which cannot be altered through investments. This simplification may not accurately represent real-world scenarios, where many industries are considering adopting low-carbon processes and technologies. The transport of fuel (FT, gas, oil) between the regions is not unlimited which may overestimate the flexibility provided by these commodities.


\printbibliography

\appendix


\begin{figure*}
    \centering
    \includegraphics[width=\linewidth]{net-negative-0.1/full/figures/90_nodes/sankey_diagramm.png}
    \caption{Sankey diagram of the optimal operation for a net-negative 10\% scenario.}
    \label{fig:sankey_diagramm}
\end{figure*}

\begin{figure}
    \centering
    \includegraphics*[width=0.8\linewidth]{comparison/emission-reduction/figures/90_nodes/objective_heatmap.png}
    \caption{Total annual system cost for the different scenarios and \carbon{} targets. While the `Baseline' scenario has neither a \carbon{} nor a \hydrogen{} network, `Hybrid' is allowed to expand both.}
    \label{fig:objective_heatmap}
\end{figure}


\begin{figure}[ht]
    \centering
    \includegraphics[width=\linewidth]{comparison/emission-reduction-full/figures/90_nodes/transmission_cost_bar.png}
    \caption{Annual transmission system cost as a function of the net carbon removal scenarios considered in the study.}
    \label{fig:transmission_cost_bar}
\end{figure}


\begin{figure*}[ht]
    \centering
    \includegraphics[width=\linewidth]{difference/h2-only-full/figures/90_nodes/cost_map.png}
    \caption{Difference in cost investments between \hydrogengrid{} and Hybrid model. The left subfigure shows higher spendings per technology and region and transport system for the \hydrogengrid{} model, the right shows higher spendings in the Hydrid model.}
    \label{fig:cost_map_difference}
\end{figure*}


\end{document}

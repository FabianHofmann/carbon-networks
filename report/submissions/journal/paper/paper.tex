\documentclass[twocolumn]{article}
\usepackage{amsmath,amssymb,amsfonts}
\usepackage{cuted}  % Add the cuted package
\usepackage{caption}
\captionsetup{font=small}
\usepackage{algorithmic}
\usepackage{graphicx}
\usepackage{textcomp}
\usepackage{xcolor}
\usepackage{tabularx, multirow}
\usepackage{fancyhdr,lipsum}
\usepackage{subcaption}
\usepackage[shortcuts]{extdash}

\usepackage[%
backend=biber,bibencoding=utf8, %instead of bibtex
language=auto,
style=ieee,
sorting=none, % nyt for name, year, title
maxbibnames=10, % default: 3, et al.
%backref=true,%
natbib=true % natbib compatibility mode (\citep and~\citet still work)
]{biblatex}
\bibliography{../../../references.bib}

%define approx proportional
\def\app#1#2{%
  \mathrel{%
    \setbox0=\hbox{$#1\sim$}%
    \setbox2=\hbox{%
      \rlap{\hbox{$#1\propto$}}%
      \lower1.1\ht0\box0%
    }%
    \raise0.25\ht2\box2%
  }%
}
\def\approxprop{\mathpalette\app\relax}

% make abbreviation for co2
\newcommand{\COtwo}{CO$_2$}
\newcommand{\Htwo}{H$_2$}
\newcommand{\COgrid}{CO$_2$\=/Grid}
\newcommand{\Hgrid}{H$_2$\=/Grid}
\newcommand{\modBase}{Baseline model}
\newcommand{\modCO}{CO$_2$\=/Grid model}
\newcommand{\modH}{H$_2$\=/Grid model}
\newcommand{\modHybrid}{Hybrid model}

% alternative
\newcommand{\carbon}{CO$_2$}
\newcommand{\hydrogen}{H$_2$}
\newcommand{\carbongrid}{CO$_2$\=/Grid}
\newcommand{\hydrogengrid}{H$_2$\=/Grid}
\newcommand{\baelinemodel}{Baseline model}
\newcommand{\carbonmodel}{CO$_2$\=/Grid model}
\newcommand{\hydrogenmodel}{H$_2$\=/Grid model}
\newcommand{\hybridmodel}{Hybrid model}


\graphicspath{
    % {paper-figures}
    {figures}
    {../../../../results/}
    {results/}
}

\begin{document}


% FORMATTING:
% - https://www.nature.com/nenergy/submission-guidelines/aip-and-formatting
% - https://www.nature.com/nenergy/content abstract: 150 words, main text: 3000 words, figures: 8

\title{Assessing \Htwo{} and \COtwo{} Network Strategies in Europe}
% \title{Synthetic fuels in Europe: Transport Hydrogen to Carbon, or Carbon to Hydrogen?}

\author{
    Fabian Hofmann, Christoph Tries, Fabian Neumann, Lisa Zeyen, Tom Brown \\
    \textit{Institute of Energy Technology} \\
    \textit{Technical University of Berlin}\\
    Berlin, Germany \\
    m.hofmann@tu-berlin.de
}


\maketitle

\begin{abstract}
    Hydrogen and carbon dioxide transportation are considered crucial in climate-neutral energy systems, with hydrogen transport enabling energy imports to high-demand areas and carbon transport aiding in emissions export from high-emission areas. Yet, possible synergies and competitions between the two systems are not fully understood. Our study employs optimization techniques to develop a cost-optimal European energy system, integrating transport, storage, and sequestration of both carbon dioxide and hydrogen, along with renewable energy sources. Results indicate that a hydrogen network is more cost-effective than a carbon network, facilitating low-cost hydrogen imports to Central Europe and local carbon capture and utilization. However, in a hybrid scenario, the carbon network effectively complements the hydrogen network, promoting carbon capture from biomass and reducing reliance on direct air capture. This research demonstrates the effectiveness of combining hydrogen and carbon transport networks with power grid expansion in achieving climate neutrality in Europe.
\end{abstract}


\section{Introduction}

The transition to a carbon-neutral European economy is a pressing challenge that demands coordinated action across various energy sectors. While management of both \COtwo{} and \Htwo{} is considered a critical component of this transition, a gap exists in understanding how new hydrogen infrastructure effectively interacts with carbon management technologies, including carbon capture, transport, storage, utilization, and sequestration. Hydrogen offers an efficient way to transport and store energy over long distances and timeframes, and can be produced from electrolysis or steam methane reforming (SMR). Carbon can be stored in geological formations to prevent it from entering the atmosphere, a process known as carbon sequestration (CS). Additionally, carbon is a crucial material input for carbonaceous fuels needed for aviation (synthetic kerosene from Fischer-Tropsch synthesis), shipping (synthetic methanol) and industrial feedstocks (e.g., synthetic methane). From hereon, we jointly refer to the production of these carbonaceous fuels as Carbon Utilization (CU). Carbon can be effectively captured from industrial processes and the combustion of biomass, fossil fuels, or synthetic carbonaceous fuels through point-source carbon capture (CC) techniques, or harvested from the atmosphere using direct air capture (DAC).

It is important to differentiate between the functions of carbon capture, carbon sequestration and carbon utilization. While both academic literature and common language mostly use the term ``carbon capture, utilization and storage'' (CCUS), or similar, to refer to all three functions, we argue that this is misleading. There is inherent benefit in separating these three functions geographically (linked through carbon transport), temporally (linked through carbon storage), and in terms of economic actors performing each function. CC should be performed at the place and with the technology that can capture carbon at the lowest cost (including a potential transport premium). CS should take place where sequestration potentials exist and carbon is available at the lowest cost, and CU where the combination of hydrogen and carbon inputs is available at lowest cost. By employing this separation in our model, we allow for a nuanced optimization and detailed analysis of the carbon management system.

Recently, policymakers and industry in Europe have started developing carbon management strategies~\cite{GermanyDevelopingStrategy2023,CarbonManagementStrategie}, planning infrastructure components~\cite{CONetz}, and committing on the first carbon utilization projects~\cite{EFuelsPilotPlant2022,OrstedAssumesFull,GROUNDBREAKINGEFUELPRODUCTION,DLREfuelsDLR}. With the European Union's goal of achieving climate neutrality by 2050, a wide range of programs, funding models and initiatives have been established in this regard~\cite{eu2023netzero,europeangreendeal,europeaninnovationfund}. Initiatives like the European Hydrogen Backbone~\cite{gasforclimateEuropeanHydrogenBackbone2022} or the Hydrogen Infrastructure Map~\cite{H2InfrastructureMap} showcase the potential of hydrogen as a fuel and energy carrier, and some gas pipelines are already repurposed to transport hydrogen~\cite{RohrFreiFuer}. At the same time, business models from companies like Tree Energy Solutions~\cite{TESHydrogenLife2023}, Carbfix~\cite{WeTurnCO2}, and Equinor~\cite{adomaitisEquinorRWEBuild2023} advertise carbon management hubs that provide green hydrogen, methane, and synfuels on the one hand and offer \COtwo{} offtake on the other hand. The Northern Lights project in Norway~\cite{NorthernLightsWhat} is planning with a transport and sequestration capacity of 1.5 Mt \COtwo{} per year  to be operative in 2024, expanding to a targeted scale of 5 Mt per year of sequestration by 2030.
% TODO: frame next two sentences to be more market oriented
For Europe, the Capture Map~\cite{ToolsGreenTransition} estimates a potential of 1.7~Gt of carbon capture from point sources per year, which represent roughly 50\% of all emissions. In combination with large sequestration potentials as stated in~\cite{weiProposedGlobalLayout2021}, this highlights the vast potential for decarbonization.
To this end, the Clean Air Task Force underlines the importance of a carbon transport system in Europe to facilitate the carbon economy~\cite{lockwoodEuropeanStrategyCarbon}.

However, up to this point, it remains unclear how the two transport systems of hydrogen and carbon may complement or replace each other. Both can bridge the misalignment of sources (electrolysis, SMR, point-source CC, DAC) and sinks (CU, CS, industrial hydrogen demand) for carbon and hydrogen. A hydrogen network can supply regions with geographically fixed hydrogen demand, such as for steelmaking, with hydrogen from regions with the best renewable resources. A hydrogen network can also enable CU at the site of point-source CC. On the other hand, the carbon transport system allows for capturing and transporting carbon from areas with high emissions to regions with carbon sequestration potentials or high-quality renewable resources, the latter enabling cost-effective CU.
In the literature, the two network approaches and underlying technologies have been discussed in a number of publications, all of which, however, dealt with the isolated effects~\cite{bakkenLinearModelsOptimization2008, ,stewartFeasibilityEuropeanwideIntegrated2014,oeiModelingCarbonCapture2014,elahiMultiperiodLeastCost2014,burandtDecarbonizingChinaEnergy2019,middletonSimCCSOpensourceTool2020,bjerketvedtOptimalDesignCost2020,weiProposedGlobalLayout2021,damoreOptimalDesignEuropean2021,becattiniCarbonDioxideCapture2022}. Such techno-economic models, in contrast for example to Integrated Assessment Models, can account for the spatial distribution of carbon sources and sinks which are crucial provide a holistic view of the energy system and its technological interactions. The work in~\cite{neumannBenefitsHydrogenNetwork2022} examines the effect of a hydrogen network in Europe... (extend on hydrogen literature)

The publication by Morbee et al.~\cite{morbeeOptimisedDeploymentEuropean2012} optimizes the topology and capacity of a \COtwo{} network in Europe, but only considers the power sector without co-optimizing renewable deployment. This limited sectoral scope cannot capture important dynamics of carbon management, since it neglects the sectors that will need to handle most \COtwo{} in the future.
Another comprehensive example is found in~\cite{becattiniCarbonDioxideCapture2022}, which presents a mixed-integer model to optimize the time-evolution of a \COtwo{} transport system in Switzerland, connecting to a remote sequestration site in Norway. However, this limited spatial scope fails to consider other sequestration sites and co-benefits from connecting the \COtwo{} network to neighboring countries. Considering carbon pipeline as a technological option is a reasonable model assumption since \COtwo{} pipelines are a mature technology, given their wide-spread installations in the United States and Canada for enhanced oil recovery~\cite{righettiSitingCarbonDioxide2017,friedmannNETZEROGEOSPHERICRETURN}.
Other high-resolution energy system models do not feature detailed CU technologies and \COtwo{} transport. Euro-Calliope, for example, does neither feature carbon nor hydrogen transport, and thus forces generic synthetic fuel production to rely on captive electrolysis and DAC in the same region \cite{pickeringDiversityOptionsEliminate2022}.
In our model, we separate the functions of carbon capture (CC), carbon sequestration (CS) and utilization (CU).
% However, the models are often limited with regard to both geographical scope and detail. While representing a single country with spatial resolution may neglect synergies of international cooperation, a coarse grained representation of multiple countries may neglect important geographical properties.

To our knowledge, no study has yet considered the co-optimization and comprehensive assessment of both \COtwo{} and \Htwo{} networks in a fully sector-coupled energy system. However, we argue that such an assessment is strongly needed to avoid suboptimal investments and to identify synergies between hydrogen and carbon management technologies. In this paper, we present a detailed study of the European energy system for 2050, which includes high geographical resolution and a comprehensive representation of carbon management technologies. The study is conducted using the PyPSA-Eur energy system model and encompasses all relevant energy sectors. We explore competition between \COtwo{} and \Htwo{} networks by adding them separately. Our evaluation focuses on the transport of \COtwo{} and \Htwo{} through their respective networks on the European continent. We also analyze how an energy system with limited annual sequestration potential prioritizes decarbonization and fuel switching in various sectors, and how the construction of carbon networks varies based on different levels of available sequestration potential.


\section{Results}
\label{sec:results}


Focusing first on the Net-Zero scenario, we can highlight notable characteristics of the energy system buildouts shared by all models, as well as significant variations in system costs and technology deployment between the models (see Fig.~\ref{fig:cost_bar}).

\begin{figure}[ht!]
    \centering
    \includegraphics[width=\linewidth]{comparison/default/figures/90_nodes/cost_bar.png}
    \caption[short]{Total annual system cost subdivided into groups of technologies for the different models of the European energy system with a net-zero emission target. While the \modBase{} has neither a \COtwo{} nor an \Htwo{} network, the \modHybrid{} is allowed to expand both. ``Gas Infrastructure'' combines gas facilities for power and heat production, ``\COtwo{} Infrastructure'' combines transport, storage and sequestration, and ``\Htwo{} Infrastructure'' combines transport and storage. ``Carbon Capture at Point Sources'' combines all technologies with integrated carbon capture, including the cost of the main facility (e.g., CHP unit) and the carbon capture application.}
    \label{fig:cost_bar}
\end{figure}

Total annual system costs in the \modBase{} are highest, totalling \label{}803 billion euros. The models with additional transport options for carbon and hydrogen achieve a more efficient system allocation and lower total system costs, with around \label{}2\% cost reduction (\label{}781 billion euros) in the \modCO{}, and around \label{}4\% cost reduction in the \modH{} and the \modHybrid{} (\label{}765 and \label{}764 billion euros, respectively).

In all models, almost half of the system costs are spent on primary electricity production from wind and solar power (\label{}45\% or \label{}360 of \label{}799 billion euros total costs in the \modBase{}), with another \label{}12\% (\label{}100 billion euros) spent on hydro, nuclear and biomass. Costs for energy transport networks for electricity, gas, hydrogen and carbon are below \label{}12\% (\label{}100 billion euros), with the electricity grid making up more of the system costs than the other three networks combined. This is the case for all four models. Capital expenditures on carbonaceous fuel production are around \label{}3\% (\label{}25 billion euros) in each model, while electrolysis makes up around \label{}5\% (\label{}40 billion euros). Heat pump installations account for around \label{}8\% (60 billion euros) of total system costs.

System cost differences between the models are driven by renewable energy resources (solar, wind, and bioenergy), the \COtwo{} and \Htwo{} networks, as well as carbon capture technologies (DAC and CC at point sources).
As we show below, these are due to substantial, albeit fundamentally different, gains in flexibility and a shift in carbon utilization and storage strategies by the \Htwo{} and \COtwo{} networks.

\subsection*{Transport systems unlock point\=/source carbon capture}

In comparison to the other three models, the \modBase{} reveals significant inefficiencies: On the one hand, regions with no access to sequestration sites (i.e., ``landlocked'' regions) underutilize CC potentials from process emissions, fossil gas and biomass. On the other hand, sequestration sites rely on costly DAC to capture the carbon for CS from the air. Furthermore, hydrogen and power prices show high regional differences, indicating system inefficiencies due to transport constraints caused by non-existent hydrogen or insufficient power transmission. Prices are significantly higher in Central Europe, a region with high population density and high concentration of energy-intensive industries.

By enabling the efficient transportation of carbon, hydrogen, or both, these models reduce the reliance on high-cost DAC, and source more carbon from low-cost point-source CC on biomass and gas in industrial applications (see Fig.~\ref{fig:balance_captured_carbon}). The \modH{} only captures \label{}200~Mt of \COtwo{} from DAC compared to \label{}400~Mt in the \modBase{}. Instead, the \modH{} captures more carbon from process emissions as well as gas and biomass for industry. The \modCO{} uses only \label{}150~Mt from DAC, and instead also captures more carbon than the \modH{} from gas for industry and biomass CHPs. And finally, the \modHybrid{} captures only \label{}100~Mt from DAC, but slightly more carbon from all other CC technologies.
Also, the networks reduce the reliance on methanation in the model, reducing the overall demand for CC volumes. Another knock-on effect of the reduced reliance on DAC is reduced endogenous demand for renewables to power DAC facilities.


\begin{figure}[ht!]
    \centering
    \includegraphics[width=\linewidth]{comparison/default/figures/90_nodes/balance_bar_carbon.png}
    \caption{Balance of captured carbon for all models in the Net-Zero scenario. Positive values indicate carbon capture, negative values indicate carbon consumption. By integrating \Htwo{} and \COtwo{} networks, the predominant method for carbon removal shifts from Direct Air Capture (DAC) to a bioenergetic process that incorporates carbon capture. At the same time the reliance on methanation decreases.}
    \label{fig:balance_captured_carbon}
\end{figure}



The increase in captured carbon from point sources in the network models is mostly due to CC infrastructure being deployed on industrial or power plant facilities that are present in all models (as opposed to the deployment of additional electricity generating capacity for the purpose of carbon capture). This is evident when analyzing the capture share of the different CC technologies (see Fig.~\ref{fig:captureshare_line}).


\begin{figure}[h]
    \centering
    \includegraphics[width=\linewidth]{comparison/default/figures/90_nodes/captureshare_line.png}
    \caption{Proportion of plants with integrated carbon capture for all models in the net zero scenario. The size of the dots corresponds to the average capacity factor of the respective technology. While peak load technologies such as gas-fired combined heat and power (CHP) plants, which are only in operation on a few days a year, are not expanded for carbon capture, baseload point sources such as biomass CHP and process emissions from industry are fully (partially) developed as soon as the transport of \COtwo{} (\Htwo{}) is permitted.}
    % A CO2 network unlock BECSS potentials, most biomass emissions are captured at point sources and transported to sequestration sites. cite~\cite{rosaAssessmentCarbonDioxide2021}
    \label{fig:captureshare_line}
\end{figure}%

In the \modBase{}, the capture share is highest for biogas-to-gas facilities, followed by process emissions, biomass for industry, biomass CHPs, gas for industry, SMR and gas CHPs. For the network models, we can observe a correlation between CC share and capacity factors of the underlying technologies: Process emissions as well as gas and biomass for industry have high capacity factors (above 80\%), and CC shares close to 100\% (the only exception being gas for industry in the \modH{}). Biomass CHPs have a capacity factor of around 40\%, and CC shares of 50\% in the \modH{} and 95\% in the \modCO{}. SMR and gas CHPs have capacity factors below 20\%, and CC shares around 50\% and 0\%, respectively, across all network models. Gas CHPs serve as peak load electricity production technology, and with very few operating hours, the high investment costs into CC applications are not efficient.\footnote[1]{Similarly, SMR serves as ``peak-load'' hydrogen production technology in regions with poor renewable resources for electrolysis.}


In summary, the implementation of \COtwo{} and \Htwo{} networks is instrumental in unlocking low-cost CC potentials at geographically fixed point sources across the continent. While the \COtwo{} network facilitates the movement of carbon from low-cost sources to CU and CS sinks, reducing the reliance on high-cost DAC, the \Htwo{} network enables the cost-effective delivery of hydrogen from regions rich in renewable energy to regions with low-cost CC potentials to produce CU. However, despite the \modCO{} demonstrating higher absolute amounts of CC from point sources and higher CC shares across all point-source CC technologies, the \modH{} achieves lower total system costs (see Fig.~\ref{fig:cost_bar}). In the following section, we will explore the reasons for this.

% [potentially relevant text blobs from old 3.1]
% CU is preferably deployed in the regions where carbon, hydrogen as well as the inputs and by-product outputs (e.g., waste heat) of the synthesis processes have the lowest cost (or highest value).

% Now we turn to how carbon for CS is sourced by the different models. In the \modBase{}, because of the absence of a carbon network, all carbon that is sequestered (200 Mt, the same in each model) needs to be captured in the regions with access to an offshore CS site. The \modBase{} deploys all available low-cost point-source CC in these regions, and adds DAC in the sequestration locations with the lowest-cost electricity. In other words, the \modBase{} uses the atmosphere as a substitute carbon transport pathway, allowing \CO{} from inland point sources to escape into the atmosphere, and extracting it again from the atmosphere with DAC at the offshore locations.

% In the \modCO{}, the carbon network enables the system to capture carbon from low-cost point sources and transport it to the sequestration sites, trading off CC costs with transport costs.

% In summary, the \COtwo{} network enables the transport of carbon from the lowest-cost CC sources to CU and CS carbon sinks, reducing the need for high-cost DAC.
% The creation of an \Htwo{} network enables the cost-effective delivery of hydrogen from regions with a surplus of low-cost renewable energy to high-demand industrial centers with exogenous hydrogen demand and a surplus of low-cost CC potentials, reducing the need for DAC and for high-cost electrolysis at the industrial centers.
% This highlights how both the implementation of a \COtwo{} or an \Htwo{} network unlocks low-cost carbon capture potentials at geographically fixed point sources across the continent.


\subsection*{\Htwo{} network incentivizes distributed carbon utilization}\label{subsec:H2}

\begin{figure*}[ht!]
    \centering
    \includegraphics[width=\linewidth]{comparison/single-technologies/figures/90_nodes/balance_map_dedicated.png}
    \caption{Optimal operation, flows and prices of the carbon (top line) and hydrogen (bottom line) sectors for the \carbonmodel{} (left) and the \hydrogenmodel{} (right) in the net zero scenario. For each region, upper semicircles show the average production per technology, lower semicircles the consumption, and colors the average marginal prices. Lines and arrows show the interregional transportation. Carbon sequestration offshore are drawn as full circles. \carbon{} and \hydrogen{} transport goes from low price to high price areas to supply CU and, in the case of \carbonmodel{}, CS.
    }
    \label{fig:balance_map}
\end{figure*}

To find the reason for why the \Htwo{} network achieves lower system costs than the \COtwo{} network, it is crucial to understand the underlying mechanisms and geographical implications of the two transportation systems. Fig.~\ref{fig:balance_map} maps the average production, consumption and flow in the carbon (top row) and hydrogen (bottom row) sectors for the \modCO{} (left) and \modH{} (right). Each map displays circles indicating the production (top half-circle) and consumption (bottom half-circle) for each region, with lines and arrows representing the average import or export volumes and flow directions between regions. The color shading of each region corresponds to the average price of the carrier.

As we have also discussed above, the \modCO{} captures large amounts of low-cost carbon from bioenergy and industrial activities that were untapped in the \modBase{}, especially in Central and Eastern Europe, and almost none of which is used locally (see Fig.~\ref{fig:balance_map}, top left). A large part of the captured carbon is transported from Western Europe to the Iberian Peninsula, the British Isles and Denmark, where it is used along with low-cost electrolytic hydrogen to produce Fischer-Tropsch fuels and methanol. A second part of captured carbon is transported to methanation plants in Italy and Greece, where the synthetic methane output is fed into the well-established gas grid since Italy serves as a large gas import hub for Europe. Finally, a third part of captured carbon is transported directly to sites in the North Sea, Baltic Sea and Mediterranean Sea where it is sequestered. In absence of an \Htwo{} network, the system places most electrolysis in regions with the best renewable resources to supply local CU, and smaller amounts of electrolyzers where hydrogen is needed for industrial processes across Central Europe. The model also emphasizes gas imports into Central Europe. Hydrogen prices still vary strongly across the continent, with low price ``valleys'' in regions with good renewable resources such as the Iberian Peninsula or Denmark, and high price ``peaks'' in Central Europe (see bottom left).

To understand the dynamics of the \modCO{}, we observe that the system locates CU in the regions where it can minimize costs for its material inputs, carbon and hydrogen. As the \COtwo{} network largely smoothes out carbon prices across the continent, CU is located in the hydrogen price valleys on the periphery of the continent. Similarly, the model locates CS in regions where it can minimize costs for its only material input, carbon. Thus, CS is located in the carbon price valleys among the subset of regions with access to offshore sequestration sites. Due to the \COtwo{} network these valleys are not strongly pronounced, but still prevalent and impact CS locations. DAC is only used for CU and thus co-located in regions with a mix of low electricity costs (for running DAC) and low hydrogen costs (where CU is located, see above).

The \Htwo{} network on the other hand, transports low-cost hydrogen from regions with abundant renewable resources to regions with low-price carbon from abundant sources of fossil, biogenic and process emissions (see bottom right). The main transportation routes go from Spain and the United Kingdom to Central Europe to produce synfuels with locally captured carbon from point sources. Also, the \Htwo{} network supplies exogenous \Htwo{} demand for industry in Central Europe at much lower costs than local electrolysis or SMR.

To explain the dynamics of the \modH{}, we observe that the model locates CU in the carbon price valleys across Western, Central and Eastern Europe (see top right), as the \Htwo{} network smoothes out hydrogen prices across the entire continent (see bottom right). CS is again located in the carbon price valleys among those regions with access to offshore sequestration sites. Due to the lack of a \COtwo{} network the full CC potentials from point sources are exhausted in all offshore sequestration regions, and the carbon is then sequestered. The additionally needed carbon to reach 200~Mt of CS is captured via DAC in the offshore sequestration regions with the lowest costs for DAC, with the largest share deployed in Portugal (see top right). DAC is used for both CU and CS in this model, and for both purposes DAC is located in the regions with the lowest costs for electricity.

In summary, both network models prefer distributed CC at point sources and electrolysis in regions with abundant renewable resources, choosing the lowest-cost production sites for each while trading off transport costs. However, differences between the two models are notable for the placement of CU, CS and DAC due to their distinct \COtwo{} and \Htwo{} price patterns. CU locations are more distributed in the \modH{}, while they cluster in a few pockets of low-cost \Htwo{} in the \modCO{}. CS co-locates in the regions with the lowest \COtwo{} costs (among those with offshore sequestration potential), but these regions are very different between the models. DAC is placed in regions with the best renewable resources in both models, but it is used for only CU in the \modCO{} and for both CU and CS in the \modH{}. Finally, the system cost advantage of the \modH{} stems from being able to supply the exogenous hydrogen demands in Central Europe with low-cost hydrogen imports, while the \modCO{} is forced to deploy some electrolysis in regions with poor renewables resources as well as additional gas infrastructure in Central Europe.


\subsection*{Hybrid configuration provides further flexibilities}\label{subsec:Hybrid}

\begin{figure*}[ht!]
    \centering
    \begin{subfigure}{.5\textwidth}
        \centering
        \includegraphics[width=\linewidth]{full/figures/90_nodes/balance_map_carbon.png}
        \label{fig:capacity_map_carbon_co2}
    \end{subfigure}%
    \begin{subfigure}{.5\textwidth}
        \centering
        \includegraphics[width=\linewidth]{full/figures/90_nodes/balance_map_hydrogen.png}
        \label{fig:capacity_map_hydrogen_co2}
    \end{subfigure}
    \caption{Optimal operation, flows and prices of the carbon (left) and hydrogen (right) sectors for the \hybridmodel{} in the net zero scenario. For each region, upper semicircles show the average production per technology, lower semicircles the consumption, and colors the average marginal prices. Carbon Sequestration offshore are drawn as full circles. Lines and arrows show the interregional transportation. \carbon{} from point-source in the inland either supplies local CU with imported \hydrogen{} or facilitates sequestration in nearby offshore regions.
    % Carbon network looks the same as in~\cite{morbeeOptimisedDeploymentEuropean2012}: two backbones, one in the nothern Europe other in south east.
    }
    \label{fig:capacity_maps}
\end{figure*}

The \modHybrid{} combines the advantages of hydrogen and carbon networks. While the topology of the \Htwo{} network roughly corresponds to that of the \modH{}, the \COtwo{} network plays more of a complementary role. Fig.~\ref{fig:capacity_maps} shows the average production, consumption and transport of the two networks in the \modHybrid{}. Note that despite similar overall capacities in terms of weight moved, the investment in the hydrogen infrastructure is four times the investment in the carbon infrastructure.


Similar to the \modH{}, the \modHybrid{} takes advantage of a large hydrogen network that enables hydrogen to be transported from centralized production sites in western regions such as Spain to decentralized FT production sites across the continent. However, in the \modHybrid{} hydrogen in the UK is used locally to produce synfuels, and less hydrogen is transported to Central Europe.
The carbon network supplements this with smaller clusters of networks. In addition to a larger \COtwo{} network cluster in Central Europe, which transports carbon to the North Sea, the system is building three large linear carbon routes. These transport carbon from Romania and Bulgaria to Greece, from northern Italy to central Italy and from Hungary, the Czech Republic and Poland to the Baltic Sea. These enable the system to transport carbon available at low cost from inland biomass sources and sequester it (see Fig.~\ref{fig:captureshare_line}). This reduces the need for DAC in the system and the sequestration sites are shifted to the end points of the carbon networks. Second-order effects can be seen in the shift in FT production compared to the \modH{}. Regions that have both good renewable resources and sequestration potential could now forego sequestration and use local process emissions for FT production, as can be observed in Ireland. A notable aspect of the hybrid model is the lack of overlap in the hydrogen and carbon network topologies. This finding is tightly linked to the fact of competing functionalities to offtake carbon emissions.
A more detailed of differences between the \modH{} and the \modHybrid{} can be found in Fig.~\ref{fig:cost_map_difference}.


Despite only a further reduction of \label{}0.4 percentage points (\label{}3 billion euros) in total system costs compared to the \modH{}, the \modHybrid{} brings several advantages. These include a broader range of technologies contributing to its robustness, reduced reliance on DAC, which may be costlier than anticipated in the cost projections, and less land use for wind and solar due to the decreased necessity for DAC. The increase used of biomass with carbon capture, transport and sequestration leads to slight cost advantages in comparison to the \modH{}.



\subsection*{Net-negativity amplifies DAC and Bioenergy with CU and CS}\label{subsec:NN}

\begin{figure}[htb!]
    \centering
    \includegraphics[width=0.9\linewidth]{difference/comparison/emission-reduction-0.1/figures/90_nodes/cost_bar.png}
    \caption[short]{Net change in investments when tightening the \COtwo{} emission target from net-zero to net-negative 10\% of 1990s emissions. For all models, Direct Air Capture (DAC) contributes most to the additional \carbon{} removal, requiring further solar, wind and heat pump capacities for electricity and heat input.}
    \label{fig:net-negative_cost_bar}
\end{figure}


In transitioning from a net zero to a net emissions reduction target, a continuation of investment trends are perceived. Fig.~\ref{fig:net-negative_cost_bar} shows the net investment changes per technology group for the \Hgrid{} and \modHybrid{}s. The scenario imposes a net carbon removal of 460 Mt/a, with both models incurring a similar net cost increase - 95 billion €/a for the \modH{} and 96 billion €/a for the \modHybrid{}, balancing the earlier cost advantage of the \modHybrid{}. Both models invest relatively evenly in additional DAC, wind and heat pump systems, with DAC accounting for more than a third of the additional system costs. These remove carbon directly at sequestration sites at the coast, in particular in Portugal and UK, together with heat and power from new heat pumps at site and new solar and wind installation in the vicinity.
More heat pumps, solar and wind installations across all Central and Eastern European countries as well as biogas-to-gas installations compensate for the reduction of fossil gas based heating and powering. Reduced usage of gas in general lower the investments in gas infrastructure in both models.
Fundamental difference in investment strategies occur in the carbon capture sector: In contrast to the \modH{}, the \modHybrid{} continues the strategy of sequestering carbon captured and transported from bio-energetic inputs, expanding the carbon transport routes from Central and Eastern Europe to the near shores. At the same time, carbonaceous fuel production in Central Europe partially moves to Spain and UK, where it uses carbon from new DAC facilities without the installation of a \COtwo{} network. At the same time, solar based electrolysis in France, Nothern Italy and Switzerland expands requiring more solar capacities but less hydrogen imports from the Iberian Peninsula.
The \modH{} on the other hand, expands decentralized CU systems from the Net-Zero scenario, which are now increasingly supplied with carbon from biogas-to-gas facilities, showing up in both ``Bioenergy'' and ``Carbon Capture at Point Sources'' cost-contributions. The centralized DAC facilities at sequestration sites are more expanded than in the \modHybrid{} and are requiring additional heat from biomass CHPs.


These findings highlight the continuation of trends of the two systemic approaches, which are interchangable at marginal cost-difference: While a stand-alone \Htwo{} network combines well with highly centralized DAC and decentralized CU with bioenergy as primary carbon source, the complementation with a \COtwo{} network promotes combining DAC and CU where good renewable resources are and collecting, transporting and sequestering carbon from decentral bioenergetic processes.

% The difference in absolute investment changes between the two models, regionally displayed in Fig.~\ref{fig:net-negative_cost_map}.
% \begin{figure*}[htb!]
%     \centering
%     \includegraphics[width=\linewidth]{difference/net-negative-0.1-h2-only-full/figures/90_nodes/cost_map.png}
%     \caption[short]{Net difference in investments between the \modH{} and the \modHybrid{} for the Net-Negative scenario.}
%     \label{fig:net-negative_cost_map}
% \end{figure*}


% \section{Limitations}
% \label{sec:limitations}
% Despite the detailed model representation, there are some limitations to the validity of the results. The model itself is based on a linearized optimization with perfect foresight for the entire modeling year. In reality, long-term energy demand and renewable supply can only be roughly estimated, while short-term predictions still entail some uncertainty. The model's perfect foresight may lead to non-reproducible behavior, such as precisely aligning energy storage with future energy shortages at a specific point in time.
% The technology costs used in the model rely on cost projections that incorporate reductions based on learning rates. These learning rates are derived from historical data, which may not necessarily be indicative of future trends, and they are based on projected global installed capacities of one technology, which can result in large uncertainties. In particular, cost assumptions on DAC, electrolysis and \Htwo{} pipeline costs, could have a strong impact in the model results.
% The modelling results are heavily driven by the demand and emissions from industrial clusters. Allowing the model to relocate these and/or incorporate flexibility measures, may lead to less dependency on both \COtwo{} and \Htwo{} networks.
% The technological flexibilities in the carbon sector might not be exploited to full extent. Therefore, the \COtwo{} transport via truck or ship is not considered. Fischer-Tropsch facilities require to run on baseload with at least 90\% of their nominal capacity.
% Furthermore, the model assumes a fixed rate for industrial process emissions, which cannot be altered through investments. This simplification may not accurately represent real-world scenarios, where many industries are considering adopting low-carbon processes and technologies. The transport of fuel (FT, gas, oil) between the regions is not unlimited which may overestimate the flexibility provided by these commodities.


\section{Conclusion}
\label{sec:conclusion}

This study offers a detailed exploration of Europe's energy system in 2050, focusing on the integration and optimization of hydrogen and carbon networks to achieve climate neutrality. The analysis reveals significant efficiencies gained through these networks, particularly in terms of leveraging low-cost carbon sources and facilitating a decentralized approach to carbon utilization.

The hydrogen network demonstrates superior cost-effectiveness, primarily due to its capacity to transport low-cost hydrogen to high-demand regions and promote decentralized carbon capture and utilization. This network reduces reliance on direct air capture (DAC), offering a shift towards more economically viable and sustainable solutions.
The combination of hydrogen and carbon networks in the hybrid model introduces additional flexibilities. This model not only achieves further cost reductions but also shows resilience against potential increases in DAC costs and reduces land use for renewable energy installations.
Under net-negative emission scenarios, the importance of DAC and bioenergy with carbon capture and storage (BECCS) is amplified. Both the hydrogen and hybrid models demonstrate adaptability to these more stringent climate targets, highlighting the robustness of the networked approaches.

% TODO: distribution of current refineries might shape the CU network. These are quite decentralized and might fit to the cost-optimal layout.

The study conclusively demonstrates the pivotal role of strategically integrated hydrogen and carbon networks in steering Europe towards a sustainable and climate-neutral energy future. It emphasizes the necessity of comprehensive infrastructure planning, integrating diverse energy carriers and technologies to achieve these objectives effectively.


\section{Methodology}
\label{sec:methodology}

The study is conducted on the basis of the open-source, capacity-expansion model PyPSA-Eur~\cite{horschPyPSAEurOpenOptimisation2018,brownSynergiesSectorCoupling2018,PyPSAEurSecSectorCoupledOpen2023}.
The model optimizes the design and operation of the European energy system, encompassing the power, heat, industry, waste, agriculture, and transport sectors, including international aviation and shipping.

\begin{figure}
    \includegraphics[width=\linewidth]{baseline/figures/90_nodes/demand_bar.png}
    \caption{Assumptions on exogenous demand, derived from~\cite{piamanzGeoreferencedIndustrialSites2018,muehlenpfordtTimeSeries2019,mantzosJRCIDEES20152018,NationalEmissionsReported2023,EurostatCompleteEnergyBalance,uwekrienDemandlib2023}. The figure shows the total annual energy demand for each energy source which determine the model's endogenous investments and operation. Endogenous processes can lead to higher total production volumes of some energy carriers, e.g., the demand for methanol requires more hydrogen and carbon as secondary (energy) inputs, which are not considered here. The model are defined per region and time stamp.}
    % TODO: adjust labels to show that these are exogenous assumptions
    \label{fig:total-demand-bar}
\end{figure}
%
In our configuration, the model's time horizon spans one year with a temporal resolution of 3 hours and a spatial resolution of 90 regions. Each of the regions consists of a complex subsystem with technologies for supplying, converting, storing and transporting energy. Exogenous assumptions on energy demand and non-abatable emissions are taken from various sources~\cite{piamanzGeoreferencedIndustrialSites2018,muehlenpfordtTimeSeries2019,mantzosJRCIDEES20152018,NationalEmissionsReported2023,EurostatCompleteEnergyBalance,uwekrienDemandlib2023}. The energy demand for electricity, transport, biomass, heat and gas are defined per region and time-step.
Land transport demand is exogenously divided between electric vehicles (85\%) and fuel cell vehicles (15\%), the latter representing demand for heavy duty land transport. Demands for kerosene for aviation, methanol for shipping, and naphtha for industry are aggregated in the system scope and kept constant throughout all time-steps. Heat demand is regionally subdivided into shares of urban, rural and industrial sites. We show the sum of all energy demands in Fig.~\ref{fig:total-demand-bar}. The system emits 633~Mt \COtwo{} per year from industry, aviation, shipping and agriculture, 153~Mt of which are fossil-based process emissions. Industrial energy demand and excess heat potentials are calculated per node on the basis of~\cite{hotmaps_industrial_db}.
% use https://pypsa-eur.readthedocs.io/en/latest/licenses.html


Endogenous model results include the expansion of renewable energy sources, storage technologies, transmission capacities, heating technologies, peaking power plants, and the deployment of gray, blue or green hydrogen, among others.
The model considers various energy carriers like electricity, hydrogen, methan, methanol, liquid hydrocarbons and biomass, together with conversions technologies.
Carbon-neutral electricity is provided by wind, solar, biomass, hydro and nuclear power plants. Hydro and nuclear plant capacities cannot be extended. Weather-dependent power potentials for solar, wind and hydro are calculated from reanalysis and satellite data sets~\cite{hersbachERA5GlobalReanalysis2020,pfeifrothSurfaceRadiationData2017}  per region and time-stamp, using the open-source tool Atlite~\cite{hofmannAtliteLightweightPython2021}.
Solar and wind power can be expanded in alignment with eligible land-use restrictions calculated on the basis of~\cite{eeaCorineLandCover2012,eeaNatura2000Data2016}. We restrict the electric transmission system's expansion to 20\% of its current capacity, acknowledging the challenges in inaugurating new transmission projects.
For the use of biomass we consider only residual biomass products and no energy crops. We limit regional biomass use to the medium-level potentials from the ENPRESO database~\cite{enspreso_database,instituteforenergyandtransportjointresearchcentreJRCEUTIMESModelBioenergy2015}. Inter-regional biomass transport is permitted with transport costs considered.

In our model, we consider green, blue and gray hydrogen from electrolysis and steam methane reforming (SMR), the latter of which may be equipped with CC technology. Geological distribution and potentials for \Htwo{} storage in salt caverns are derived from~\cite{caglayanTechnicalPotentialSalt2020}. Re-electrification of hydrogen is possible via fuel cells. If enabled, hydrogen can be transported via pipelines between regions which can be expanded without limit, considering costs for pipeline segments and compressors. Pipeline flows are modelled using net transfer capacities and without flow dynamics, pressure valves, or energy demand for compression.
% No retrofitting of gas pipelines is considered, potentially overestimating hydrogen network cost.

The topology of the \COtwo{}, \Htwo{} and gas network is identical to the topology of the electricity network, connecting all neighboring regions. Transport of liquid fuels like oil, methanol and Fischer-Tropsch (FT) fuels is ``copper-plated'', since transport costs per unit of energy are negligible due to their high energy density. Throughout this study, we focus on \COtwo{} and \Htwo{} networks because of their high relevance for infrastructure investments and thus public policy decisions. The electricity network and gas network are both included in the model with full geographical detail, but not further analyzed. Electricity networks are already in place, and we restrict further extension due to concerns of public acceptance. Gas networks also already exist, and will most likely experience decreased use in the future, removing any bottlenecks or constraints on the optimal system buildout or operation.

Our model features three drop-in fuel production technologies for CU: methanation, methanolization, and Fischer-Tropsch synthesis. The processed fuels are not limited in their total quantity of use or production. Methane is transported through the gas network, while methanol and FT fuel can be transferred inter-regionally without additional costs, since dense fuels have negligible transport costs per unit of energy. Synthetic methane substitutes natural gas or biogas, serving combined heat and power plants, residential heating gas boilers, or industrial process heat. Synthetic methanol decarbonizes marine industry fuel demands, and FT fuels replace fossil oil for naphtha production, aviation kerosene, or agricultural machinery oil.

To supply carbon needed for CS and CU, the system can choose to deploy carbon capture technology at various point sources (see below), or through DAC facilities. The concept of a merit order for captured carbon plays a pivotal role in optimizing the deployment of carbon capture technologies based on their economic feasibility. This merit order ranks all CC technologies according to their relative costs to capture an additional marginal ton\footnote[2]{The cost value for a technology depends on the exact system configuration and may vary by scenario and even by region, since many technologies produce other (primary) outputs such as electricity and heat. The costs to capture an additional marginal ton of carbon may also be different from the average cost of capture.} of carbon (see Fig.~X). At the lower end of the cost spectrum, CC technologies applied to process emissions, such as those from cement, offer a cost-effective starting point. Following this, biomass combined heat and power (CHP) systems provide a dual benefit of energy production and carbon capture. Moving up the scale, gas used in industrial applications and biomass employed for industrial processes represent more costly yet viable options to capture carbon. DAC, an emerging technology capable of extracting \COtwo{} directly from the atmosphere, stands at a higher cost level due to its current high capital costs and energy demand. Finally, based on our model results, biogas-to-gas upgrading, a process that refines biogas to natural gas quality, incurs the highest costs in the merit order per marginal ton of captured \COtwo{}. Biogas input is a high-cost fuel which based on the endogenous modeling decisions is not used in large quantities. To the extent that biogas-to-gas facilities are used by the model to supply additional (carbon-neutral) gas as fuel, adding CC infrastructure incurs low costs (and thus ranges on the left-hand side of the merit order curve). However, the price of capturing additional \COtwo{} from biogas-to-gas upgrading is high because a substantial amount of the biogas fuel costs factors into the marginal cost of captured \COtwo{}. This merit order framework is crucial for strategically deploying necessary carbon capture solutions while balancing economic considerations. If we consider the spatial aspect of distributed carbon capture potentials, the \COtwo{} network can exploit comparative cost advantages between CO2 bidding zones to utilize the lowest-cost CC facilities across the continent. We assume a capture rate of 90\% for CC on process emissions, SMR, biogas-to-gas, as well as gas and biomass used in industry, and 95\% for CC on biomass and gas CHPs.

To store carbon, we differentiate between short-term storage in steel tanks and long-term, irreversible sequestration in underground sequestration sites such as porous rock formations or depleted gas reservoirs. Costs for both options are included in the model. For carbon sequestration, we only consider offshore sites as potential sinks. We make this choice because offshore sequestration sites typically have larger capacity compared to onshore sequestration sites in saline aquifers and due to concerns over public safety for infrastructure near populated areas. Our estimates for carbon sequestration potentials are conservative, limiting total sequestration to 25~Mt per site and calculating annual storage availability over 25 years. Furthermore, we decide to cap the total sequestered \COtwo{} to 200~Mt per year for our Net-Zero scenario. This is enough to offset the hardest-to-abate fossil process emissions and some limited slack of fossil fuels to be used where they can be most valuable. This is in order to avoid offsetting fossil emissions with carbon removal if they can technologically be avoided in the first place. All technology cost assumptions are taken for the year 2040 and sourced from an open-source database~\cite{lisazeyenPyPSATechnologydataTechnology2023}.


Addressing climate targets, we define two \COtwo{} target scenarios:
\begin{itemize}
    \item[] \textit{Net\=/Zero}: aligning with the EU's 2050 emission targets. Also denoted as NZ.
    \item[] \textit{Net\=/Negative}: 10\% net-negative emissions relative to 1990 levels, equaling 460~Mt \COtwo{} annually. Also denoted as NN.
\end{itemize}
Unless otherwise mentioned, our reference is the Net-Zero scenario. For the Net-Negative scenario, the cap of total carbon sequestration is adjusted from 200~Mt to 660~Mt per year accordingly.

Beyond \COtwo{} targets, we consider four expansion scenarios, referred to as models:
\begin{itemize}
    \item[] \textit{Baseline:} Neither \COtwo{} nor \Htwo{} networks are constructed.
    \item[] \textit{\COgrid{}:} Only the \COtwo{} network is developed.
    \item[] \textit{\Hgrid{}:} Only the \Htwo{} network is developed.
    \item[] \textit{Hybrid:} Both \COtwo{} and \Htwo{} networks are developed.
\end{itemize}

\printbibliography

\appendix


% \begin{figure*}
%     \centering
%     \includegraphics[width=\linewidth]{full/figures/90_nodes/sankey_diagramm.png}
%     \caption{Sankey diagram of the optimal operation for a net-zero scenario.}
%     \label{fig:sankey_diagramm}
% \end{figure*}

% \begin{figure*}
%     \centering
%     \includegraphics[width=\linewidth]{net-negative-0.1/full/figures/90_nodes/sankey_diagramm.png}
%     \caption{Sankey diagram of the optimal operation for a net-negative 10\% scenario.}
%     \label{fig:sankey_diagramm}
% \end{figure*}

\begin{figure}
    \centering
    \includegraphics*[width=0.8\linewidth]{comparison/emission-reduction/figures/90_nodes/objective_heatmap.png}
    \caption{Total annual system cost for the different network models and \COtwo{} targets. While the \modBase{} has neither a \COtwo{} nor an \Htwo{} network, the \modHybrid{} is allowed to expand both.}
    \label{fig:objective_heatmap}
\end{figure}


\begin{figure}[ht]
    \centering
    \includegraphics[width=\linewidth]{comparison/emission-reduction-full/figures/90_nodes/cost_bar_transmission.png}
    \caption{Annual transmission system cost as a function of the net carbon removal scenarios considered in the study.}
    \label{fig:cost_bar_transmission}
\end{figure}


\begin{figure*}[ht]
    \centering
    \includegraphics[width=\linewidth]{difference/h2-only-full/figures/90_nodes/cost_map.png}
    \caption{Difference in regional costs between the \Hgrid{} and \modHybrid{}s. The left subfigure shows higher spendings per technology and region and transport system for the \modH{}, the right shows higher spendings in the \modHybrid{}.}
    \label{fig:cost_map_difference}
\end{figure*}

\begin{figure}[htb!]
    \centering
    \includegraphics[width=0.9\linewidth]{difference/comparison/subsidy/figures/90_nodes/cost_bar.png}
    \caption[short]{Net change in system cost when reducing assumed costs for \COtwo{} pipelines by 50\%.}
    \label{fig:half-CO2-price_cost_bar}
\end{figure}


% [Dynamics \modCO{} vs \modH{} (old text blocks)]
% There are four notable dynamics or differences between the two network strategies:
% First, the production of FT fuels always moves to where low-cost carbon and low-cost hydrogen can be brought together. In the \modCO{ }, carbon is moved from Western Europe to the Iberian Peninsula, the British Isles and to Denmark to produce FT fuels centrally in few regions. In the \modH{}, hydrogen is transported in the opposite direction to produce FT fuels decentrally across all regions in Central Europe with locally captured \COtwo{} from point sources.
% Second, the \modCO{ } captures more low-cost carbon from biomass, especially in Central and Eastern Europe, and less high-cost carbon from DAC in Greece and Southern Italy. Instead, the \COtwo{} network transports carbon from Central and Eastern Europe to Greece and Italy, where it is sequestered, or feeds into CU together with locally produced, low-cost electrolytic hydrogen. The \modH{} directly operates electrolyzers and DAC in Greece and Southern Italy, with electricity from low-cost renewables, and thus avoids significant network buildouts in this region.
% Third, sequestration in the \modH{} occurs more decentralized, spread out across several regions with access to offshore sequestration sites, making use of point-source CC potentials in each of these regions. In the \modCO{ }, sequestration is clustered at fewer sequestration sinks, each collecting low-cost \COtwo{} from neighboring regions. However, there is no centralization at only one or two sites, since all the sequestration sites offer the same costs and the system minimizes system-wide carbon transport distances to about eight to ten sequestration sites.
% And fourth, how the three types of CU are spread across the continent is driven by the costs of carbon and hydrogen (the latter mostly driven by the costs of renewable electricity) as well as by renewables capacity factors: Clearly, all carbonaceous fuel production requires sites where both carbon and hydrogen are available, and they moreover prefer sites with low carbon and hydrogen prices. Based on our model assumptions, FT fuel synthesis has very limited flexibility and needs to be run alost at baseload (ramp-down only possible to 90\% of maximum load). Thus, for FT fuel synthesis sites in Northern Europe with more electricity from wind power are preferred to sites in Southern Europe, powered mostly by solar. Methanathion and Methanolisation both have higher flexibility, being able to ramp down to 50\% load during the night, for example. Thus, these fuel production sites prefer the best renewable resource sites in Southern Europe. And finally, as methanation produces synthetic methane that needs to be fed into the gas grid, methanation prefers those sites with good interconnection to the existing gas network. Since Italy possesses a large gas network with easy interconnection to Central Europe, methanation prefers sites in Southern Italy (and Greece, which would be connected to Southern Italy) over the Iberian Peninsula, where Methanolisation is more prevalent.

% In summary, the implementation of a \COtwo{} network allows for connecting regions with abundant, low-price carbon (from point sources) to regions that can sequester the carbon or that can produce low-cost hydrogen to make CU products (FT fuel, methanol, synthetic methane). DAC and electrolysis are deployed in the regions with the best renewable resources to meet carbonaceous fuel demands.
% In the absence of a \COtwo{} network and with an \Htwo{} network on the other hand, regions with access to offshore sequestration sites capture all carbon from low-cost point sources and then sequester it. The system deploys additional DAC facilities at the locations with sequestration access and low-cost renewables to capture the remaining necessary carbon for sequestration. Carbonaceous fuel demand is met at the locations which offer the best mix of low-cost carbon from point sources and low-cost hydrogen from renewables, with some additional electrolysis and DAC required to fully meet the demand.
% Sequestered carbon is nearly always transported before

% Therefore, one can conclude that a \COtwo{} network favors a decentralized CC and a CU centralized at regions with abundant renewable resources. An \Htwo{} network favors a centralized electrolysis system at regions with abundant renewable resources and a decentralized CU system.

% Together with understanding the dynamics described above, we can highlight three cost advantages of the \Htwo{} network in contrast to the \COtwo{} network: Capturing carbon at the lowest-cost regions reduces 4 billion euros in spending on DAC. However, the \modH{} produces electrolytic hydrogen in the lowest-cost regions and in total spends 7 billion euros less on renewable energy (14 billion euros less on solar, and 7 billion euros more on wind) as well as 1 billion euros less on SMR. Furthermore, without an \Htwo{} network, additional \Htwo{} storage is needed in Spain at a cost of 5 billion euros to store hydrogen produced during the summer for utilization in FT fuel production during the winter.
% And finally, the \modCO{ } spends 7 billion euros more on gas boilers, gas plants, gas infrastructure, and methanation.
% TODO: why??
% The costs of the respective networks, on the other hand, are almost equal: the \modCO{ } spends 8 billion euros on the \COtwo{} network, while the \modH{} even spends 10 billion euros on the \Htwo{} network. These factors add up to account for 14 billion euros of the difference of 20 billion euros in system costs between the two scenarios (785 billion euros for the \COgrid{} and 765 billion euros for the \modH{}, see Fig.~\ref{fig:objective_heatmap}).


\end{document}

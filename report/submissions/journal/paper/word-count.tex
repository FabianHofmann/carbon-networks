Hydrogen and carbon dioxide transport can both play an important role in climate-neutral energy systems. Hydrogen networks help serve regions with high energy demand, while excess emissions are transported away in carbon dioxide networks. When it comes to the synthesis of carbonaceous fuels, it is less clear which input should be transported: hydrogen to carbon point sources or carbon to low-cost hydrogen. We explore the potential synergies and competition of both networks in a cost-optimal carbon-neutral European energy system. In direct comparison, a hydrogen network is more cost-effective than a carbon network, since it serves to transport hydrogen to demand and to point source of carbon for utilization. However, in a hybrid scenario where both networks are present, the carbon network effectively complements the hydrogen network, promoting carbon capture from biomass and reducing reliance on direct air capture. This study shows how the European energy system in a future net-zero and net-negative emissions scenarios may benefits from integrating both networks.

    In future climate-neutral scenarios, the transport of hydrogen and carbon dioxide can play competing roles.
    Hydrogen transport allows energy-intensive regions to import low-cost hydrogen. Carbon transport allows regions with carbon sinks such as carbon sequestration and synthetic fuel production to import low-cost carbon from distributed carbon capture sources.
    Whether carbon and hydrogen transport networks are complementing or substituting each other remains unclear. This is especially relevant for the production of carbonaceous fuels such as Fischer-Tropsch fuels, methanol or synthetic methane, which are made from both hydrogen and carbon dioxide, and in most locations require at least one of these feedstocks to be imported.
    To address this gap, our study employs optimization techniques to design the first cost-optimal European energy system that fully incorporates carbon management technologies, hydrogen transport and storage, and renewable energy sources with a high spatial and temporal resolution considering all energy intensive sectors.
    Our findings reveal that when comparing either-or-models, a system with an hydrogen network provides lower system costs than a system with a carbon network. The former enables the transportation of hydrogen from centralized production sites with low-cost renewable energy in the Iberian Peninsula and the British Isles to serve distributed hydrogen demands across the European continent. The latter facilitates carbon capture at lower costs from distributed point sources and reduces the need for higher cost direct air capture. However, the system with only a carbon network needs to meet distributed hydrogen demand across the continent by locally deploying electrolysis regardless of renewable energy costs.
    In a hybrid model with both types of networks, the carbon network is a cost-effective complement to the hydrogen network facilitating point-source carbon capture at low costs and transport to sequestration sites. In all models, additional demand for synthetic fuels is met at sites with favorable renewable energy resources that enable both electrolysis and direct air capture at low costs.
    We show that theses findings holds true against the backdrop of a net-zero energy system as well as a scenario with net-negative emissions.
    Overall, our work demonstrates the cost-effectiveness of a multi-grid system that includes both hydrogen and carbon transport networks and power grid expansion to achieve climate neutrality in Europe.
    The paper underpins the need for a complementing hydrogen network and \carbon{} network to achieve climate neutrality.

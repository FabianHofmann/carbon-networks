Hydrogen and carbon dioxide transportation are considered crucial in climate-neutral energy systems, with hydrogen enabling energy imports to high-demand areas and carbon transport aiding in emissions export from high-emission zones. Yet, the synergy or competition between these systems is not fully understood. Our study employs optimization techniques to develop a cost-optimal European energy system, integrating transport, storage, and sequestration of both carbon dioxide and hydrogen, along with renewable energy sources. Results indicate that a hydrogen network is more cost-effective than a carbon network, facilitating low-cost hydrogen imports and local carbon capture. However, in a hybrid scenario, the carbon network effectively complements the hydrogen network, promoting carbon capture from biomass and reducing reliance on direct air capture. This research demonstrates the effectiveness of combining hydrogen and carbon transport networks with power grid expansion in achieving climate neutrality in Europe.

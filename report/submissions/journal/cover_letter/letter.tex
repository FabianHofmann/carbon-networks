% FDUletter_example.tex - an example latex file to illustrate FDUletter.cls
%
% Template by Brian Wood (brian.wood@oregonstate.edu).  Please feel free to send suggestions for changes; this template/cls is not exactly elegantly done!
% Modified by Huang Weiran (huangweiran1998@outlook.com) to fit the need of FDU students.
% Based on SCU version and delete the watermark
% SHU version modified by Zhendong Li (zhendong.li2001@outlook.com) to fit SHU students.

\documentclass[12pt]{SHUletter}
\usepackage{tikz}
\usepackage{eurosym}
\usepackage{xcolor}
	\definecolor{tubred}{RGB}{153,0,0}
\usepackage{lipsum}
\usepackage{fancyhdr}
\usepackage{lastpage}
\usepackage{eso-pic}
%
% This section is just a bunch of busywork so that the second and following pages read ``Page X of Y''
\pagestyle{fancy}
\fancyhf{}
\renewcommand{\headrulewidth}{0pt}
\renewcommand{\footrulewidth}{0pt}
\rhead{Page \thepage \hspace{1pt} of \pageref{LastPage}}
%
%
% Set custom font here. Comment this line out if you do not have a Cambria font (originally included with this template) installed; computer modern (or whatever your current default font is) will be substituted.
%
%\setmainfont{[Cambria.ttf]}[BoldFont  = [CambriaBold.ttf], ItalicFont  = [CambriaItalic.ttf], BoldItalicFont = [CambriaBoldItalic.ttf] ]

\newcommand{\watermark}[3]{\AddToShipoutPictureBG{
\parbox[b][\paperheight]{\paperwidth}{
\vfill%
\centering%
\begin{tikzpicture}
    \path (0,0) -- (\paperwidth,\paperheight);
    \node[opacity=.06] at (current page.center)
    {\includegraphics[width=0.55\textwidth]{tublogo.pdf.pdf}};   % sculogo watermark
    \end{tikzpicture}
\vfill}}}

% The material below is a whole big dang thing whose purpose is just to set up a fixed coordinate system for \tikz so that you can put the Department or School address in the upper right-hand side without it moving all around every time you change something in the page.  I think it works.
% Defining a new coordinate system for the page:
%
% --------------------------
% |(-1,1)    (0,1)    (1,1)|
% |                        |
% |(-1,0)    (0,0)    (1,0)|
% |                        |
% |(-1,-1)   (0,-1)  (1,-1)|
% --------------------------
\makeatletter
\def\parsecomma#1,#2\endparsecomma{\def\page@x{#1}\def\page@y{#2}}
\tikzdeclarecoordinatesystem{page}{
	\parsecomma#1\endparsecomma
	\pgfpointanchor{current page}{north east}
	% Save the upper right corner
	\pgf@xc=\pgf@x%
	\pgf@yc=\pgf@y%
	% save the lower left corner
	\pgfpointanchor{current page}{south west}
	\pgf@xb=\pgf@x%
	\pgf@yb=\pgf@y%
	% Transform to the correct placement
	\pgfmathparse{(\pgf@xc-\pgf@xb)/2.*\page@x+(\pgf@xc+\pgf@xb)/2.}
	\expandafter\pgf@x\expandafter=\pgfmathresult pt
	\pgfmathparse{(\pgf@yc-\pgf@yb)/2.*\page@y+(\pgf@yc+\pgf@yb)/2.}
	\expandafter\pgf@y\expandafter=\pgfmathresult pt
}
\makeatother


%%------------------------------------------------------------------------------%%
%

%%%%%%%%%%% Put Recommender(your mentor) Information Here %%%%%%%%%%%
%
\def\name{Dr. Fabian Hofmann,\\Christoph Tries,\\Dr. Fabian Neumann,\\Elisabeth Zeyen,\\Prof. Dr. Tom Brown.
}
%
% List your degree(s), licences, etc. here if you like.
%\def\What{, Your degrees, etc.}
%
% Set the name of your Department or School here
% I honestly don't know why the negative spacing is necessary to get the alignment of the first line correct.  This must be a ``\tikz'' thing.
%%%%%%%%%%%%%%%%%%  School or Department %%%%%%%%%%%%%%%
\def\Where{\hspace{-1.2mm}\textbf{\color{tubred}
	Department of Digital\\Transformation in Energy Systems,\\ TU Berlin
}}

%%%%%%%%%%%%  Additional Contact Information %%%%%%%%%%%
%
% Set your preferred primary contact address here.
\def\Address{Straße des 17. Juni 135
}
%
\def\CityZip{Berlin, 10623 \\
	Germany}
%
% Set your e-mail here
\def\Email{\textbf{\color{tubred}E-mail}: m.hofmann@tu-berlin.de}
%
% Set your preferred contact number here
%\def\TEL{\textbf{\color{tubred}Tel/Fax}: +86 28 85412720}
%
% Set your department or personal website here
\def\URL{\textbf{\color{tubred}URL}: {https://www.ensys.tu-berlin.de}}
%

%%%%%%%%%%%%%%%%  Footer information  %%%%%%%%%%%%%%%%%%
%
%  The next line is for your college, used as a footer.  If you prefer not to have this, just comment out these lines in favor of the line labeled "[[Alternate]]" below
\def\school{\small{
		TU Berlin $\cdot$
		~Department of Digital\\ Transformation in Energy Systems $\cdot$
		~25, Einsteinufer  $\cdot$
		~Berlin, Germany} }
% \def\school{~}  % [[Alternate]]
%

%%%%%%%%%%%%%%%%%%%%%  Signature line  %%%%%%%%%%%%%%%%%%%%%
%
% Set your signature line here.
% One can add a signature image in a PDF file using the following code; this requires a file called "signature_block.pdf" to be installed in the same folder as the .tex file.  The vertical spacing (\vspace) and the scaling will have to be adjusted to get things to look correct for your particular signature image. Alternatively, comment out the following line in favor of the one labeled "[[Alternate]]" if you want to sign a paper copy of the letter.
%
\signature{
	\name
}
%\signature{\name}  % [[Alternate]]
%%------------------------------------------------------------------------------%%



% This block sets up the address on the right-hand side of the header.
%
% The following lines just compile the information you set up into the LaTex letter variable "address" for later use.
%
%The following command "clears out" the default address so that it can be better set using \tikz
\address{}

\def\newaddress{
	\Where\\
	\Address\\
	\CityZip\\
	%\TEL\\
	\Email\\
	\URL
}
%
%%%%%%%%%%%  DATE  %%%%%%%%%%%%%%%%%%%%%%%%%
% If you want a date different from the current date, comment out the next line in favor of the line labeled "[[Alternate]]".  The ``\vspace{10mm}'' just moves the date down a tiny bit so it doesn't interfere with the header.  This can be adjusted to your preference.
%
\date{\vspace{11mm} \today}
%\date{\vspace{10mm} 20 September 2020}  %[[Alternate]]
%
%%%%%%%%%%% Set the subject here if there is one  %%%%
%\subject{Stuff} % optional subject line

\newcommand{\carbon}{CO$_2$}
\newcommand{\hydrogen}{H$_2$}


%%-----------------------------------------------------------------------------------------%%
%
\begin{document}
	%
	%
	%%%%%%%%  The "To" address goes here.
	%
   %Also, you can use this
	% \begin{letter}{
	% 		Target University\\
	% 		Some Addresss\\
	% 		SomeTown, SomeState 					       				  		 ~~SomeZip
	% 	}
 \begin{letter}{Dear Editor,}

		% This line sets up the return address to the right-side of the OSU logo.  The location is set with absolute node addresses using ``\tikz''.  It can still be a bit fussy, and you may need to alter this a little to get things to look right.  The bit that changes the position are the numbers in parentheses ``at (14.2,2.7)''
		%
		\begin{tikzpicture}[remember picture,overlay,,every node/.style={anchor=center}]
		\node[text width=7cm] at (page cs:0.5,0.73){\small \newaddress};
		\end{tikzpicture}

		%%%%%%  The ``opening'' is just the method of address you would like to use at the start of the letter.
		%
		\opening{}


		%%%%%%%%%% Body of letter   %%%%%%%%%%%%%%
		% Remove it if you do not want watermark
		%\watermark{}{}{}

		%%%%%%%%%%% 	Text   %%%%%%%%%%%%%%
		Thank you for the opportunity to submit our manuscript to Nature Energy. We are pleased to present our paper titled “\hydrogen{} and \carbon{} Network Strategies for the European Energy System”. This collaborative effort from researchers at the Institute of Energy Technology, Technical University of Berlin, contributes to the critical discourse on climate-neutral energy systems in Europe.

		In our study, we introduce the first large-scale investment model to analyze potential synergies and competition between hydrogen and carbon dioxide transport networks within a carbon-neutral European energy system.  Our model represents an advancement over existing models by optimizing a comprehensive range of technologies for the production, transportation, and storage of various energy carriers across all relevant sectors at high temporal and spatial resolution.

		The publication of this paper is timely, considering that policymakers and industry leaders are currently formulating plans and investments for hydrogen and carbon dioxide management systems. Our study offers critical insights and practical implications for the integration of synthetic fuel production and permanent carbon storage (carbon sequestration) into future energy systems.

		Our analysis not only confirms the role of hydrogen and carbon dioxide transport systems in providing crucial flexibilities to carbon-neutral energy systems. It also illustrates how their combination enhances energy efficiency and social welfare. We demonstrate that a hydrogen grid is optimally used to transport low-cost hydrogen from regions with abundant renewable resources to distributed carbon sources, such as industrial clusters and biomass combustion plants, to enable on-site synthetic fuel production. Conversely, a carbon dioxide network efficiently transports low-cost carbon from point sources near the coast to sequestration sites for permanent storage. Such an approach increase carbon capture rates from biomass combustion and reduces reliance on cost-intensive direct air capture. Furthermore, the benefits of deploying both networks are amplified under a net carbon-removal target.
		As underscored by our analysis, the integration of hydrogen and carbon dioxide networks into European energy policy is vital for a resilient, carbon-neutral, or negative future energy system.

		This manuscript has not been published nor is it under consideration elsewhere. It has not been previously discussed with any Nature Energy editor.

		We believe that our findings align with the scope of Nature Energy and will interest your readership. We appreciate the editor's time in considering our manuscript for publication.

		%%%%%%% ``closing'' sets the sign-off line.
		\closing{Sincerely,}
		% Comment out/in the lines below as necessary
		%\encl{If an enclosure is provided, let them know what it is.}
		%\ps{A postscript if that is a thing you do.}
		%\cc{Someone Who Cares (and is copied).}
	\end{letter}

\end{document}
%
%%-----------------------------------------------------------------------------------------%%

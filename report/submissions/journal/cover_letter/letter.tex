% FDUletter_example.tex - an example latex file to illustrate FDUletter.cls
%
% Template by Brian Wood (brian.wood@oregonstate.edu).  Please feel free to send suggestions for changes; this template/cls is not exactly elegantly done!
% Modified by Huang Weiran (huangweiran1998@outlook.com) to fit the need of FDU students.
% Based on SCU version and delete the watermark
% SHU version modified by Zhendong Li (zhendong.li2001@outlook.com) to fit SHU students.

\documentclass[12pt]{SHUletter}
\usepackage{tikz}
\usepackage{eurosym}
\usepackage{xcolor}
	\definecolor{tubred}{RGB}{153,0,0}
\usepackage{lipsum}
\usepackage{fancyhdr}
\usepackage{lastpage}
\usepackage{eso-pic}
%
% This section is just a bunch of busywork so that the second and following pages read ``Page X of Y''
\pagestyle{fancy}
\fancyhf{}
\renewcommand{\headrulewidth}{0pt}
\renewcommand{\footrulewidth}{0pt}
\rhead{Page \thepage \hspace{1pt} of \pageref{LastPage}}
%
%
% Set custom font here. Comment this line out if you do not have a Cambria font (originally included with this template) installed; computer modern (or whatever your current default font is) will be substituted.
%
%\setmainfont{[Cambria.ttf]}[BoldFont  = [CambriaBold.ttf], ItalicFont  = [CambriaItalic.ttf], BoldItalicFont = [CambriaBoldItalic.ttf] ]

\newcommand{\watermark}[3]{\AddToShipoutPictureBG{
\parbox[b][\paperheight]{\paperwidth}{
\vfill%
\centering%
\begin{tikzpicture}
    \path (0,0) -- (\paperwidth,\paperheight);
    \node[opacity=.06] at (current page.center)
    {\includegraphics[width=0.55\textwidth]{tublogo.pdf.pdf}};   % sculogo watermark
    \end{tikzpicture}
\vfill}}}

% The material below is a whole big dang thing whose purpose is just to set up a fixed coordinate system for \tikz so that you can put the Department or School address in the upper right-hand side without it moving all around every time you change something in the page.  I think it works.
% Defining a new coordinate system for the page:
%
% --------------------------
% |(-1,1)    (0,1)    (1,1)|
% |                        |
% |(-1,0)    (0,0)    (1,0)|
% |                        |
% |(-1,-1)   (0,-1)  (1,-1)|
% --------------------------
\makeatletter
\def\parsecomma#1,#2\endparsecomma{\def\page@x{#1}\def\page@y{#2}}
\tikzdeclarecoordinatesystem{page}{
	\parsecomma#1\endparsecomma
	\pgfpointanchor{current page}{north east}
	% Save the upper right corner
	\pgf@xc=\pgf@x%
	\pgf@yc=\pgf@y%
	% save the lower left corner
	\pgfpointanchor{current page}{south west}
	\pgf@xb=\pgf@x%
	\pgf@yb=\pgf@y%
	% Transform to the correct placement
	\pgfmathparse{(\pgf@xc-\pgf@xb)/2.*\page@x+(\pgf@xc+\pgf@xb)/2.}
	\expandafter\pgf@x\expandafter=\pgfmathresult pt
	\pgfmathparse{(\pgf@yc-\pgf@yb)/2.*\page@y+(\pgf@yc+\pgf@yb)/2.}
	\expandafter\pgf@y\expandafter=\pgfmathresult pt
}
\makeatother


%%------------------------------------------------------------------------------%%
%

%%%%%%%%%%% Put Recommender(your mentor) Information Here %%%%%%%%%%%
%
\def\name{Philipp Glaum \& Fabian Hofmann
}
%
% List your degree(s), licences, etc. here if you like.
%\def\What{, Your degrees, etc.}
%
% Set the name of your Department or School here
% I honestly don't know why the negative spacing is necessary to get the alignment of the first line correct.  This must be a ``\tikz'' thing.
%%%%%%%%%%%%%%%%%%  School or Department %%%%%%%%%%%%%%%
\def\Where{\hspace{-1.2mm}\textbf{\color{tubred}
	Department of Digital\\Transformation in Energy Systems,\\ TU Berlin
}}

%%%%%%%%%%%%  Additional Contact Information %%%%%%%%%%%
%
% Set your preferred primary contact address here.
\def\Address{Straße des 17. Juni 135
}
%
\def\CityZip{Berlin, 10623 \\
	Germany}
%
% Set your e-mail here
\def\Email{\textbf{\color{tubred}E-mail}: p.glaum@tu-berlin.de}
%
% Set your preferred contact number here
%\def\TEL{\textbf{\color{tubred}Tel/Fax}: +86 28 85412720}
%
% Set your department or personal website here
\def\URL{\textbf{\color{tubred}URL}: {https://www.ensys.tu-berlin.de}}
%

%%%%%%%%%%%%%%%%  Footer information  %%%%%%%%%%%%%%%%%%
%
%  The next line is for your college, used as a footer.  If you prefer not to have this, just comment out these lines in favor of the line labeled "[[Alternate]]" below
\def\school{\small{
		TU Berlin $\cdot$
		~Department of Digital\\ Transformation in Energy Systems $\cdot$
		~25, Einsteinufer  $\cdot$
		~Berlin, Germany} }
% \def\school{~}  % [[Alternate]]
%

%%%%%%%%%%%%%%%%%%%%%  Signature line  %%%%%%%%%%%%%%%%%%%%%
%
% Set your signature line here.
% One can add a signature image in a PDF file using the following code; this requires a file called "signature_block.pdf" to be installed in the same folder as the .tex file.  The vertical spacing (\vspace) and the scaling will have to be adjusted to get things to look correct for your particular signature image. Alternatively, comment out the following line in favor of the one labeled "[[Alternate]]" if you want to sign a paper copy of the letter.
%
\signature{
	\name
}
%\signature{\name}  % [[Alternate]]
%%------------------------------------------------------------------------------%%



% This block sets up the address on the right-hand side of the header.
%
% The following lines just compile the information you set up into the LaTex letter variable "address" for later use.
%
%The following command "clears out" the default address so that it can be better set using \tikz
\address{}

\def\newaddress{
	\Where\\
	\Address\\
	\CityZip\\
	%\TEL\\
	\Email\\
	\URL
}
%
%%%%%%%%%%%  DATE  %%%%%%%%%%%%%%%%%%%%%%%%%
% If you want a date different from the current date, comment out the next line in favor of the line labeled "[[Alternate]]".  The ``\vspace{10mm}'' just moves the date down a tiny bit so it doesn't interfere with the header.  This can be adjusted to your preference.
%
\date{\vspace{11mm} \today}
%\date{\vspace{10mm} 20 September 2020}  %[[Alternate]]
%
%%%%%%%%%%% Set the subject here if there is one  %%%%
%\subject{Stuff} % optional subject line



%%-----------------------------------------------------------------------------------------%%
%
\begin{document}
	%
	%
	%%%%%%%%  The "To" address goes here.
	%
   %Also, you can use this
	% \begin{letter}{
	% 		Target University\\
	% 		Some Addresss\\
	% 		SomeTown, SomeState 					       				  		 ~~SomeZip
	% 	}
 \begin{letter}{Dear Editor,}

		% This line sets up the return address to the right-side of the OSU logo.  The location is set with absolute node addresses using ``\tikz''.  It can still be a bit fussy, and you may need to alter this a little to get things to look right.  The bit that changes the position are the numbers in parentheses ``at (14.2,2.7)''
		%
		\begin{tikzpicture}[remember picture,overlay,,every node/.style={anchor=center}]
		\node[text width=7cm] at (page cs:0.5,0.73){\small \newaddress};
		\end{tikzpicture}

		%%%%%%  The ``opening'' is just the method of address you would like to use at the start of the letter.
		%
		\opening{}


		%%%%%%%%%% Body of letter   %%%%%%%%%%%%%%
		% Remove it if you do not want watermark
		%\watermark{}{}{}

		%%%%%%%%%%% 	Text   %%%%%%%%%%%%%%
		Thank you very much for the opportunity to submit our manuscript to Applied Energy. Hereby we submit the paper “Leveraging the Existing German Transmission Grid with Dynamic Line Rating”.

		In our work, we present the first large-scale investment model that evaluates the benefits of Dynamic Line Rating (DLR) for renewable energy integration.
		Taking Germany as an example, the model includes a high temporal and spatial resolution as well as a high level of detail to enable a realistic representation of the system dynamics.
		The paper appears at an important time as  politicians and Transmission System Operators (TSOs) have recognized the problem of lacking transmission capacities when moving towards high renewable shares in Europe and all over the world.
		% In many official reports DLR is considered as a promising measure to relief the transmission grid.

		Our analysis not only confirms and quantitatively describes transmission system benefits but also shows how the implementation of DLR affects the optimal system design for different shares of renewable power production. By better integrating wind power resources, especially offshore, DLR leads to significant cost savings of 5\% of the system cost, mainly due to the reduced need for cost-driving energy storage. This key finding can be generalized to all countries with normal to high wind resources and is therefore of particular importance for international network operators and the research community. Furthermore, we provide an open-source workflow which can be used to reproduce the results and extend the model to other countries.

		The paper was proofread by Prof. Tom Brown, an energy system expert and native tongue speaker, as well as from Rena Kuwahata from Ampacimon, an expert on transmission lines and DLR applications.

		We want to thank the editor for taking the time to consider the publication of our paper in Applied Energy.


		% What is the novelty of this work? Is the paper appealing to a popular or scientific audience?
		% Why the authors think the paper is important and why the journal should publish it? Has the article
		% been checked by a native tongue speaker with expertise in the field? Are you available as a reviewer
		% for at least three other articles for Applied Energy during the current year? In addition to answering
		% those five questions, the authors should also describe in one or two paragraphs the significance of
		% their work and what new information is described in the manuscript.


		%%%%%%% ``closing'' sets the sign-off line.
		\closing{Sincerely,}
		% Comment out/in the lines below as necessary
		%\encl{If an enclosure is provided, let them know what it is.}
		%\ps{A postscript if that is a thing you do.}
		%\cc{Someone Who Cares (and is copied).}
	\end{letter}

\end{document}
%
%%-----------------------------------------------------------------------------------------%%

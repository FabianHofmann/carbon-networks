Enabling a carbon-neutral European economy requires internationally coordinated investments into all sectors of the energy system. In the light of hard-to-abate emissions from process industry, energy-intensive transport and backup heat or power sources, it is important to determine the extent to which a system-wide carbon management is needed. Different technologies such as direct air capture, carbon capture at point sources and carbon transport, utilization in hydrocarbon synthesis and storage are considered as valuable contributors to formulate a comprehensive carbon strategy and take the final step to climate-neutrality.
While the literature has already showed the synergies of these technologies, the question of how to design the infrastructure at high resolution together with a CO$_2$ transport system has not been addressed so far. Moreover, there is a growing number of individually planned carbon management projects whereas the formulation of a European carbon strategy is still missing from the political agenda.
In this paper, we present the first study of a cost-optimal design of the European energy system with a full representation of carbon management technologies including a CO$_2$ transport network, considering all sectors with high spatial resolution. In addition, the system is allowed to optimize the layout of renewable energy sources as well as energy transmission and storage systems based on hydrogen and methane.
Assuming net-zero emissions, we show that carbon management technologies provide critical cost-beneficial flexibilities to the European energy system. In particular, we show that building a CO$_2$ network complements the built out of a hydrogen network and is cost-effective across different rates of carbon sequestration in the offshore regions (100 -- 1000 Mt/a). Combined with a hydrogen grid, the CO$_2$ grid leads to cost savings of about 6\%, as it allows industrial and densely populated inland sites to source hydrogen from coastal regions with favorable renewable energy potential and, in return, supply CO$_2$ that can be sequestered or utilized for synthetic hydrocarbons for petrochemical feedstocks, aviation and shipping. At high sequestration rates, direct air capture facilities complement carbon capture technologies at point sources, leading however to higher use of fossil energy carriers in the power and heating sector. The optimization identifies three main carbon sinks that dominate and determine the CO$_2$ network design, namely a large sequestration cluster in the North Sea and two clusters on the coasts of Portugal and Greece.
The results highlight the potential of a comprehensive carbon management strategy as a basis for designing future carbon networks and identifying the most promising sites for carbon capture and storage.

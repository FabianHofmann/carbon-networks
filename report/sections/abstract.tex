Achieving a carbon-neutral European economy requires internationally coordinated efforts across all energy sectors, especially those with hard-to-abate emissions like heavy industry. Against this backdrop, It is critical to assess the need for comprehensive carbon management involving technologies such as carbon capture, transport, use, and storage to develop an effective carbon strategy for climate neutrality.

This paper provides the first cost-optimal design for a European energy system that fully incorporates carbon management technologies including a carbon transport network, and considers all emissions-intensive sectors. It optimizes carbon technologies, hydrogen transport and storage, and the design of renewable energy sources.

We show that carbon management technologies aiming at net-zero emissions provide systemic flexibility to the European energy system. A carbon network is cost-effective at different carbon capture rates (200 - 1000 Mt/a), leading to average cost savings of €13 billion per year. The \carbon{} network enables viable point source carbon capture, reducing the need for direct air capture facilities. We show that high sequestration rates are cost-beneficial by allowing continued fossil fuel use in some sectors, but require massive capacity expansion which could jeopardize timely attainment of climate neutrality.

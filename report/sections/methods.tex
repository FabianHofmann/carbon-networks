To model the detailed expansion of the carbon network, we use the open-source PyPSA-Eur-Sec~\cite{PyPSAEurSecSectorCoupledOpen2023} model. This is a fully sector-coupled model with high spatial and temporal resolution and detailed transmission infrastructure for electricity, hydrogen, methane, and carbon networks. The model enables joint optimization of investment and operational decisions for generation, storage, conversion, and transmission infrastructures. For our study, we use an approach with 181~geographic regions for Europe and a temporal resolution of 3~hours. These settings allow us to detail the future carbon network and capture the dynamics of material flow between industrial carbon sources and carbon sinks for both use and storage.

PyPSA-Eur-Sec uses exogenous assumptions for the regional energy demand of the power sector, industry, transport, buildings and agriculture. This includes energy demand for shipping and aviation, and non-energy feedstocks for the chemical industry. To take account of climate neutrality targets, we set a binding cap on zero emissions in 2050. Biomass use in each region is limited by its biomass potential, but biomass transport between regions is allowed and the model accounts for transport costs. Furthermore, we limit the expansion of the electric transmission system to 50\% of today's total capacity to account for the difficulties in placing new transmission projects. The geographical layout of process emissions from industry is derived from~\cite{piamanzGeoreferencedIndustrialSites2018}. For carbon sequestration, only offshore sites are considered as potential sinks (see Figure~\ref{fig:carbonSequestrationPotentials}). This is partly because offshore storage potential tends to be greater than onshore potential, and partly because carbon storage infrastructure projects near population centers may be difficult to implement due to public safety concerns. We make conservative assumptions in estimating storage potential by truncating the total potential to 25~Mt per site and calculate annual storage availability when filled over 25 years.
On the carbon utilization side, we include three drop-in fuel production technologies in our model: Methanation, Methanolization, and Fischer-Tropsch (FT) synthesis. For the fuels to be considered carbon-neutral substitutes, they must be produced using carbon from a technology with negative emissions. Synthetic methane can be used as a substitute for natural gas or biogas and is used in combined heat and power (CHP) plants, in gas boilers for home heating, or to meet the gas needs of industry. Synthetic methanol is used in the model to decarbonize fuel demand in the marine industry. Finally, FT fuels can replace fossil oil in the production of naphtha for industry and kerosene for aviation, or be used as machinery oil for agriculture.
All technology cost assumptions for 2050 are listed taken from~\cite{lisazeyenPyPSATechnologydataTechnology2023}.

To explore the impact of carbon sequestration on the optimal layout of carbon management technologies, we vary the global carbon sequestration potential from 200~Mt to 1000~Mt per year in steps of 200~Mt.

% \begin{figure}
%     \centering
%     \includegraphics[width=\linewidth]{sequestration_map.pdf}
%     \caption{Sequestration map}
%     \label{fig:sequestration_map}
% \end{figure}


% \begin{figure}[h]
%     \centering
%     \includegraphics[width=\linewidth]{capacity_map_electricity_co2network_1000.pdf}
%     \caption{Optimal capacities per of the electricity sector for a sequestration of 1000 Mt/a.}
%     \label{fig:capacity_map_noco2network_1000}
% \end{figure}


% \begin{figure}[h]
%     \centering
%     \includegraphics[width=\linewidth]{capacity_map_carbon_co2network_1000.pdf}
%     \caption{Optimal capacities per of the carbon sector for a sequestration of 1000 Mt/a.}
%     \label{fig:capacity_map_noco2network_1000}
% \end{figure}



% \begin{figure*}
%     \centering
%     \includegraphics[width=\linewidth]{balance_map_carbon_co2network_1000.pdf}
%     \caption{Optimal operation per sector for a sequestration of 1000 Mt/a.}
%     \label{fig:balance_map_noco2network_1000}
% \end{figure*}


% \begin{figure}
%     \centering
%     \includegraphics[width=\linewidth]{operation_area_carbon_co2network.pdf}
%     \caption{Balance of captured CO$_2$ emissions for the optimal operation with CO$_2$ network.}
%     \label{fig:operation_area_carbon_co2network}
% \end{figure}

% \begin{figure}
%     \centering
%     \includegraphics[width=\linewidth]{operation_area_carbon_noco2network.pdf}
%     \caption{Balance of captured CO$_2$ emissions for the optimal operation without CO$_2$ network.}
%     \label{fig:operation_area_carbon_noco2network}
% \end{figure}


% \begin{figure}
%     \centering
%     \includegraphics[width=\linewidth]{operation_bar_co2network_200.pdf}
% \end{figure}
